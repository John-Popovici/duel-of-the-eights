\documentclass{article}

\usepackage{tabularx}
\usepackage{booktabs}

\title{Reflection and Traceability Report on \progname}

\author{\authname}

\date{}

%% Comments

\usepackage{color}

\newif\ifcomments\commentstrue %displays comments
%\newif\ifcomments\commentsfalse %so that comments do not display

\ifcomments
\newcommand{\authornote}[3]{\textcolor{#1}{[#3 ---#2]}}
\newcommand{\todo}[1]{\textcolor{red}{[TODO: #1]}}
\else
\newcommand{\authornote}[3]{}
\newcommand{\todo}[1]{}
\fi

\newcommand{\wss}[1]{\authornote{blue}{SS}{#1}} 
\newcommand{\plt}[1]{\authornote{magenta}{TPLT}{#1}} %For explanation of the template
\newcommand{\an}[1]{\authornote{cyan}{Author}{#1}}

%% Common Parts

\newcommand{\progname}{SFWRENG 4G06} % PUT YOUR PROGRAM NAME HERE
\newcommand{\authname}{Team 9, dice\_devs
\\ John Popovici
\\ Nigel Moses
\\ Naishan Guo
\\ Hemraj Bhatt
\\ Isaac Giles} % AUTHOR NAMES                  

\usepackage{hyperref}
    \hypersetup{colorlinks=true, linkcolor=blue, citecolor=blue, filecolor=blue,
                urlcolor=blue, unicode=false}
    \urlstyle{same}
                                


\begin{document}

\maketitle

\plt{Reflection is an important component of getting the full benefits from a
learning experience.  Besides the intrinsic benefits of reflection, this
document will be used to help the TAs grade how well your team responded to
feedback.  Therefore, traceability between Revision 0 and Revision 1 is and
important part of the reflection exercise.  In addition, several CEAB (Canadian
Engineering Accreditation Board) Learning Outcomes (LOs) will be assessed based
on your reflections.}

\section{Changes in Response to Feedback}

\plt{Summarize the changes made over the course of the project in response to
feedback from TAs, the instructor, teammates, other teams, the project
supervisor (if present), and from user testers.}

\plt{For those teams with an external supervisor, please highlight how the feedback 
from the supervisor shaped your project.  In particular, you should highlight the 
supervisor's response to your Rev 0 demonstration to them.}

\plt{Version control can make the summary relatively easy, if you used issues
and meaningful commits.  If you feedback is in an issue, and you responded in
the issue tracker, you can point to the issue as part of explaining your
changes.  If addressing the issue required changes to code or documentation, you
can point to the specific commit that made the changes.  Although the links are
helpful for the details, you should include a label for each item of feedback so
that the reader has an idea of what each item is about without the need to click
on everything to find out.}

\plt{If you were not organized with your commits, traceability between feedback
and commits will not be feasible to capture after the fact.  You will instead
need to spend time writing down a summary of the changes made in response to
each item of feedback.}

\plt{You should address EVERY item of feedback.  A table or itemized list is
recommended.  You should record every item of feedback, along with the source of
that feedback and the change you made in response to that feedback.  The
response can be a change to your documentation, code, or development process.
The response can also be the reason why no changes were made in response to the
feedback.  To make this information manageable, you will record the feedback and
response separately for each deliverable in the sections that follow.}

\plt{If the feedback is general or incomplete, the TA (or instructor) will not
be able to grade your response to feedback.  In that case your grade on this
document, and likely the Revision 1 versions of the other documents will be
low.} 

\subsection{Instructor Feedback}

\begin{itemize}
    \item \textbf{Issue:} Team Contributions
    \begin{itemize}
        \item \textbf{Feedback:} Commit numbers are not even across group members, others appear to be contributing in other ways but either commit numbers need to be more even, or alternate contribution methods must be better documented
        \item \textbf{Response:} Team members have worked hard to ensure that commit numbers throughout the last third of the course have been much more consistent across the board, and work that is not captured by commits has been documented in project management contributions files. 
    \end{itemize}
\end{itemize}

\subsection{Supervisor Feedback}


\subsection{TA Feedback}

\begin{itemize}
    \item \textbf{Issue:} Interactive Lobby
    \begin{itemize}
        \item \textbf{Feedback:} Request for social hub to facilitate matchmaking
        \item \textbf{Response:} Added to post-MVP road-map since matchmaking was pushed out of scope
    \end{itemize}

    \item \textbf{Issue:} Team Contributions
    \begin{itemize}
        \item \textbf{Feedback:} Commit numbers are not even across group members, try to even out commits more
        \item \textbf{Response:} Team members have picked up open issues when nothing is assigned, to ensure work is consistently available for all members. All members have worked hard to ensure that commit numbers throughout the last third of the course have been much more consistent across the board. 
    \end{itemize}
\end{itemize}


\subsection{Usability Testing Feedback}

\subsubsection{UI/UX Improvements}
\begin{itemize}
    \item \textbf{Issue:} LineEdit Text Validation
    \begin{itemize}
        \item \textbf{Feedback:} No input length restrictions
        \item \textbf{Response:} Implemented SMS-length chat (160 chars) and connection code limits
    \end{itemize}
    
    \item \textbf{Issue:} LineEdit Text Overlap
    \begin{itemize}
        \item \textbf{Feedback:} Text overflowed UI borders
        \item \textbf{Response:} Adjusted margins and text wrapping
    \end{itemize}
    
    \item \textbf{Issue:} Scoreboard Size Changes
    \begin{itemize}
        \item \textbf{Feedback:} Scoreboard changes size when hovering to view points
        \item \textbf{Response:} Fixed layout constraints to avoid sizing issues
    \end{itemize}
    
    \item \textbf{Issue:} Blurry UI Scaling
    \begin{itemize}
        \item \textbf{Feedback:} Pixelation was noticed with window scaling
        \item \textbf{Response:} Implemented nearest-neighbor scaling for bitmap elements
    \end{itemize}
    
    \item \textbf{Issue:} Checkbox Formatting
    \begin{itemize}
        \item \textbf{Feedback:} Unclear state indicators (check boxes) used throughout game
        \item \textbf{Response:} Redesign with better labeling (\textbf{planned})
    \end{itemize}
\end{itemize}

\subsubsection{Gameplay Enhancements}
\begin{itemize}
    \item \textbf{Issue:} Round Transition Feedback
    \begin{itemize}
        \item \textbf{Feedback:} Unclear progression from last roll of hand to next round
        \item \textbf{Response:} Add visual/audio cues and round summaries (\textbf{planned})
    \end{itemize}
    
    \item \textbf{Issue:} Scoring Explanations
    \begin{itemize}
        \item \textbf{Feedback:} Needed in-game hand guides to explain scoring
        \item \textbf{Response:} Created interactive tutorial with dice examples
    \end{itemize}
    
    \item \textbf{Issue:} Bonus Threshold Clarity
    \begin{itemize}
        \item \textbf{Feedback:} Unclear requirements for attaining the bonus
        \item \textbf{Response:} Add live tracking and tooltips (\textbf{planned})
    \end{itemize}
\end{itemize}

\subsubsection{Multiplayer Features}
\begin{itemize}
    \item \textbf{Issue:} Opponent Name Integration
    \begin{itemize}
        \item \textbf{Feedback:} Generic "Opponent" labels get used in chat
        \item \textbf{Response:} Implement actual profile name usage in chat (\textbf{planned})
    \end{itemize}
    
    \item \textbf{Issue:} Chat During Waiting Screen
    \begin{itemize}
        \item \textbf{Feedback:} Inaccessible chat during time when waiting for other user to roll
        \item \textbf{Response:} Layer reordering for UI access to chat feature throughout round (\textbf{planned})
    \end{itemize}
    
    \item \textbf{Issue:} Interactive Lobby
    \begin{itemize}
        \item \textbf{Feedback:} Request for social hub to facilitate matchmaking
        \item \textbf{Response:} Added to post-MVP road-map
    \end{itemize}
    
    \item \textbf{Issue:} Connection Status Clarity
    \begin{itemize}
        \item \textbf{Feedback:} Not sufficiently clear when client player connects to the host lobby
        \item \textbf{Response:} Add SFX and status indicators to show connection (\textbf{planned})
    \end{itemize}
\end{itemize}

\subsubsection{Technical Improvements}
\begin{itemize}
    \item \textbf{Issue:} Profile Persistence
    \begin{itemize}
        \item \textbf{Feedback:} Profile data is not persisting across multiple openings of the game
        \item \textbf{Response:} Implemented new file storage location that allows for data persistence when the game gets packed into an executable
    \end{itemize}
    
    \item \textbf{Issue:} Round Count Mismatch
    \begin{itemize}
        \item \textbf{Feedback:} D4 preset has 1 too many rounds and gets stuck on last round with no playable hands left to select
        \item \textbf{Response:} Reduced number of rounds associated with preset
    \end{itemize}
    
    \item \textbf{Issue:} Debug Module
    \begin{itemize}
        \item \textbf{Feedback:} Needed logging system to better explain issues appearing during usability testing
        \item \textbf{Response:} Created auto-loaded debug manager
    \end{itemize}
\end{itemize}

\subsubsection{Art/Animation}
\begin{itemize}
    \item \textbf{Issue:} Tavern Animation Sync
    \begin{itemize}
        \item \textbf{Feedback:} Tavern theme animations are out of sync
        \item \textbf{Response:} Re-sync tavern animation for dice bouncing
    \end{itemize}
    
    \item \textbf{Issue:} Dice Contrast
    \begin{itemize}
        \item \textbf{Feedback:} D6 dice UI not contrasted enough, creating poor visibility
        \item \textbf{Response:} Updated color schemes to improve visibility
    \end{itemize}
    
    \item \textbf{Issue:} D4 Model Replacement
    \begin{itemize}
        \item \textbf{Feedback:} Confusing layout of current D4 model
        \item \textbf{Response:} Add alternate D4 model that is easier to interpret (\textbf{planned})
    \end{itemize}
\end{itemize}

\subsubsection{Audio Improvements}
\begin{itemize}
    \item \textbf{Issue:} Music Variety
    \begin{itemize}
        \item \textbf{Feedback:} Music gets too repetitive
        \item \textbf{Response:} Added more music variety to game to reduce the frequency at which users hear the same audio
    \end{itemize}
\end{itemize}

\subsubsection{Tutorial System}
\begin{itemize}
    \item \textbf{Issue:} Tutorial Tabs
    \begin{itemize}
        \item \textbf{Feedback:} Tutorial section gets hard to navigate for different game variants
        \item \textbf{Response:} Add tabs to tutorial section for each variant of the game (\textbf{planned})
    \end{itemize}
    
    \item \textbf{Issue:} Update Tutorial Theming
    \begin{itemize}
        \item \textbf{Feedback:} Tutorial does not match current game theme
        \item \textbf{Response:} Updated tutorial to match most recent theming of the game
    \end{itemize}
\end{itemize}

\subsubsection{Network Features}
\begin{itemize}
    \item \textbf{Issue:} Client Name Transfer
    \begin{itemize}
        \item \textbf{Feedback:} Client name does not get passed to host when connecting
        \item \textbf{Response:} Should be updated to use client's name that gets passed (\textbf{planned})
    \end{itemize}
    
    \item \textbf{Issue:} End Game Chat Visibility
    \begin{itemize}
        \item \textbf{Feedback:} Chat should not remain visible on end screen
        \item \textbf{Response:} New functionality added to hide chat when game ends
    \end{itemize}
\end{itemize}

\subsubsection{Scoreboard System}
\begin{itemize}
    \item \textbf{Issue:} Scoreboard Scroll Clarity
    \begin{itemize}
        \item \textbf{Feedback:} Unclear that scoreboard can be scrolled, and how far the scrolling goes
        \item \textbf{Response:} Enhanced scrollbar theming to make scrollbar and handle more prominent (\textbf{planned})
    \end{itemize}
    
    \item \textbf{Issue:} Scoreboard Hover Explanations
    \begin{itemize}
        \item \textbf{Feedback:} Scoring explanation should be added for each scoring option when hovering
        \item \textbf{Response:} Opted to create a scoring quick guide that explains each scoring option
    \end{itemize}
\end{itemize}

\subsubsection{Player Differentiation}
\begin{itemize}
    \item \textbf{Issue:} Opponent Dice Visibility
    \begin{itemize}
        \item \textbf{Feedback:} Confusing which dice UI represents your own dice and which represents the opponents
        \item \textbf{Response:} Added grey overlay over opponent dice UI and made UI box smaller to indicate it is not interactable
    \end{itemize}
    
    \item \textbf{Issue:} Tab Screen Overlay
    \begin{itemize}
        \item \textbf{Feedback:} Pressing tab should show both player's scores as an overlay
        \item \textbf{Response:} Pressing tab show's both player's scores as an overlay (\textbf{planned})
    \end{itemize}
\end{itemize}


\subsection{SRS and Hazard Analysis}

\subsection{Design and Design Documentation}

\subsection{VnV Plan and Report}

\section{Challenge Level and Extras}

\subsection{Challenge Level}

\plt{State the challenge level (advanced, general, basic) for your project.  Your challenge level should exactly match what is included in your problem statement.  This should be the challenge level agreed on between you and the course instructor.}

\subsection{Extras}

\plt{Summarize the extras (if any) that were tackled by this project.  Extras
can include usability testing, code walkthroughs, user documentation, formal
proof, GenderMag personas, Design Thinking, etc.  Extras should have already
been approved by the course instructor as included in your problem statement.}

\section{Design Iteration (LO11 (PrototypeIterate))}

\plt{Explain how you arrived at your final design and implementation.  How did
the design evolve from the first version to the final version?} 

\plt{Don't just say what you changed, say why you changed it.  The needs of the
client should be part of the explanation.  For example, if you made changes in
response to usability testing, explain what the testing found and what changes
it led to.}

\section{Design Decisions (LO12)}

\plt{Reflect and justify your design decisions.  How did limitations,
 assumptions, and constraints influence your decisions?  Discuss each of these
 separately.}

\section{Economic Considerations (LO23)}

\plt{Is there a market for your product? What would be involved in marketing your 
product? What is your estimate of the cost to produce a version that you could 
sell?  What would you charge for your product?  How many units would you have to 
sell to make money? If your product isn't something that would be sold, like an 
open source project, how would you go about attracting users?  How many potential 
users currently exist?}

\section{Reflection on Project Management (LO24)}

\plt{This question focuses on processes and tools used for project management.}

\subsection{How Does Your Project Management Compare to Your Development Plan}

\plt{Did you follow your Development plan, with respect to the team meeting plan, 
team communication plan, team member roles and workflow plan.  Did you use the 
technology you planned on using?}

\subsection{What Went Well?}

\plt{What went well for your project management in terms of processes and 
technology?}

\subsection{What Went Wrong?}

\plt{What went wrong in terms of processes and technology?}

\subsection{What Would you Do Differently Next Time?}

\plt{What will you do differently for your next project?}

\section{Reflection on Capstone}

\plt{This question focuses on what you learned during the course of the capstone project.}

\subsection{Which Courses Were Relevant}

\plt{Which of the courses you have taken were relevant for the capstone project?}

\subsection{Knowledge/Skills Outside of Courses}

\plt{What skills/knowledge did you need to acquire for your capstone project
that was outside of the courses you took?}

\end{document}