\documentclass{article}

\usepackage{float}
\restylefloat{table}

\usepackage{booktabs}
\usepackage{pgf}
\usepackage{graphicx}
\graphicspath{ {./images/} }

\title{Team Contributions: Final\\\progname}

\author{\authname}

\date{}

\input{../Comments}
%% Common Parts

\newcommand{\progname}{SFWRENG 4G06} % PUT YOUR PROGRAM NAME HERE
\newcommand{\authname}{Team 9, dice\_devs
\\ John Popovici
\\ Nigel Moses
\\ Naishan Guo
\\ Hemraj Bhatt
\\ Isaac Giles} % AUTHOR NAMES                  

\usepackage{hyperref}
    \hypersetup{colorlinks=true, linkcolor=blue, citecolor=blue, filecolor=blue,
                urlcolor=blue, unicode=false}
    \urlstyle{same}
                                


\begin{document}

\maketitle

This document summarizes the contributions of each team member for the final
demonstration and documentation.  The time period of interest is the time
between Rev 0 and the Final documentation.

\section{Team Meeting Attendance}

% \wss{For each team member how many team meetings have they attended over the time period of interest.  This number should be determined from the meeting issues in the team's repo.  The first entry in the table should be the total number of team meetings held by the team.}

\begin{table}[H]
\centering
\begin{tabular}{ll}
\toprule
\textbf{Student} & \textbf{Meetings}\\
\midrule
Total & X\\
John P. & X\\
Nigel M. & X\\
Naishan G. & X\\
Isaac G. & X\\
Hemraj B. & X\\
\bottomrule
\end{tabular}
\end{table}

% \wss{If needed, an explanation for the counts can be provided here.}

\section{Supervisor/Stakeholder Meeting Attendance}

% \wss{For each team member how many supervisor/stakeholder team meetings have they attended over the time period of interest.  This number should be determined from the supervisor meeting issues in the team's repo.  The first entry in the table should be the total number of supervisor and team meetings held by the team.  If there is no supervisor, there will usually be meetings with stakeholders (potential users) that can serve a similar purpose.}

\begin{table}[H]
\centering
\begin{tabular}{ll}
\toprule
\textbf{Student} & \textbf{Meetings}\\
\midrule
Total & X\\
John P. & X\\
Nigel M. & X\\
Naishan G. & X\\
Isaac G. & X\\
Hemraj B. & X\\
\bottomrule
\end{tabular}
\end{table}

% \wss{If needed, an explanation for the counts can be provided here.}

\section{Lecture Attendance}

% \wss{For each team member how many lectures have they attended over the time period of interest.  This number should be determined from the lecture issues in the team's repo.  The first entry in the table should be the total number of lectures since the beginning of the term.}

\begin{table}[H]
\centering
\begin{tabular}{ll}
\toprule
\textbf{Student} & \textbf{Lectures}\\
\midrule
Total & X\\
John P. & X\\
Nigel M. & X\\
Naishan G. & X\\
Isaac G. & X\\
Hemraj B. & X\\
\bottomrule
\end{tabular}
\end{table}

% \wss{If needed, an explanation for the lecture attendance can be provided here.}

\section{TA Document Discussion Attendance}

% \wss{For each team member how many of the informal document discussion meetings with the TA were attended over the time period of interest.}

\begin{table}[H]
\centering
\begin{tabular}{ll}
\toprule
\textbf{Student} & \textbf{Lectures}\\
\midrule
Total & X\\
John P. & X\\
Nigel M. & X\\
Naishan G. & X\\
Isaac G. & X\\
Hemraj B. & X\\
\bottomrule
\end{tabular}
\end{table}

% \wss{If needed, an explanation for the attendance can be provided here.}

\section{Commits}

% \wss{For each team member how many commits to the main branch have been made over the time period of interest.  The total is the total number of commits for the entire team since the beginning of the term.  The percentage is the percentage of the total commits made by each team member.}

%TODO
% The number of commits
\pgfmathsetmacro{\CJ}{1}
\pgfmathsetmacro{\CN}{1}
\pgfmathsetmacro{\CNS}{1}
\pgfmathsetmacro{\CI}{1}
\pgfmathsetmacro{\CH}{1}

% Calculate the total
\pgfmathsetmacro{\CT}{\CJ + \CN + \CNS + \CH + \CI}

% Calculate the percentages
\pgfmathsetmacro{\PJ}{\CJ / \CT * 100}
\pgfmathsetmacro{\PN}{\CN / \CT * 100}
\pgfmathsetmacro{\PNS}{\CNS / \CT * 100}
\pgfmathsetmacro{\PI}{\CI / \CT * 100}
\pgfmathsetmacro{\PH}{\CH / \CT * 100}

\begin{table}[H]
\centering
\begin{tabular}{lll}
\toprule
\textbf{Student} & \textbf{Commits} & \textbf{Percent}\\
\midrule
Total & \pgfmathprintnumber[fixed,precision=0]\CT & 100\% \\
\midrule
John P. & \CJ & \pgfmathprintnumber[fixed,precision=2]\PJ\% \\
Nigel M. & \CN & \pgfmathprintnumber[fixed,precision=2]\PN\% \\
Naishan G. & \CNS & \pgfmathprintnumber[fixed,precision=2]\PNS\% \\
Isaac G. & \CI & \pgfmathprintnumber[fixed,precision=2]\PI\% \\
Hemraj B. & \CH & \pgfmathprintnumber[fixed,precision=2]\PH\% \\
\bottomrule
\end{tabular}
\end{table}

%TODO
% \wss{If needed, an explanation for the counts can be provided here.  For instance, if a team member has more commits to unmerged branches, these numbers can be provided here.  If multiple people contribute to a commit, git allows for multi-author commits.}

\section{Issue Tracker}

% \wss{For each team member how many issues have they authored (including open and closed issues (O+C)) and how many have they been assigned (only counting closed issues (C only)) over the time period of interest.}

\begin{table}[H]
\centering
\begin{tabular}{lll}
\toprule
\textbf{Student} & \textbf{Authored (O+C)} & \textbf{Assigned (C only)}\\
\midrule
John P. & X & X \\
Nigel M. & X & X \\
Naishan G. & X & X \\
Isaac G. & X & X \\
Hemraj B. & X & X \\
\bottomrule
\end{tabular}
\end{table}

% \wss{If needed, an explanation for the counts can be provided here.}

% \wss{If your team has additional metrics of productivity, please feel free to add them to this report.}

\end{document}