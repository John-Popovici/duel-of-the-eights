\documentclass{article}

\usepackage{float}
\restylefloat{table}

\usepackage{booktabs}
\usepackage{pgf}
\usepackage{graphicx}
\graphicspath{ {./images/} }

\title{Team Contributions: Rev 0\\\progname}

\author{\authname}

\date{}

\input{../Comments}
%% Common Parts

\newcommand{\progname}{SFWRENG 4G06} % PUT YOUR PROGRAM NAME HERE
\newcommand{\authname}{Team 9, dice\_devs
\\ John Popovici
\\ Nigel Moses
\\ Naishan Guo
\\ Hemraj Bhatt
\\ Isaac Giles} % AUTHOR NAMES                  

\usepackage{hyperref}
    \hypersetup{colorlinks=true, linkcolor=blue, citecolor=blue, filecolor=blue,
                urlcolor=blue, unicode=false}
    \urlstyle{same}
                                


\begin{document}

\maketitle

This document summarizes the contributions of each team member for the Rev 0
Demo.  The time period of interest is the time between the POC demo and the Rev
0 demo. This includes the week of 2024-11-24 up to 2025-01-29.

\newpage
\section{Demo Plans}

\noindent The team will demonstrate the refinement of the following key components of the system during the POC demonstration, as well as the addition of new features, including: 

\begin{enumerate}
    \item \textbf{Game Setup and Customization:}
    \begin{itemize}
        \item Demonstrate how users can set up a new local area network multiplayer game.
        \item Showcase customization of gameplay attributes, such as adjusting the number of dice and player health.
    		\item Showcase the new pre-made and specialized game variants. These are streamlined versions of the custom game with limited customization options, allowing each variant to implement distinct gameplay styles. 
        \item Through this, the modularity of the system will be displayed.
    \end{itemize}
    
    \item \textbf{Gameplay Mechanics:}
    \begin{itemize}
        \item Conduct a walkthrough of a round of gameplay, highlighting how players reroll the dice and accumulate points.
        \item Explain the scoring rules and demonstrate how player points are added.
        \item Demonstrate the new Bluffing Feature,which adds a new layer of strategy to the game that expands player interaction.
        \item Showcase the implementation of the Usability features suggested from prior feedback and testing.
        \item This will showcase the basic game flow.
        
    \end{itemize}
    
    \item \textbf{Game State and Progression:}
    \begin{itemize}
        \item Showcase how the game state is saved, i.e. how the game tracks and displays each player's current dice and hand points.
        \item Demonstrate the endgame conditions, illustrating what happens when all rounds are played and points are tallied.
        \item This will show how the system preserves game data.
    \end{itemize}
    
    \item \textbf{Multiplayer Mode:}
    \begin{itemize}
    		\item Showcase our implementation of a Server, allowing players to connect with each other over the Wide Area Network (WAN). (note that while our server hasn't been integrated into the playable game yet, we have plans to do so in the future. ) 
        \item Demonstrate how players take turns simultaneously and show the tracking of player dice for both players.
        \item This will highlight how data is synchronized between both players and how game integrity is preserved.
        
    \end{itemize}
    
    \item \textbf{Error Handling and Edge Cases:}
    \begin{itemize}
        \item Showcase implemented safeguards, such as preventing invalid moves and handling unexpected inputs through demonstrating typical game actions.
        \item Demonstrate the system's response in case of such scenarios.
        \item This will showcase the rigidity and stability of the system.
    \end{itemize}
\end{enumerate}

\newpage
\section{Team Meeting Attendance}

\wss{For each team member how many team meetings have they attended over the
time period of interest.  This number should be determined from the meeting
issues in the team's repo.  The first entry in the table should be the total
number of team meetings held by the team.}

% track starting week of 2024-11-24
% is:issue label:meeting
\begin{table}[H]
\centering
\begin{tabular}{ll}
\toprule
\textbf{Student} & \textbf{Meetings}\\
\midrule
Total & 7\\
John P. & 6\\
Nigel M. & 6\\
Naishan G. & 7\\
Isaac G. & 6\\
Hemraj B. & 6\\
\bottomrule
\end{tabular}
\end{table}

\wss{If needed, an explanation for the counts can be provided here.}

\newpage
\section{Supervisor/Stakeholder Meeting Attendance}

\wss{For each team member how many supervisor/stakeholder team meetings have
they attended over the time period of interest.  This number should be determined
from the supervisor meeting issues in the team's repo.  The first entry in the
table should be the total number of supervisor and team meetings held by the
team.  If there is no supervisor, there will usually be meetings with
stakeholders (potential users) that can serve a similar purpose.}


\begin{table}[H]
\centering
\begin{tabular}{ll}
\toprule
\textbf{Student} & \textbf{Meetings}\\
\midrule
Total & 0\\
John P. & 0\\
Nigel M. & 0\\
Naishan G. & 0\\
Isaac G. & 0\\
Hemraj B. & 0\\
\bottomrule
\end{tabular}
\end{table}

\wss{If needed, an explanation for the counts can be provided here.}

Multiple rounds of communication were had, back and fourth, but there was no official meeting. One is planned for between the submission due date of this report and the Rev0 demonstration.

\newpage
\section{Lecture Attendance}

\wss{For each team member how many lectures have they attended over the time
period of interest.  This number should be determined from the lecture issues in
the team's repo.  The first entry in the table should be the total number of
lectures since the beginning of the term.}

\begin{table}[H]
\centering
\begin{tabular}{ll}
\toprule
\textbf{Student} & \textbf{Lectures}\\
\midrule
Total & 2\\
John P. & 1\\
Nigel M. & 1\\
Naishan G. & 2\\
Isaac G. & 0\\
Hemraj B. & 0\\
\bottomrule
\end{tabular}
\end{table}

\wss{If needed, an explanation for the lecture attendance can be provided here.}

\newpage
\section{TA Document Discussion Attendance}

\wss{For each team member how many of the informal document discussion meetings
with the TA were attended over the time period of interest.}

\begin{table}[H]
\centering
\begin{tabular}{ll}
\toprule
\textbf{Student} & \textbf{Lectures}\\
\midrule
Total & 1\\
John P. & 1\\
Nigel M. & 1\\
Naishan G. & 1\\
Isaac G. & 1\\
Hemraj B. & 0\\
\bottomrule
\end{tabular}
\end{table}

\wss{If needed, an explanation for the attendance can be provided here.}

\newpage
\section{Commits}

\wss{For each team member how many commits to the main branch have been made
over the time period of interest.  The total is the total number of commits for
the entire team since the beginning of the term.  The percentage is the
percentage of the total commits made by each team member.}

%TODO UPDATE NUMBERS
% The number of commits
\pgfmathsetmacro{\CJ}{32}
\pgfmathsetmacro{\CN}{65}
\pgfmathsetmacro{\CNS}{8}
\pgfmathsetmacro{\CI}{10}
\pgfmathsetmacro{\CH}{43}

% Calculate the total
\pgfmathsetmacro{\CT}{\CJ + \CN + \CNS + \CH + \CI}

% Calculate the percentages
\pgfmathsetmacro{\PJ}{\CJ / \CT * 100}
\pgfmathsetmacro{\PN}{\CN / \CT * 100}
\pgfmathsetmacro{\PNS}{\CNS / \CT * 100}
\pgfmathsetmacro{\PI}{\CI / \CT * 100}
\pgfmathsetmacro{\PH}{\CH / \CT * 100}

\begin{table}[H]
\centering
\begin{tabular}{lll}
\toprule
\textbf{Student} & \textbf{Commits} & \textbf{Percent}\\
\midrule
Total & \pgfmathprintnumber[fixed,precision=0]\CT & 100\% \\
\midrule
John P. & \CJ & \pgfmathprintnumber[fixed,precision=2]\PJ\% \\
Nigel M. & \CN & \pgfmathprintnumber[fixed,precision=2]\PN\% \\
Naishan G. & \CNS & \pgfmathprintnumber[fixed,precision=2]\PNS\% \\
Isaac G. & \CI & \pgfmathprintnumber[fixed,precision=2]\PI\% \\
Hemraj B. & \CH & \pgfmathprintnumber[fixed,precision=2]\PH\% \\
\bottomrule
\end{tabular}
\end{table}

\wss{If needed, an explanation for the counts can be provided here.  For
instance, if a team member has more commits to unmerged branches, these numbers
can be provided here.  If multiple people contribute to a commit, git allows for
multi-author commits.}

\newpage
\section{Issue Tracker}

\wss{For each team member how many issues have they authored (including open and
closed issues (O+C)) and how many have they been assigned (only counting closed
issues (C only)) over the time period of interest.}

% is:issue author:John-Popovici
% is:issue author:nigelmoses32
% is:issue author:STARS952
% is:issue author:Isaac020717
% is:issue author:HemrajB87

% is:issue is:closed assignee:John-Popovici 
% is:issue is:closed assignee:nigelmoses32
% is:issue is:closed assignee:STARS952
% is:issue is:closed assignee:Isaac020717
% is:issue is:closed assignee:HemrajB87

% UPDATE NUMBERS
\begin{table}[H]
\centering
\begin{tabular}{lll}
\toprule
\textbf{Student} & \textbf{Authored (O+C)} & \textbf{Assigned (C only)}\\
\midrule
John P. & 14 & 12 \\
Nigel M. & 18 & 8 \\
Naishan G. & 0 & 0 \\
Isaac G. & 4 & 5 \\
Hemraj B. & 2 & 3 \\
\bottomrule
\end{tabular}
\end{table}

\wss{If needed, an explanation for the counts can be provided here.}

\newpage
\section{CICD}

CI/CD has been implemented to automate pdf generation. Our team has implemented a github action that takes advantage of the existing Makefiles in our project to automatically build pdfs whenever latex files are committed to our repository. This action cleans existing pdfs and artifacts, and automatically generates bibliographies and pdf outputs on commits.

\end{document}