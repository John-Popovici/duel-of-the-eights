\documentclass[12pt, titlepage]{article}

\usepackage{booktabs}
\usepackage{tabularx}
\usepackage{hyperref}
\usepackage{float}
\usepackage{adjustbox}
\usepackage{float}
\hypersetup{
    colorlinks,
    citecolor=blue,
    filecolor=black,
    linkcolor=red,
    urlcolor=blue
}
\usepackage[round]{natbib}

%% Comments

\usepackage{color}

\newif\ifcomments\commentstrue %displays comments
%\newif\ifcomments\commentsfalse %so that comments do not display

\ifcomments
\newcommand{\authornote}[3]{\textcolor{#1}{[#3 ---#2]}}
\newcommand{\todo}[1]{\textcolor{red}{[TODO: #1]}}
\else
\newcommand{\authornote}[3]{}
\newcommand{\todo}[1]{}
\fi

\newcommand{\wss}[1]{\authornote{blue}{SS}{#1}} 
\newcommand{\plt}[1]{\authornote{magenta}{TPLT}{#1}} %For explanation of the template
\newcommand{\an}[1]{\authornote{cyan}{Author}{#1}}

%% Common Parts

\newcommand{\progname}{SFWRENG 4G06} % PUT YOUR PROGRAM NAME HERE
\newcommand{\authname}{Team 9, dice\_devs
\\ John Popovici
\\ Nigel Moses
\\ Naishan Guo
\\ Hemraj Bhatt
\\ Isaac Giles} % AUTHOR NAMES                  

\usepackage{hyperref}
    \hypersetup{colorlinks=true, linkcolor=blue, citecolor=blue, filecolor=blue,
                urlcolor=blue, unicode=false}
    \urlstyle{same}
                                


\begin{document}

\title{System Verification and Validation Plan for \\\progname: \\Dice Duels: Duel of the Eights} 
\author{\authname}
\date{\today}
	
\maketitle

\pagenumbering{roman}

\section*{Revision History}

\begin{table}[hp]
\caption{Revision History} \label{TblRevisionHistory}
\begin{tabularx}{\textwidth}{llX}
\toprule
\textbf{Date} & \textbf{Developer(s)} & \textbf{Change}\\
\midrule
2024-10-27 & John Popovici & Preparing file and added 1.1, 1.2, 1.3\\
2024-10-28 & John Popovici & Removed table of symbols; Added 1.4\\
2024-10-30 & Hemraj Bhatt & Added content to section 2\\
2024-10-31 & John Popovici & Finalized section 1 formatting and content\\
2024-11-01 & Isaac Giles & Filled in content for section 3\\
2024-11-02 & Naishan Guo & Added new content for section 3\\
2024-11-04 & Hemraj Bhatt & Added content to reflection \\
2024-11-04 & Hemraj Bhatt & Updated section 2 content \\
2024-11-04 & John Popovici & Added reflections \\
2024-11-04 & John Popovici & Added symbols table new section 1 \\
2025-03-07 & John Popovici & Updated extras and added usability testing questions \\
2025-03-08 & Isaac Giles & Added section 5 unit test section \\
2025-03-10 & John Popovici & Updated and converted tables to matrices \\
2025-03-08 & Isaac Giles & Corrected module names in section 5 \\
\bottomrule
\end{tabularx}
\end{table}

~\\
\iffalse
\wss{The intention of the VnV plan is to increase confidence in the software.
However, this does not mean listing every verification and validation technique
that has ever been devised.  The VnV plan should also be a \textbf{feasible}
plan. Execution of the plan should be possible with the time and team available.
If the full plan cannot be completed during the time available, it can either be
modified to ``fake it'', or a better solution is to add a section describing
what work has been completed and what work is still planned for the future.}

\wss{The VnV plan is typically started after the requirements stage, but before
the design stage.  This means that the sections related to unit testing cannot
initially be completed.  The sections will be filled in after the design stage
is complete.  the final version of the VnV plan should have all sections filled
in.}
\fi

\newpage

\tableofcontents

\listoftables
% \wss{Remove this section if it isn't needed}

% \listoffigures
% \wss{Remove this section if it isn't needed}

\newpage

%\iffalse
\section{Symbols, Abbreviations, and Acronyms}

\renewcommand{\arraystretch}{1.2}
\begin{table}[hp]
\caption{Symbols, Abbreviations, and Acronyms}
\begin{tabular}{l l} 
  \toprule		
  \textbf{symbol} & \textbf{description}\\
  \midrule 
  SRS & Software Requirements Specification\\
  HA & Hazard Analysis\\
  MIS & Module Interface Specification\\
  MG & Module Guide\\
  VnV or V\&V & Verification and Validation\\
  CI & Continuous Integration\\
  CD & Continuous Development\\
  Test-X & Test number X\\
  RX & Functional Requirement number X\\
  NFRX & Non Functional Requirement number X\\
  \bottomrule
\end{tabular}\\
\end{table}

% \wss{symbols, abbreviations, or acronyms --- you can simply reference the SRS \citep{SRS} tables, if appropriate}

% \wss{Remove this section if it isn't needed}

\newpage
%\fi

\pagenumbering{arabic}

%This document ... \wss{provide an introductory blurb and roadmap of the Verification and Validation plan}

\section{General Information}

\subsection{Summary}

This project creates an online multiplayer game platform that allows for the creation of
custom Yahtzee-like games and variants. This family of games will come with some presets
such as classic Yahtzee, an octahedron version, and more, but will allow for the
users to set their own variables such as number of dice, what kind of dice, and some elements
of scoring, and then play that game. Kinds of dice would include cubes and octahedrons,
among other multi-sided dice, and scoring could be calculated at the end of the game, as in
classic Yahtzee, or on a per-round basis, where hands go in a head-to-head matchup.

\subsection{Objectives}

\iffalse
\wss{State what is intended to be accomplished.  The objective will be around
  the qualities that are most important for your project.  You might have
  something like: ``build confidence in the software correctness,''
  ``demonstrate adequate usability.'' etc.  You won't list all of the qualities,
  just those that are most important.}

\wss{You should also list the objectives that are out of scope.  You don't have 
the resources to do everything, so what will you be leaving out.  For instance, 
if you are not going to verify the quality of usability, state this.  It is also 
worthwhile to justify why the objectives are left out.}

\wss{The objectives are important because they highlight that you are aware of 
limitations in your resources for verification and validation.  You can't do everything, 
so what are you going to prioritize?  As an example, if your system depends on an 
external library, you can explicitly state that you will assume that external library 
has already been verified by its implementation team.}
\fi

The main objective of this project is to ensure robust and flexible functionality that supports customizability in game creation, with confidence in their interactability. Specifically, the project will focus on:

\begin{itemize}
	\item Demonstrating Adequate Usability: User interactions should be intuitive, enabling smooth navigation and game setup without requiring extensive technical knowledge.
	\item Supporting User-Driven Customizability: Emphasis will be placed on making the platform adaptable, allowing users to easily create and modify games with various dice types and scoring rules. These elements must be modular such that a custom game can make use of these components in a flexible fashion.
	\item Ensuring Software Correctness: Testing will confirm accurate dice roll usage, score calculations, and game outcomes to build confidence in the correctness of the core mechanics.
\end{itemize}

Out of Scope Objectives: Due to time constraints, certain areas are out of scope for this project. These include:

\begin{itemize}
	\item Probability Testing: While a statistical analysis could be done on the randomness of dice rolls, it would be out of scope for our current plans and we will proceed with a general assumption that functions that rely on random values are as accurately pseudo-random as needed, without set seeding.
	\item Performance Optimization: While performance will be monitored, extensive optimization for faulty connections or intricate rule configurations will be left out, with plans to address this in potential future releases.
\end{itemize}

This prioritization allows us to allocate resources efficiently, focusing on core functionality and foundational elements.

\subsection{Challenge Level and Extras}

\iffalse
\wss{State the challenge level (advanced, general, basic) for your project.
Your challenge level should exactly match what is included in your problem
statement.  This should be the challenge level agreed on between you and the
course instructor.  You can use a pull request to update your challenge level
(in TeamComposition.csv or Repos.csv) if your plan changes as a result of the
VnV planning exercise.}

\wss{Summarize the extras (if any) that were tackled by this project.  Extras
can include usability testing, code walkthroughs, user documentation, formal
proof, GenderMag personas, Design Thinking, etc.  Extras should have already
been approved by the course instructor as included in your problem statement.
You can use a pull request to update your extras (in TeamComposition.csv or
Repos.csv) if your plan changes as a result of the VnV planning exercise.}
\fi

Through the implementation of not just a single Yahtzee game variant, but by implementing a system where the user can create a custom game variant and connect to another online player to play it, we are looking to develop our project to be both a fun game and a capstone project. The challenge difficulty approved is that of general.

%TODO update problem statement and gitlab once we know difficulty level and what extras
\iffalse
, we are looking to achieve the advanced challenge level. We would also provide some pre-set game variants that have been tested and were found to be more fun than others.
\fi

We will include extra elements to aid in better developing our game and allow us to design for a fun gaming experience. Following industry practices in game development, we will do the following extras:
\begin{itemize}
	\item Usability Testing built off of different elements of focus groups, surveys, and interviews. Questions and elements of focus are highlighted in section 6.2
	\item User manual
\end{itemize}

\subsection{Relevant Documentation}

\iffalse
\wss{Reference relevant documentation.  This will definitely include your SRS
  and your other project documents (design documents, like MG, MIS, etc).  You
  can include these even before they are written, since by the time the project
  is done, they will be written.  You can create BibTeX entries for your
  documents and within those entries include a hyperlink to the documents.}
\citet{SRS}
\wss{Don't just list the other documents.  You should explain why they are relevant and 
how they relate to your VnV efforts.}
\fi

The following documents are crucial for the Verification and Validation process, ensuring the project meets its objectives and operates reliably:

\begin{itemize}
	\item \href{https://github.com/John-Popovici/duel-of-the-eights/blob/main/docs/SRS/SRS.pdf}{Software Requirements Specification (SRS)}: This document outlines the core requirements for the platform, defining features and limitations, as well as expected behaviors. The SRS serves as a primary reference to validate that the developed software meets user and system requirements.
	\item \href{https://github.com/John-Popovici/duel-of-the-eights/blob/main/docs/HazardAnalysis/HazardAnalysis.pdf}{Hazard Analysis (HA)}: This document is key to identifying and managing risks associated with system failures, particularly in dice physics, scoring, and multiplayer stability.
	\item \href{https://github.com/John-Popovici/duel-of-the-eights/blob/main/docs/Design/SoftDetailedDes/MIS.pdf}{Module Interface Specification (MIS)}: The MIS details the interfaces between components, which is essential for ensuring seamless interaction across modules and verifying that dependencies are managed correctly.
	\item \href{https://github.com/John-Popovici/duel-of-the-eights/blob/main/docs/Design/SoftArchitecture/MG.pdf}{Module Guide (MG)}: The MG provides a structural breakdown of the system's components, allowing verification that each module aligns with the SRS specifications and functions as intended.
\end{itemize}


\section{Plan}

%\wss{Introduce this section.  You can provide a roadmap of the sections to come.}

\noindent This section covers the project's verification and validation (V\&V) strategy, including the methodologies and strategies used to ensure software dependability, correctness, and usability. Each component of the plan will establish verification and validation procedures appropriate for each stage of the project lifecycle, from reviewing the Software Requirements Specification (SRS) to testing the developed system.

\subsection{Verification and Validation Team}
\begin{table}[H]
  \centering
  %\renewcommand{\arraystretch}{1.2} 
  \begin{adjustbox}{max width=0.80\textwidth} 
    \begin{tabularx}{\linewidth}{|p{2.5cm}|p{2.75cm}|X|} 
    \hline
    \textbf{Team Member} & \textbf{Role} & \textbf{Responsibilities} \\ \hline
    Isaac Giles & SRS\newline Verification Lead & Oversees SRS reviews, organizes feedback, coordinates checklist compilation, and ensures any new or modified requirements are accurate. \\ \hline
    John Popovici & Design\newline Verification Lead & Oversees design reviews, develops checklists, and organizes peer review meetings regarding design. They also ensure that any design modifications that either are outcomes of meetings or decisions made within the team are properly formulated and accurate.  \\ \hline
    Nai Shan Guo & Implementation Tester & Performs unit tests, isolates tests for core capabilities, and ensures that individual code components are in line with the implementation plan outlined in this document. In charge of testing any new modifications and green-lighting them. \\ \hline
    Nigel Moses & Software\newline Validation Lead & In charge of ensuring that the software matches user requirements through using feedback provided by supervisor and peer reviews. Additionally, in charge of performing validation strategies outlined in the document. \\ \hline
    Hemraj Bhatt & Tool and\newline Automation Lead & Oversees the installation and usage of automated testing tools, unit testing frameworks, and test coverage metric tracking tools. In charge of sourcing the correct tools needed for the specific system as well. \\ \hline
    Dr.\newline Paul Rapoport & Project\newline Supervisor & Provides oversight, participates in structured review sessions, and provides feedback on requirements, design, and implementation reviews. Due to the supervisor's area of expertise, feedback will focus on requirements and design as opposed to programming implementation advice. \\ \hline
    \end{tabularx}
  \end{adjustbox}
  \caption{Verification and Validation Team Roles}
  \label{tab:team}
\end{table}

\newpage
\subsection{SRS Verification Plan}

\noindent The SRS verification will take an organized approach, including supervisor and peer review sessions, to ensure that the stated requirements are clear, accurate, and thorough. 

\begin{itemize}
    \item \textbf{Structured Review Meeting}: Conducted with the supervisor in person to present the essential SRS components. The team will highlight the essential functional and non-functional requirements and solicit feedback on the clarity and feasibility of each requirement. The feedback will then be tracked as issues and be used to make any modifications deemed necessary.
    \item \textbf{Checklist Creation}: A detailed checklist will be prepared to analyze each SRS item, ensuring the requirements are accurate and feasible for execution. The checklist will be utilized at each review session. If the checklist is filled out completely, the team will know that there are no necessary changes and if it is not then necessary modifications will be made to the SRS.
    \item \textbf{Peer Review Sessions}: The team will have our peers, such as classmates, complete task-based inspections of the SRS to identify any ambiguities or missing components. Tasks may include examining the document for completeness, by ensuring that all functional and non-functional requirements are properly stated, checking for terminology consistency, and ensuring that any assumptions or restrictions are well-defined and reasonably made. This will ensure that the SRS is complete and viable. 
\end{itemize}

\subsection{Design Verification Plan}

\noindent To ensure that the design complies with the SRS standards, the team will conduct a structured review process including the entire team, external peers and the teams supervisor:

\begin{itemize}
    \item \textbf{Design Checklists}:  A checklist will be developed to guide reviewers through each design component and confirm whether or not they correspond to both the functional and non-functional requirements outlined in the SRS.
    \item \textbf{Peer Review}: Classmates and primary reviewers will use the checklist to determine if the design is consistent with the SRS. This includes the code structure, front-end design, and game mechanics. 
    \item \textbf{Structured Review Meeting}: The team will schedule a review session with our supervisor where each design element will be presented. The goal being to receive feedback and make any necessary modifications. 
\end{itemize}

\subsection{Verification and Validation Plan Verification Plan}

\noindent The V\&V plan itself will be verified to make sure it adequately addresses every facet of the project and sufficiently plans appropriate methods of verification and validation.

\begin{itemize}
  \item \textbf{Mutation Testing}: To guarantee robustness of our software, we'll use mutation testing to assess how well the test cases work and to make sure that the methods used for verification can accurately detect software faults.
  \item \textbf{Review Sessions}: To ensure that all necessary verification and validation methods are covered, the V\&V plan will be reviewed by the teams peers and supervisor utilizing a checklist. The team will also independently review the methods to ensure all bases are covered.
\end{itemize}

\subsection{Implementation Verification Plan}

\noindent The implementation verification will primarily focus on unit testing and manual testing, emphasizing the isolation of modules and testing them with constant inputs mimicking user inputs. Given that project is a game, testing poses unique challenges and nuances that require careful consideration.

\begin{itemize}
  \item \textbf{Unit Testing}: The team will employ isolated unit tests, setting fixed inputs to verify core functionalities like game logic, scoring systems, and user input handling.  Simulations of dice rolls and the scores that go along with them are one example. Additionally, to ensure that tests only concentrate on the module's functionality and are not influenced by outside factors, mocking will be used to mimic external dependencies.
  \item \textbf{Manual Testing}: Manual testing will be used to evaluate the game from a player's point of view and make sure that gameplay is fluid, intuitive, and stimulating, manual testing will be used. In order to find any possible errors or discrepancies in the game mechanics, test cases will cover a range of scenarios, including edge cases and user interactions.
  \item \textbf{Code Walkthroughs}: In order to collect feedback on code correctness, standard compliance adherence, and alignment with design specifications outlined in the SRS, team-based code walkthroughs will be carried out. These walkthroughs will be held with our peers and external individuals with expertise in the field.
  \item \textbf{Static Analysis}: FxCop, a free static code analysis tool from Microsoft, will be utilized to evaluate the code. It excels at analyzing .NET managed code assemblies for adherence to Microsoft's .NET Framework Design Guidelines. This tool will help ensure code quality and identify potential issues early in the development cycle. It is particularly well-suited for this project, as the game is being developed using C\# on the .NET version of Godot.
\end{itemize}

\subsection{Automated Testing and Verification Tools}

\noindent To maintain code quality and reliability, teh team will implement a variety of automated testing tools. Any mentioned tools are not finalized and may be substituted based on team preferences and needs:

  \begin{itemize}
    \item \textbf{Unit Testing Framework}: The team will employ Godot's 'GUT Test' unit testing framework, to test all in-game functionality that can be reasonably tested via unit tests. This will help ensure that each game feature functions consistently and accurately.
    \item \textbf{Continuous Integration (CI)}: The team will implement continuous integration through GitHub Actions to automate pdf generation when latex files are committed, and also to run unit tests on pull requests whenever code changes are made. Unit tests will be written within the GUT(Godot Unit Testing) testing framework and we will create a github action to automatically download Godot and GUT, to run the full test suite and block code merges when test cases fail.
  \end{itemize}

\subsection{Software Validation Plan}

\noindent Our software validation will involve confirming that the system meets user expectations and the requirements outlined in the SRS.

\begin{itemize}
  \item \textbf{Stakeholder Review Sessions}: In order to make sure that requirements match user expectations and needs, we will arrange semi-regular review meetings with our supervisor and potential users, who are stakeholders. These meetings will be used to modify requirements as needed and will provide validation for newly developed components of the system.
  \item \textbf{Task-Based Inspection}: In order to demonstrate that the system performs as intended, the team will develop use-case scenarios and conduct task-based inspections to validate whether or not the system operates correctly. For example, one use-case scenario could be a player seeking to accomplish a "Full House" by rolling three dice of one number and two of another. A task-based inspection would include ensuring that the system correctly recognizes and scores this combination as a Full House, awarding the required points.
  \item \textbf{Rev 0 Demo Feedback}: For the purpose of further refining the system and optimizing it prior to the final release, the team will collect feedback from our supervisor both before and after the Rev 0 demo. 
\end{itemize}

\noindent These strategies are additions to the ones outlined in section 2.2.


\section{System Tests}

This section covers tests for functional and non-functional requirements to verify the game's adherence to its design specifications. The functional requirements focus on verifying the gameplay mechanics, score calculations, and user interface, while the non-functional requirements test aspects like performance, usability, and system compatibility.

\subsection{Tests for Functional Requirements}

Functional tests are organized into different areas based on gameplay features, score handling, and user interface components, as defined in the software requirements specification (SRS).

\subsubsection{Gameplay and Rules}

Testing here covers the core game mechanics, such as player vs player mode, score calculations, and dice roll simulation.

\paragraph{\label{test-1}Test: PvP Support}
\begin{itemize}
    \item \textbf{Test ID:} Test-1
    \item \textbf{Control:} Manual
    \item \textbf{Initial State:} Game loaded to main menu.
    \item \textbf{Input:} Two players connect to a match.
    \item \textbf{Output:} Game begins with both players connected, and each takes turns as expected.
    \item \textbf{Test Case Derivation:} Confirms successful setup and turn-taking in online mode, satisfying R1.
    \item \textbf{How test will be performed:} Two players join a match in PvP mode and verify that the game can be initiated and turns can be taken as expected.
\end{itemize}

\paragraph{\label{test-2}Test: Score Calculation}
\begin{itemize}
    \item \textbf{Test ID:} Test-2
    \item \textbf{Control:} Automated
    \item \textbf{Initial State:} Active game session with dice rolled.
    \item \textbf{Input:} Player achieves four of a kind and selects the four of a kind option.
    \item \textbf{Output:} Correct score added from the four of a kind option.
    \item \textbf{Test Case Derivation:} Ensures that score calculations adhere to specified Yahtzee-like rules, satisfying R2.
    \item \textbf{How test will be performed:} Test case evaluates various combinations to ensure scoring accuracy.
\end{itemize}

\paragraph{\label{test-3}Test: Dice Roll Physics}
\begin{itemize}
    \item \textbf{Test ID:} Test-3
    \item \textbf{Control:} Manual
    \item \textbf{Initial State:} Game session loaded.
    \item \textbf{Input:} Player rolls the dice.
    \item \textbf{Output:} 3D dice roll with randomized, realistic outcome.
    \item \textbf{Test Case Derivation:} Validates realistic physics for dice roll, ensuring that each outcome resembles real world physics, satisfying R3.
    \item \textbf{How test will be performed:} Tester initiates rolls and observes animation and outcomes.
\end{itemize}

\paragraph{\label{test-4}Test: Simultaneous turn implementation}
\begin{itemize}
    \item \textbf{Test ID:} Test-4
    \item \textbf{Control:} Manual
    \item \textbf{Initial State:} Online game session loaded, Both players rolled once, one player confirmed choice of dice to reroll and is in waiting screen
    \item \textbf{Input:} Second player confirms choice of dice to reroll 
    \item \textbf{Output:} Game rerolls chosen dice for both players, updates both players with the results of both rerolls, and allows both players to select new dice to reroll at the same time
    \item \textbf{Test Case Derivation:} Verifies that one player can’t move to their next turn until both players are ready, thus enforcing the rules of Simultaneous turn implementation and satisfying requirement R6
    \item \textbf{How test will be performed:} Two testers in the same online session will end their turns in different orders to confirm the functionality of the implementation of Simultaneous turns.
\end{itemize}

\subsubsection{User Interface and Settings}

This section validates the interface, display, and customization options to ensure they meet user needs and game standards.

\paragraph{\label{test-5}Test: User interface initiation}
\begin{itemize}
    \item \textbf{Test ID:} Test-5
    \item \textbf{Control:} Manual
    \item \textbf{Initial State:} Game loaded to main menu
    \item \textbf{Input:} Player connects to a match
    \item \textbf{Output:} Game begins and a UI displaying important information, such as Player names, scores, number of rolls and time limits, is displayed.
    \item \textbf{Test Case Derivation:} confirms the Display of UI elements conveying important information to the User, thus satisfying requirement R8
    \item \textbf{How test will be performed:} Upon the beginning of a match, testers will verify that expected UI is visible for display, and updates as the game goes on.
\end{itemize}


\paragraph{\label{test-6}Test: Dice Selection}
\begin{itemize}
    \item \textbf{Test ID:} Test-6
    \item \textbf{Control:} Automated
    \item \textbf{Initial State:} Mid-game, with player allowed to roll again.
    \item \textbf{Input:} Player selects specific dice to roll and keeps those remaining.
    \item \textbf{Output:} Only selected dice are re-rolled, while unselected dice remain.
    \item \textbf{Test Case Derivation:} Validates dice selection mechanics as per game rules, satisfying R7.
    \item \textbf{How test will be performed:} Automated test case selects some dice and performs a roll. The test then checks that the dice that were selected are rolling, and the dice that were not selected are not rolling.
\end{itemize}

\paragraph{\label{test-7}Test: Game Presets}
\begin{itemize}
    \item \textbf{Test ID:} Test-7
    \item \textbf{Control:} Manual
    \item \textbf{Initial State:} Game loaded to game select menu
    \item \textbf{Input:} Player selects a preset game option
    \item \textbf{Output:} A new game session begins with the pre-made settings applied.
    \item \textbf{Test Case Derivation:} Confirms that each preset produces a unique variant of the game, Satisfying R10
    \item \textbf{How test will be performed:} Tester selects the pre-made options and runs the games to confirm functionality.
\end{itemize}

\paragraph{\label{test-8}Test: Custom Game Settings Modification}
\begin{itemize}
    \item \textbf{Test ID:} Test-8
    \item \textbf{Control:} Manual
    \item \textbf{Initial State:} Custom Game settings menu open.
    \item \textbf{Input:} Player modifies settings (e.g., changes dice sides, scoring rules).
    \item \textbf{Output:} Changes are successfully applied to new game sessions.
    \item \textbf{Test Case Derivation:} Verifies that customizable game settings reflect unique variants, satisfying R9 and R15.
    \item \textbf{How test will be performed:} Tester modifies various settings, saving them, and initiates games to confirm functionality.
\end{itemize}

\subsection{Tests for Nonfunctional Requirements}

Nonfunctional tests focus on measuring performance, responsiveness, reliability, and usability. These tests ensure the game performs well under different conditions and on various systems.

\subsubsection{Performance and Usability}

\paragraph{\label{test-9}Test: Frame Rate Consistency}
\begin{itemize}
    \item \textbf{Test ID:} Test-9
    \item \textbf{Type:} Dynamic, Manual (from usability testing)
    \item \textbf{Initial State:} Game in progress.
    \item \textbf{Input/Condition:} Standard gameplay, including dice rolls, UI interactions, and animations.
    \item \textbf{Output/Result:} Game maintains a frame rate of at least 30 FPS.
    \item \textbf{How test will be performed:} Frame rate monitored by testers during gameplay on low and high-end systems to ensure stability.
\end{itemize}

\paragraph{\label{test-10}Test: Input Responsiveness}
\begin{itemize}
    \item \textbf{Test ID:} Test-10
    \item \textbf{Type:} Dynamic, Manual (from usability testing)
    \item \textbf{Initial State:} Game loaded.
    \item \textbf{Input:} Player performs multiple actions (e.g., roll dice, end turn).
    \item \textbf{Output:} System responds within 500 ms of input.
    \item \textbf{How test will be performed:} Poor response times will be noted by users during usability testing.
\end{itemize}

\subsubsection{System Compatibility and Reliability}

\paragraph{\label{test-11}Test: System Compatibility}
\begin{itemize}
    \item \textbf{Test ID:} Test-11
    \item \textbf{Type:} Manual
    \item \textbf{Initial State:} Game installed on a Windows 10 system.
    \item \textbf{Input/Condition:} Launch and operate the game as per standard.
    \item \textbf{Output/Result:} Game loads and operates without compatibility issues.
    \item \textbf{How test will be performed:} Run game on Windows 10 (and MacOS - stretch goal) to confirm compatibility, satisfying NFR3 (and NFR10).
\end{itemize}

\paragraph{\label{test-12}Test: Crash Resilience in Multiplayer}
\begin{itemize}
    \item \textbf{Test ID:} Test-12
    \item \textbf{Type:} Dynamic, Manual
    \item \textbf{Initial State:} Multiplayer game session in progress.
    \item \textbf{Input/Condition:} Two players connected and actively playing.
    \item \textbf{Output/Result:} Game remains stable, with crashes occurring less than 1\% of the time.
    \item \textbf{How test will be performed:} Simulated load testing over numerous sessions, logging any crashes to validate NFR4.
\end{itemize}

\newpage
\subsection{Traceability Between Test Cases and Requirements}

The following tables provides traceability between the test cases and the requirements. Each test case is mapped to one or more requirements that it validates, ensuring coverage of our functional and non-functional requirements. Tables for keys are provided to help clarify test and requirements.

\begin{table}[H]
    \centering
    \caption{Key for Test Cases}
    \begin{tabular}{|c|l|}
        \hline
        \textbf{Test Case ID} & \textbf{Test Case Name} \\
        \hline
        \hyperref[test-1]{Test-1} & PvP Support \\
        \hyperref[test-2]{Test-2} & Score Calculation \\
        \hyperref[test-3]{Test-3} & Dice Roll Physics \\
        \hyperref[test-4]{Test-4} & Simultaneous turn implementation \\
        \hyperref[test-5]{Test-5} & User interface initiation \\
        \hyperref[test-6]{Test-6} & Dice Selection \\
        \hyperref[test-7]{Test-7} & Game Presets \\
        \hyperref[test-8]{Test-8} & Custom Game Settings Modification \\
        \hyperref[test-9]{Test-9} & Frame Rate Consistency \\
        \hyperref[test-10]{Test-10} & Input Responsiveness \\
        \hyperref[test-11]{Test-11} & System Compatibility \\
        \hyperref[test-12]{Test-12} & Crash Resilience in Multiplayer \\
        \hline
    \end{tabular}
    \label{tab:traceability_key1}
\end{table}

\begin{table}[H]
    \centering
    \caption{Key for Requirements}
    \begin{tabular}{|c|l|}
        \hline
        \textbf{Requirement ID} & \textbf{Requirement Name} \\
        \hline
R1 & Online player vs player mode \\  
R2 & Score calculation using Yahtzee rules \\  
R3 & Realistic physics for 3D dice rolls \\  
R4 & Dice values determined by actual roll outcome \\  
R5 & Support for different-sided dice \\  
R6 & Simultaneous turn-based mechanism \\  
R7 & Player selection of dice to roll \\  
R8 & User interface displaying game information \\  
R9 & Customizable game settings \\  
R10 & Preset game modes \\  
R11* & Local player vs player mode \\  
R12* & Player vs computer mode \\  
R13* & Algorithmic computer opponent \\  
R14* & Online matchmaking \\  
R15* & Saving custom game settings \\  
R16* & Round statistics display \\  
R17 & Accurate display of current game state \\  
NFR1 & Minimum 30 FPS performance \\  
NFR2 & Clear and easy-to-use user interface \\  
NFR3 & Support for Windows 10 and later \\  
NFR4 & Multiplayer crash rate below 1\% \\  
NFR5 & Maximum input response time of 500ms \\  
NFR6 & Modular and maintainable codebase \\  
NFR7 & Optimization for low-end systems \\  
NFR8 & Enjoyability rating of at least 75\% \\  
NFR9 & Consistent UI and 3D visual style \\  
NFR10* & MacOS support \\ 
        \hline
    \end{tabular}
    \label{tab:traceability_key2}
\end{table}
\textit{* indicates stretch goal - out of scope}



\begin{table}[H]
    \centering
    \caption{Traceability Matrix for Test Cases and Requirements}
    \begin{tabular}{|c|cccccccccccc|}
\hline
 & \multicolumn{12}{c|}{\textbf{Tests}} \\
\hline
\textbf{Req.} & \multicolumn{1}{c|}{\hyperref[test-1]{1}} & \multicolumn{1}{c|}{\hyperref[test-2]{2}} & \multicolumn{1}{c|}{\hyperref[test-3]{3}}
& \multicolumn{1}{c|}{\hyperref[test-4]{4}} & \multicolumn{1}{c|}{\hyperref[test-5]{5}} & \multicolumn{1}{c|}{\hyperref[test-6]{6}}
& \multicolumn{1}{c|}{\hyperref[test-7]{7}} & \multicolumn{1}{c|}{\hyperref[test-8]{8}} & \multicolumn{1}{c|}{\hyperref[test-9]{9}}
& \multicolumn{1}{c|}{\hyperref[test-10]{10}} & \multicolumn{1}{c|}{\hyperref[test-11]{11}} & \hyperref[test-12]{12} \\ \hline

R1 & \multicolumn{1}{c|}{X}  & \multicolumn{1}{c|}{}  & \multicolumn{1}{c|}{}
& \multicolumn{1}{c|}{}  & \multicolumn{1}{c|}{}  & \multicolumn{1}{c|}{}
& \multicolumn{1}{c|}{}  & \multicolumn{1}{c|}{}  & \multicolumn{1}{c|}{}
& \multicolumn{1}{c|}{}   & \multicolumn{1}{c|}{}   &    \\ \hline

R2 & \multicolumn{1}{c|}{}  & \multicolumn{1}{c|}{X}  & \multicolumn{1}{c|}{}
& \multicolumn{1}{c|}{}  & \multicolumn{1}{c|}{}  & \multicolumn{1}{c|}{}
& \multicolumn{1}{c|}{}  & \multicolumn{1}{c|}{}  & \multicolumn{1}{c|}{}
& \multicolumn{1}{c|}{}   & \multicolumn{1}{c|}{}   &    \\ \hline

R3 & \multicolumn{1}{c|}{}  & \multicolumn{1}{c|}{}  & \multicolumn{1}{c|}{X}
& \multicolumn{1}{c|}{}  & \multicolumn{1}{c|}{}  & \multicolumn{1}{c|}{}
& \multicolumn{1}{c|}{}  & \multicolumn{1}{c|}{}  & \multicolumn{1}{c|}{}
& \multicolumn{1}{c|}{}   & \multicolumn{1}{c|}{}   &    \\ \hline

R4 & \multicolumn{1}{c|}{}  & \multicolumn{1}{c|}{}  & \multicolumn{1}{c|}{X}
& \multicolumn{1}{c|}{}  & \multicolumn{1}{c|}{}  & \multicolumn{1}{c|}{}
& \multicolumn{1}{c|}{}  & \multicolumn{1}{c|}{}  & \multicolumn{1}{c|}{}
& \multicolumn{1}{c|}{}   & \multicolumn{1}{c|}{}   &    \\ \hline

R5 & \multicolumn{1}{c|}{}  & \multicolumn{1}{c|}{}  & \multicolumn{1}{c|}{X}
& \multicolumn{1}{c|}{}  & \multicolumn{1}{c|}{}  & \multicolumn{1}{c|}{}
& \multicolumn{1}{c|}{}  & \multicolumn{1}{c|}{}  & \multicolumn{1}{c|}{}
& \multicolumn{1}{c|}{}   & \multicolumn{1}{c|}{}   &    \\ \hline

R6 & \multicolumn{1}{c|}{}  & \multicolumn{1}{c|}{}  & \multicolumn{1}{c|}{}
& \multicolumn{1}{c|}{X}  & \multicolumn{1}{c|}{}  & \multicolumn{1}{c|}{}
& \multicolumn{1}{c|}{}  & \multicolumn{1}{c|}{}  & \multicolumn{1}{c|}{}
& \multicolumn{1}{c|}{}   & \multicolumn{1}{c|}{}   &    \\ \hline

R7 & \multicolumn{1}{c|}{}  & \multicolumn{1}{c|}{}  & \multicolumn{1}{c|}{X}
& \multicolumn{1}{c|}{}  & \multicolumn{1}{c|}{}  & \multicolumn{1}{c|}{X}
& \multicolumn{1}{c|}{}  & \multicolumn{1}{c|}{}  & \multicolumn{1}{c|}{}
& \multicolumn{1}{c|}{}   & \multicolumn{1}{c|}{}   &    \\ \hline

R8 & \multicolumn{1}{c|}{}  & \multicolumn{1}{c|}{}  & \multicolumn{1}{c|}{}
& \multicolumn{1}{c|}{}  & \multicolumn{1}{c|}{X}  & \multicolumn{1}{c|}{}
& \multicolumn{1}{c|}{}  & \multicolumn{1}{c|}{}  & \multicolumn{1}{c|}{}
& \multicolumn{1}{c|}{}   & \multicolumn{1}{c|}{}   &    \\ \hline

R9 & \multicolumn{1}{c|}{}  & \multicolumn{1}{c|}{}  & \multicolumn{1}{c|}{}
& \multicolumn{1}{c|}{}  & \multicolumn{1}{c|}{}  & \multicolumn{1}{c|}{}
& \multicolumn{1}{c|}{}  & \multicolumn{1}{c|}{X}  & \multicolumn{1}{c|}{}
& \multicolumn{1}{c|}{}   & \multicolumn{1}{c|}{}   &    \\ \hline

R10 & \multicolumn{1}{c|}{}  & \multicolumn{1}{c|}{}  & \multicolumn{1}{c|}{}
& \multicolumn{1}{c|}{}  & \multicolumn{1}{c|}{}  & \multicolumn{1}{c|}{}
& \multicolumn{1}{c|}{X}  & \multicolumn{1}{c|}{}  & \multicolumn{1}{c|}{}
& \multicolumn{1}{c|}{}   & \multicolumn{1}{c|}{}   &    \\ \hline

R11 & \multicolumn{1}{c|}{}  & \multicolumn{1}{c|}{}  & \multicolumn{1}{c|}{}
& \multicolumn{1}{c|}{}  & \multicolumn{1}{c|}{}  & \multicolumn{1}{c|}{}
& \multicolumn{1}{c|}{}  & \multicolumn{1}{c|}{}  & \multicolumn{1}{c|}{}
& \multicolumn{1}{c|}{}   & \multicolumn{1}{c|}{}   &    \\ \hline

R12 & \multicolumn{1}{c|}{}  & \multicolumn{1}{c|}{}  & \multicolumn{1}{c|}{}
& \multicolumn{1}{c|}{}  & \multicolumn{1}{c|}{}  & \multicolumn{1}{c|}{}
& \multicolumn{1}{c|}{}  & \multicolumn{1}{c|}{}  & \multicolumn{1}{c|}{}
& \multicolumn{1}{c|}{}   & \multicolumn{1}{c|}{}   &    \\ \hline

R13 & \multicolumn{1}{c|}{}  & \multicolumn{1}{c|}{}  & \multicolumn{1}{c|}{}
& \multicolumn{1}{c|}{}  & \multicolumn{1}{c|}{}  & \multicolumn{1}{c|}{}
& \multicolumn{1}{c|}{}  & \multicolumn{1}{c|}{}  & \multicolumn{1}{c|}{}
& \multicolumn{1}{c|}{}   & \multicolumn{1}{c|}{}   &    \\ \hline

R14 & \multicolumn{1}{c|}{X}  & \multicolumn{1}{c|}{}  & \multicolumn{1}{c|}{}
& \multicolumn{1}{c|}{}  & \multicolumn{1}{c|}{}  & \multicolumn{1}{c|}{}
& \multicolumn{1}{c|}{}  & \multicolumn{1}{c|}{}  & \multicolumn{1}{c|}{}
& \multicolumn{1}{c|}{}   & \multicolumn{1}{c|}{}   &    \\ \hline

R15 & \multicolumn{1}{c|}{}  & \multicolumn{1}{c|}{}  & \multicolumn{1}{c|}{}
& \multicolumn{1}{c|}{}  & \multicolumn{1}{c|}{}  & \multicolumn{1}{c|}{}
& \multicolumn{1}{c|}{}  & \multicolumn{1}{c|}{X}  & \multicolumn{1}{c|}{}
& \multicolumn{1}{c|}{}   & \multicolumn{1}{c|}{}   &    \\ \hline

R16 & \multicolumn{1}{c|}{}  & \multicolumn{1}{c|}{}  & \multicolumn{1}{c|}{}
& \multicolumn{1}{c|}{}  & \multicolumn{1}{c|}{}  & \multicolumn{1}{c|}{}
& \multicolumn{1}{c|}{}  & \multicolumn{1}{c|}{}  & \multicolumn{1}{c|}{}
& \multicolumn{1}{c|}{}   & \multicolumn{1}{c|}{}   &    \\ \hline

R17 & \multicolumn{1}{c|}{}  & \multicolumn{1}{c|}{}  & \multicolumn{1}{c|}{}
& \multicolumn{1}{c|}{}  & \multicolumn{1}{c|}{}  & \multicolumn{1}{c|}{}
& \multicolumn{1}{c|}{}  & \multicolumn{1}{c|}{}  & \multicolumn{1}{c|}{}
& \multicolumn{1}{c|}{}   & \multicolumn{1}{c|}{}   &    \\ \hline

NFR1 & \multicolumn{1}{c|}{}  & \multicolumn{1}{c|}{}  & \multicolumn{1}{c|}{}
& \multicolumn{1}{c|}{}  & \multicolumn{1}{c|}{}  & \multicolumn{1}{c|}{}
& \multicolumn{1}{c|}{}  & \multicolumn{1}{c|}{}  & \multicolumn{1}{c|}{X}
& \multicolumn{1}{c|}{}   & \multicolumn{1}{c|}{}   &    \\ \hline

NFR2 & \multicolumn{1}{c|}{X}  & \multicolumn{1}{c|}{}  & \multicolumn{1}{c|}{}
& \multicolumn{1}{c|}{}  & \multicolumn{1}{c|}{}  & \multicolumn{1}{c|}{}
& \multicolumn{1}{c|}{}  & \multicolumn{1}{c|}{}  & \multicolumn{1}{c|}{}
& \multicolumn{1}{c|}{}   & \multicolumn{1}{c|}{}   &    \\ \hline

NFR3 & \multicolumn{1}{c|}{}  & \multicolumn{1}{c|}{}  & \multicolumn{1}{c|}{}
& \multicolumn{1}{c|}{}  & \multicolumn{1}{c|}{}  & \multicolumn{1}{c|}{}
& \multicolumn{1}{c|}{}  & \multicolumn{1}{c|}{}  & \multicolumn{1}{c|}{}
& \multicolumn{1}{c|}{}   & \multicolumn{1}{c|}{X}   &    \\ \hline

NFR4 & \multicolumn{1}{c|}{}  & \multicolumn{1}{c|}{}  & \multicolumn{1}{c|}{}
& \multicolumn{1}{c|}{}  & \multicolumn{1}{c|}{}  & \multicolumn{1}{c|}{}
& \multicolumn{1}{c|}{}  & \multicolumn{1}{c|}{}  & \multicolumn{1}{c|}{}
& \multicolumn{1}{c|}{}   & \multicolumn{1}{c|}{}   &  X  \\ \hline

NFR5 & \multicolumn{1}{c|}{}  & \multicolumn{1}{c|}{}  & \multicolumn{1}{c|}{}
& \multicolumn{1}{c|}{}  & \multicolumn{1}{c|}{}  & \multicolumn{1}{c|}{}
& \multicolumn{1}{c|}{}  & \multicolumn{1}{c|}{}  & \multicolumn{1}{c|}{}
& \multicolumn{1}{c|}{X}   & \multicolumn{1}{c|}{}   &    \\ \hline

NFR6 & \multicolumn{1}{c|}{}  & \multicolumn{1}{c|}{}  & \multicolumn{1}{c|}{}
& \multicolumn{1}{c|}{}  & \multicolumn{1}{c|}{}  & \multicolumn{1}{c|}{}
& \multicolumn{1}{c|}{}  & \multicolumn{1}{c|}{}  & \multicolumn{1}{c|}{}
& \multicolumn{1}{c|}{}   & \multicolumn{1}{c|}{}   &    \\ \hline

NFR7 & \multicolumn{1}{c|}{}  & \multicolumn{1}{c|}{}  & \multicolumn{1}{c|}{}
& \multicolumn{1}{c|}{}  & \multicolumn{1}{c|}{}  & \multicolumn{1}{c|}{}
& \multicolumn{1}{c|}{}  & \multicolumn{1}{c|}{}  & \multicolumn{1}{c|}{}
& \multicolumn{1}{c|}{}   & \multicolumn{1}{c|}{}   &    \\ \hline

NFR8 & \multicolumn{1}{c|}{}  & \multicolumn{1}{c|}{}  & \multicolumn{1}{c|}{}
& \multicolumn{1}{c|}{}  & \multicolumn{1}{c|}{}  & \multicolumn{1}{c|}{}
& \multicolumn{1}{c|}{}  & \multicolumn{1}{c|}{}  & \multicolumn{1}{c|}{}
& \multicolumn{1}{c|}{}   & \multicolumn{1}{c|}{}   &    \\ \hline

NFR9 & \multicolumn{1}{c|}{}  & \multicolumn{1}{c|}{}  & \multicolumn{1}{c|}{}
& \multicolumn{1}{c|}{}  & \multicolumn{1}{c|}{}  & \multicolumn{1}{c|}{}
& \multicolumn{1}{c|}{}  & \multicolumn{1}{c|}{}  & \multicolumn{1}{c|}{}
& \multicolumn{1}{c|}{}   & \multicolumn{1}{c|}{}   &    \\ \hline

NFR10 & \multicolumn{1}{c|}{}  & \multicolumn{1}{c|}{}  & \multicolumn{1}{c|}{}
& \multicolumn{1}{c|}{}  & \multicolumn{1}{c|}{}  & \multicolumn{1}{c|}{}
& \multicolumn{1}{c|}{}  & \multicolumn{1}{c|}{}  & \multicolumn{1}{c|}{}
& \multicolumn{1}{c|}{}   & \multicolumn{1}{c|}{X}   &    \\ \hline
    \end{tabular}
    \label{tab:traceability}
\end{table}


\newpage
\section{Unit Test Description}
% Reference MIS and explain philosophy for test case selection.

\subsection{Unit Testing Scope}
% Specify modules that are out of scope and rationale for prioritization.

The unit testing scope includes the Dice Container Module and the Scoring Module. Other modules related to UI and networking are not the focus of unit testing at this stage. The testing priority is given to modules that directly impact gameplay mechanics and fairness.

\subsection{Tests for Functional Requirements}

\subsubsection{DynamicDiceContainer Module}
% Justification for coverage of the module and test selection approach.

\begin{enumerate}
    \item Test: Roll dice functionality
    \begin{itemize}
        \item Type: Functional, Automatic
        \item Initial State: New dice container instance with new dice added
        \item Input: Call the roll dice functionality
        \item Output: All dice get rolled
        \item Test Case Derivation: Ensures that dice always get rolled when requested
        \item How test will be performed: Instantiate new dice container with newly instantiated dice nodes and then verify that all dice are in a rolling state after calling the roll function
    \end{itemize}
    
    \item Test: Rolling selected dice
    \begin{itemize}
        \item Type: Functional, Automatic
        \item Initial State: New dice container with new dice added, some selected and some not
        \item Input: Call the roll selected dice functionality
        \item Output: Selected dice get rolled and unselected dice do not get rolled
        \item Test Case Derivation: Ensures that only selected dice get rolled
        \item How test will be performed: Instantiate new dice container with newly instantiated dice nodes and then verify that selected dice are in a rolling state after calling the roll function and unselected dice are not in a rolling state
    \end{itemize}

    \item Test: Dice selection can be inverted
    \begin{itemize}
        \item Type: Functional, Automatic
        \item Initial State: New dice container with new dice added, some selected and some not
        \item Input: Call invert selection functionality on dice
        \item Output: Selected dice become unselected and unselected dice become selected
        \item Test Case Derivation: Ensures that users can choose to roll selected or keep selected dice
        \item How test will be performed: Instantiate new dice container with newly instantiated dice nodes where some are selected and some are not, then call the invert selection function and verify that the selected dice are now unselected and vice versa
    \end{itemize}

    \item Test: Dice values can be read
    \begin{itemize}
        \item Type: Functional, Automatic
        \item Initial State: New dice container with new dice added, and roll all dice
        \item Input: Call the get dice values functionality
        \item Output: All dice values get returned
        \item Test Case Derivation: Ensures that values can be read after dice get rolled
        \item How test will be performed: Instantiate new dice container with newly instantiated dice nodes and roll all dice, then call the get dice values function and verify that all dice return values
    \end{itemize}

    \item Test: Dice collisions can be toggled
    \begin{itemize}
        \item Type: Functional, Automatic
        \item Initial State: New dice container with new dice added
        \item Input: Call functionality to toggle collisions off, and then back on
        \item Output: All dice lose collidable property, then regain it
        \item Test Case Derivation: Ensures that dice collisions can be toggled as needed for dice animations
        \item How test will be performed: Instantiate new dice container with newly instantiated dice nodes, then call the toggle collisions function and verify that each die has collisions disabled, then call the function again and verify that each die has collisions enabled again
    \end{itemize}

\end{enumerate}

\subsubsection{ScoreCalculator Module}

\begin{enumerate}
    \item Test: Calculate score for a full house
    \begin{itemize}
        \item Type: Functional, Automatic
        \item Initial State: Set of dice showing full house
        \item Input: Dice values
        \item Output: Expected full house score
        \item Test Case Derivation: Ensures correct score computation
        \item How test will be performed: Pass dice values and check score
    \end{itemize}

    \item Test: Calculate score for a kind
    \begin{itemize}
        \item Type: Functional, Automatic
        \item Initial State: Set of dice showing a kind
        \item Input: Dice values and the quantity for the kind
        \item Output: Expected kind score
        \item Test Case Derivation: Ensures scoring logic applies correctly
        \item How test will be performed: Pass kind roll and verify score
    \end{itemize}

    \item Test: Score calculation for a straight
    \begin{itemize}
        \item Type: Functional, Automatic
        \item Initial State: Set of dice showing sequence
        \item Input: Dice values and some expected sequence quantity
        \item Output: Expected straight score
        \item Test Case Derivation: Ensures proper scoring for straights
        \item How test will be performed: Provide sequence values and check score
    \end{itemize}

    \item Test: Score calculation for singles
    \begin{itemize}
        \item Type: Functional, Automatic
        \item Initial State: Set of dice showing singles
        \item Input: Dice values and the single to score on
        \item Output: Expected singles score
        \item Test Case Derivation: Ensures proper scoring for singles
        \item How test will be performed: Provide singles values and check score
    \end{itemize}

    \item Test: Score calculation for chance
    \begin{itemize}
        \item Type: Functional, Automatic
        \item Initial State: Random set of dice
        \item Input: Dice values
        \item Output: Expected chance score
        \item Test Case Derivation: Ensures proper scoring for chance
        \item How test will be performed: Provide dice values and check score
    \end{itemize}

\end{enumerate}

\subsection{Traceability Between Test Cases and Modules}

\begin{table}[H]
  \centering
  \begin{tabular}{p{0.4\textwidth} p{0.5\textwidth}}
    \toprule
    \textbf{Test Case} & \textbf{Modules} \\
    \midrule
    test\_roll\_dice & DynamicDiceContainer Module \\
    test\_roll\_selected\_dice & DynamicDiceContainer Module \\
    test\_invert\_dice\_selection & DynamicDiceContainer Module \\
    test\_read\_dice\_values & DynamicDiceContainer Module \\
    test\_toggle\_dice\_collisions & DynamicDiceContainer Module \\
    test\_full\_house\_scoring & ScoreCalculator Module \\
    test\_kind\_scoring & ScoreCalculator Module \\
    test\_straight\_scoring & ScoreCalculator Module \\
    test\_singles\_scoring & ScoreCalculator Module \\
    test\_chance\_scoring & ScoreCalculator Module \\
    \bottomrule
  \end{tabular}
  \caption{Table of Traceability Between Test Cases and Modules}
  \label{TblTraceability}
\end{table}

\begin{table}[H]
  \centering
  \begin{tabular}{|l|c|c|}
  \hline
    \textbf{Test Case} & \textbf{DynamicDiceContainer} & \textbf{ScoreCalculator} \\ \hline
    test\_roll\_dice & X & \\ \hline
    test\_roll\_selected\_dice & X & \\ \hline
    test\_invert\_dice\_selection & X & \\ \hline
    test\_read\_dice\_values & X & \\ \hline
    test\_toggle\_dice\_collisions & X & \\ \hline
    test\_full\_house\_scoring & & X \\ \hline
    test\_kind\_scoring & & X \\ \hline
    test\_straight\_scoring & & X \\ \hline
    test\_singles\_scoring & & X \\ \hline
    test\_chance\_scoring & & X \\ \hline
  \end{tabular}
  \caption{Matrix of Traceability Between Test Cases and Modules}
  \label{TblTraceability2}
\end{table}

\newpage
\bibliographystyle{plainnat}

\bibliography{../../refs/References}

\newpage

\section{Appendix}

This is where you can place additional information.

\subsection{Symbolic Parameters}

The definition of the test cases will call for SYMBOLIC\_CONSTANTS.
Their values are defined in this section for easy maintenance.

\subsection{Usability Survey Questions?}

%\wss{This is a section that would be appropriate for some projects.}
There will be ongoing discussion and interactions while running the multiplayer game, with the observer taking notes during the process. After gameplay, the user will be directed to a form where they can answer the following questions:

\begin{itemize}
	\item Gameplay Clarity and Usability
	\begin{itemize}
		\item What is your name?
		\item Was the tutorial or onboarding process intuitive?
		\item Were the game controls easy to understand?
		\item Did you encounter any areas of confusion while playing?
	\end{itemize}
	
	\item Engagement and Enjoyment
	\begin{itemize}
		\item How engaging did you find the game?
		\item Would you play this game again?
		\item What aspect of the game did you enjoy the most?
	\end{itemize}
	
	\item Multi-Player Interaction
	\begin{itemize}
		\item Did you play the health-scored variant?
		\item Did you feel like you were playing against another player?
		\item Was the multi-player aspect of the game engaging?
		\item What aspect of the multi-player game did you enjoy the most?
		\item What aspect of the multi-player game could be improved?
	\end{itemize}
	
	\item Difficulty and Challenge
	\begin{itemize}
		\item Did the game feel fair and balanced?
		\item Was there a particular strategy that felt overpowered?
		\item Did you feel that your decisions significantly impacted the outcome of the game?
	\end{itemize}
	
	\item Suggestions and Improvements
	\begin{itemize}
		\item If you have played previous versions of the game, what did you notice has changed?
		\item What changes would you suggest to improve the game experience?
		\item Are there any additional features or modifications you would like to see?
	\end{itemize}
	
	
\end{itemize}

\newpage{}
\section*{Appendix --- Reflection}

% \wss{This section is not required for CAS 741}

% The information in this section will be used to evaluate the team members on the graduate attribute of Lifelong Learning.

% The purpose of reflection questions is to give you a chance to assess your own
learning and that of your group as a whole, and to find ways to improve in the
future. Reflection is an important part of the learning process.  Reflection is
also an essential component of a successful software development process.  

Reflections are most interesting and useful when they're honest, even if the
stories they tell are imperfect. You will be marked based on your depth of
thought and analysis, and not based on the content of the reflections
themselves. Thus, for full marks we encourage you to answer openly and honestly
and to avoid simply writing ``what you think the evaluator wants to hear.''

Please answer the following questions.  Some questions can be answered on the
team level, but where appropriate, each team member should write their own
response:


\begin{enumerate}
  \item What went well while writing this deliverable? 
  \begin{itemize}
    \item We were able to use past year examples to set up our document and understand what contents to add to each section - Hemraj B
    \item Since we had already done the requirements and hazard analysis documentation, we had a better understanding of the project we are designing for. Additionally, we have a better idea of what the project will come to look like given we are developing our proof of concept. - John P.
  \end{itemize}
  \item What pain points did you experience during this deliverable, and how
    did you resolve them?
    \begin{itemize}
      \item Sections 3 and 4, were confusing in terms of us thinking that both essentially meant the same thing. We resolved this by looking at the previous year's examples and were able to figure out the differences between the two sections - Hemraj B
      \item We wanted to focus more attention to the proof of concept, but this deadline could come useful for near the end of the project. - John P.
      \item When it comes to choosing a challenge level, this is something that needs to be set with a meeting with the professor. We had initially had the project as advanced on the proposal, but the professor said everyone is expected to start with a general level. A meeting with the professor is something that would have helped this section. - John P.
    \end{itemize}
  \item What knowledge and skills will the team collectively need to acquire to
  successfully complete the verification and validation of your project?
  Examples of possible knowledge and skills include dynamic testing knowledge,
  static testing knowledge, specific tool usage, Valgrind etc.  You should look to
  identify at least one item for each team member.
  \begin{itemize}
    \item We will need to familiarize ourselves with testing methods that were mentioned in our document such as Coverlet and FxCop  - Hemraj B
    \item We are working with GoDot, so we need to do some exploration with what tools best work with that software. Additionally, since this is a game we are developing where most elements are driven by user interaction. This also brings some difficulties and thus something that we need to explore. - John P.
    \item As of right now there is no CI/CD element to our repository, especially given that 100\% of the commits to the main branch as of right now are for documentation. This is a skill we need to develop and brainstorm how we can best make use of this technology. - John P.
  \end{itemize}
  \item For each of the knowledge areas and skills identified in the previous
  question, what are at least two approaches to acquiring the knowledge or
  mastering the skill?  Of the identified approaches, which will each team
  member pursue, and why did they make this choice?
  \begin{itemize}
    \item We will acquire knowledge of Coverlet and FxCop through watching educational videos on YouTube and we will create trial projects to practice the software and familiarise ourselves with the tools available on each software - Hemraj B
    \item We need to do research on GoDot and how verification tools can work with it. This would come through looking at online resources that have some experience with GoDot and we can use their expertise, but could also come through locally testing different software methods and bringing them to the team once integrated. Similarly with testing mostly user driven interactions, we would need to brainstorm as a team and test options, but perhaps also bring it up to the professor and TAs who might have useful ideas. - John P.
    \item The problem of having CI/CD testing for games has been brought up in lecture, and as such is an issue the professor and TAs should be aware of. Asking for their advice and experience would be useul in this regard, as would searching online forums and videos for what others have done. This can take place at the same time as the above bullet point, since GoDot is primarily used for game development. - John P.
  \end{itemize}
\end{enumerate}

\end{document}