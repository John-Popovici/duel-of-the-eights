\documentclass[12pt, titlepage]{article}

\usepackage{amsmath, mathtools}

\usepackage[round]{natbib}
\usepackage{amsfonts}
\usepackage{amssymb}
\usepackage{graphicx}
\usepackage{colortbl}
\usepackage{xr}
\usepackage{hyperref}
\usepackage{longtable}
\usepackage{xfrac}
\usepackage{tabularx}
\usepackage{float}
\usepackage{siunitx}
\usepackage{booktabs}
\usepackage{multirow}
\usepackage[section]{placeins}
\usepackage{caption}
\usepackage{fullpage}

\hypersetup{
bookmarks=true,     % show bookmarks bar?
colorlinks=true,       % false: boxed links; true: colored links
linkcolor=red,          % color of internal links (change box color with linkbordercolor)
citecolor=blue,      % color of links to bibliography
filecolor=magenta,  % color of file links
urlcolor=cyan          % color of external links
}

\usepackage{array}

\externaldocument{../../SRS/SRS}

\input{../../Comments}
%% Common Parts

\newcommand{\progname}{SFWRENG 4G06} % PUT YOUR PROGRAM NAME HERE
\newcommand{\authname}{Team 9, dice\_devs
\\ John Popovici
\\ Nigel Moses
\\ Naishan Guo
\\ Hemraj Bhatt
\\ Isaac Giles} % AUTHOR NAMES                  

\usepackage{hyperref}
    \hypersetup{colorlinks=true, linkcolor=blue, citecolor=blue, filecolor=blue,
                urlcolor=blue, unicode=false}
    \urlstyle{same}
                                


\begin{document}

\title{Module Interface Specification for\\\progname:\\Dice Duels: Duel of the Eights}

\author{\authname}

\date{\today}

\maketitle

\pagenumbering{roman}

\section{Revision History}

\begin{table}[hp]
\caption{Revision History} \label{TblRevisionHistory}
\begin{tabularx}{\textwidth}{llX}
\toprule
\textbf{Date} & \textbf{Developer(s)} & \textbf{Change}\\
\midrule
20-01-10 & Hemraj Bhatt & Added content to section 3\\
20-01-11 & John Popovici & Added content to section 4; Updated section 2\\
\dots & \dots & \dots\\
\bottomrule
\end{tabularx}
\end{table}


\newpage

\tableofcontents

\newpage

\pagenumbering{arabic}

\section{Symbols, Abbreviations and Acronyms}

See \href{https://github.com/John-Popovici/duel-of-the-eights/blob/main/docs/SRS/SRS.pdf}{SRS Documentation} and \href{file:///C:/Users/a/Documents/Coding/duel-of-the-eights/docs/Design/SoftArchitecture/MG.pdf}{MG Documentation}.

%\wss{Also add any additional symbols, abbreviations or acronyms}

\section{Introduction}

The following document details the Module Interface Specifications for the Duel of the Eights game. The game takes inspiration from Yahtzee and introduces a platform that enables players to create custom Yahtzee-like game variants. This platform includes preset options such as classic Yahtzee and an octahedron version, while also offering flexibility for users to define their own game variables. Players can customize aspects such as the number and type of dice, ranging from cubed (6-sided) to octahedral (8-sided) and other multi-sided dice, as well as scoring mechanisms. Scoring can either follow the traditional end-of-game calculation seen in classic Yahtzee or adopt a per-round format, allowing for head-to-head matchups. The game can be played both locally and online, allowing players to be able to play with one another regardless of their distance from one another.
\wss{Fill in your project name and description}

Complementary documents include the System Requirement Specifications
and Module Guide.  The full documentation and implementation can be
found at \url{https://github.com/John-Popovici/duel-of-the-eights/tree/main}.  \wss{provide the url for your repo}

\newpage
\section{Notation}

% \wss{You should describe your notation.  You can use what is below as a starting point.}

The structure of the MIS for modules comes from \citet{HoffmanAndStrooper1995},
with the addition that template modules have been adapted from
\cite{GhezziEtAl2003}. Template used from \href{https://github.com/smiths/capTemplate}{the \progname GitHub}.\\
%The mathematical notation comes from Chapter 3 of \citet{HoffmanAndStrooper1995}.  For instance, the symbol := is used for a multiple assignment statement and conditional rules follow the form $(c_1 \Rightarrow r_1 | c_2 \Rightarrow r_2 | ... | c_n \Rightarrow r_n )$.

The following table summarizes the primary scene-building classes and elements used by \textit{Dice Duels: Duel of the Eights} through Godot. Many different types of nodes are used, each with specific functions and properties.

\begin{center}
\renewcommand{\arraystretch}{1.2}
\noindent 
\begin{tabular}{l l} 
\toprule 
\textbf{Object} & \textbf{Description}\\ 
\midrule
Object & Dynamic base class holding properties, methods, signals, and scripts.\\
Node & Inherits object, can be assigned as a child of another node.\\
Scene & A tree of nodes.\\
\bottomrule
\end{tabular} 
\end{center}

The following table summarizes the primary coding data types used by \textit{Dice Duels: Duel of the Eights} through Godot.

\begin{center}
\renewcommand{\arraystretch}{1.2}
\noindent 
\begin{tabular}{p{1.5cm} c p{10cm}} 
\toprule 
\textbf{Data Type} & \textbf{Notation} & \textbf{Description}\\ 
\midrule
null & \textit{null} & Denotes the absence of a value. \\
boolean & \textit{bool} & Holds one of 2 values, true or false.\\
integer & \textit{int} & A number without a fractional component within the bounds of signed 64-bit integer type [$-2^{63}$, $2^{63} - 1$].\\
%infinity & \textit{INF} or $\infty$ & denotes a number larger than the maximum representable. Can also be negative to denote a number smaller \\
%not a number & \textit{NAN} & a non-numerical value that has been defined as a number \\
float or decimal & \textit{float} & A decimal number, as representable by 64-bit double-precision floating-point.\\
string & \textit{String} or $"\dots"$ & A sequence of unicode characters.\\
signal & \textit{signal} & Allows connected objects to react to events.\\
\bottomrule
\end{tabular} 
\end{center}

The following table summarizes the primary data structures used by \textit{Dice Duels: Duel of the Eights} through Godot.

\begin{center}
\renewcommand{\arraystretch}{1.2}
\noindent 
\begin{tabular}{l l l} 
\toprule 
\textbf{Structure} & \textbf{Example} & \textbf{Description}\\ 
\midrule
array & $[5, "Hello", 2.3]$ & Sequence of data elements.\\
dictionary & $\{"White": 35, "Red": 10\}$ & Set of key and value pairs.\\
\bottomrule
\end{tabular} 
\end{center}

\noindent
The specification of \textit{Dice Duels: Duel of the Eights} uses some derived data types such as Vector3. Vector3 contains three floating point numbers, representing coordinates or direction vectors. In addition, \textit{Dice Duels: Duel of the Eights} uses functions, which are defined by the data types of their inputs and outputs. Local functions are described by giving their type signature followed by their specification.

\section{Module Decomposition}

The following table is taken directly from the Module Guide document for this project.

\begin{table}[h!]
\centering
\begin{tabular}{p{0.3\textwidth} p{0.6\textwidth}}
\toprule
\textbf{Level 1} & \textbf{Level 2}\\
\midrule

{Hardware-Hiding} & ~ \\
\midrule

\multirow{7}{0.3\textwidth}{Behaviour-Hiding} & Input Parameters\\
& Output Format\\
& Output Verification\\
& Temperature ODEs\\
& Energy Equations\\ 
& Control Module\\
& Specification Parameters Module\\
\midrule

\multirow{3}{0.3\textwidth}{Software Decision} & {Sequence Data Structure}\\
& ODE Solver\\
& Plotting\\
\bottomrule

\end{tabular}
\caption{Module Hierarchy}
\label{TblMH}
\end{table}

\newpage
~\newpage

\section{MIS of \wss{Module Name}} \label{Module} \wss{Use labels for
  cross-referencing}

\wss{You can reference SRS labels, such as R\ref{R_Inputs}.}

\wss{It is also possible to use \LaTeX for hypperlinks to external documents.}

\subsection{Module}

\wss{Short name for the module}

\subsection{Uses}


\subsection{Syntax}

\subsubsection{Exported Constants}

\subsubsection{Exported Access Programs}

\begin{center}
\begin{tabular}{p{2cm} p{4cm} p{4cm} p{2cm}}
\hline
\textbf{Name} & \textbf{In} & \textbf{Out} & \textbf{Exceptions} \\
\hline
\wss{accessProg} & - & - & - \\
\hline
\end{tabular}
\end{center}

\subsection{Semantics}

\subsubsection{State Variables}

\wss{Not all modules will have state variables.  State variables give the module
  a memory.}

\subsubsection{Environment Variables}

\wss{This section is not necessary for all modules.  Its purpose is to capture
  when the module has external interaction with the environment, such as for a
  device driver, screen interface, keyboard, file, etc.}

\subsubsection{Assumptions}

\wss{Try to minimize assumptions and anticipate programmer errors via
  exceptions, but for practical purposes assumptions are sometimes appropriate.}

\subsubsection{Access Routine Semantics}

\noindent \wss{accessProg}():
\begin{itemize}
\item transition: \wss{if appropriate} 
\item output: \wss{if appropriate} 
\item exception: \wss{if appropriate} 
\end{itemize}

\wss{A module without environment variables or state variables is unlikely to
  have a state transition.  In this case a state transition can only occur if
  the module is changing the state of another module.}

\wss{Modules rarely have both a transition and an output.  In most cases you
  will have one or the other.}

\subsubsection{Local Functions}

\wss{As appropriate} \wss{These functions are for the purpose of specification.
  They are not necessarily something that is going to be implemented
  explicitly.  Even if they are implemented, they are not exported; they only
  have local scope.}

\newpage

\bibliographystyle {plainnat}
\bibliography {../../../refs/References}

\newpage

\section{Appendix} \label{Appendix}

\wss{Extra information if required}

\newpage{}

\section*{Appendix --- Reflection}

\wss{Not required for CAS 741 projects}

The information in this section will be used to evaluate the team members on the
graduate attribute of Problem Analysis and Design.

\input{../../Reflection.tex}

\begin{enumerate}
  \item What went well while writing this deliverable? 
  \item What pain points did you experience during this deliverable, and how
    did you resolve them?
  \item Which of your design decisions stemmed from speaking to your client(s)
  or a proxy (e.g. your peers, stakeholders, potential users)? For those that
  were not, why, and where did they come from?
  \item While creating the design doc, what parts of your other documents (e.g.
  requirements, hazard analysis, etc), it any, needed to be changed, and why?
  \item What are the limitations of your solution?  Put another way, given
  unlimited resources, what could you do to make the project better? (LO\_ProbSolutions)
  \item Give a brief overview of other design solutions you considered.  What
  are the benefits and tradeoffs of those other designs compared with the chosen
  design?  From all the potential options, why did you select the documented design?
  (LO\_Explores)
\end{enumerate}


\end{document}