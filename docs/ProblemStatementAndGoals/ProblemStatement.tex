\documentclass{article}

\usepackage{tabularx}
\usepackage{booktabs}

\title{Problem Statement and Goals\\\progname}

\author{\authname}

\date{}

%% Comments

\usepackage{color}

\newif\ifcomments\commentstrue %displays comments
%\newif\ifcomments\commentsfalse %so that comments do not display

\ifcomments
\newcommand{\authornote}[3]{\textcolor{#1}{[#3 ---#2]}}
\newcommand{\todo}[1]{\textcolor{red}{[TODO: #1]}}
\else
\newcommand{\authornote}[3]{}
\newcommand{\todo}[1]{}
\fi

\newcommand{\wss}[1]{\authornote{blue}{SS}{#1}} 
\newcommand{\plt}[1]{\authornote{magenta}{TPLT}{#1}} %For explanation of the template
\newcommand{\an}[1]{\authornote{cyan}{Author}{#1}}

%% Common Parts

\newcommand{\progname}{SFWRENG 4G06} % PUT YOUR PROGRAM NAME HERE
\newcommand{\authname}{Team 9, dice\_devs
\\ John Popovici
\\ Nigel Moses
\\ Naishan Guo
\\ Hemraj Bhatt
\\ Isaac Giles} % AUTHOR NAMES                  

\usepackage{hyperref}
    \hypersetup{colorlinks=true, linkcolor=blue, citecolor=blue, filecolor=blue,
                urlcolor=blue, unicode=false}
    \urlstyle{same}
                                


\begin{document}

\maketitle

\begin{table}[hp]
\caption{Revision History} \label{TblRevisionHistory}
\begin{tabularx}{\textwidth}{llX}
\toprule
\textbf{Date} & \textbf{Developer(s)} & \textbf{Change}\\
\midrule
2024-09-12 & John Popovici & Added content to section 1.1, 1.2, and 1.4\\
2024-09-12 & John Popovici & Added content to section 4\\
2024-09-17 & John Popovici & Added content to section 1.3; Updated section 4 to better reflect extras\\
2024-09-17 & John Popovici & Added content to section 2 and 3\\
2024-09-19 & John Popovici & Updated section 2 and 3 to have measurable goals\\
2024-09-20 & John Popovici & Added content to the reflection section\\
2024-09-24 & John Popovici & Spelling and updated goals\\
2024-10-25 & John Popovici & Refined preset game goal as per TA feedback\\
\dots & \dots & \dots\\
\bottomrule
\end{tabularx}
\end{table}

\newpage
\section{Problem Statement}

%\wss{You should check your problem statement with the
%\href{https://github.com/smiths/capTemplate/blob/main/docs/Checklists/ProbState-Checklist.pdf}{problem statement checklist}.} 
%\wss{You can change the section headings, as long as you include the required information.}

\subsection{Problem}

Games are a staple of entertainment and are used to bring people together for both competition and fun.
There can often be a desire to share the experience of playing a game with someone but meeting can be hard or impossible in-person and so having an online version of popular games allows for such opportunities.

One such game is the game of Yahtzee, which while it does have online versions, are limited to the classic rule-set and do not allow for variants to be designed and played.
Yahtzee is a game many know yet is not often played, yet by leveraging the most interesting game mechanics and creating different variants, we could reinvigorate the genre while paying homage to the original.

As developers, we are aiming to create an environment in which Yahtzee and Yahtzee variants can be designed and played with some preset rule-sets to be used as a starting-off point. This would allow for both the classic game as well as different variants of it, such as using 8-sided dice, to be played with a friend online.

\subsection{Inputs and Outputs}

%\wss{Characterize the problem in terms of ``high level'' inputs and outputs.  
%Use abstraction so that you can avoid details.}
Upon running the program, a game user would be capable of selecting to play either preset game configurations, or to customize their own given a set of game rules they can select and modify. In starting the game, these game rules would be instantiated and the user could play the Yahtzee version selected with another online player.

In terms of interacting with the game, two players will exist and have the ability to interact with the current game state, allowing them to make decisions based on game events. The game state will be presented through a visual user interface displaying different elements of the game, including the dice and points system(s).

\subsection{Stakeholders}

Our stakeholders are primarily those people that would spend their time playing the game, falling into the following categories.

Traditional Yahtzee enthusiasts are familiar with the classic Yahtzee ruleset and are generally older and may prefer the simplicity and nostalgia of the original game. They may not be too tech-savvy, but still wish to revisit the game and connect with friends, possibly exploring customization in moderation, without overwhelming complexity.

Typical gamers can span from casual to more dedicated gamers, have experience with online multiplayer, and are looking for an engaging experience with a friend. Having a faster paced experience with more skill expression and instant feedback is something they would be interested in.


\subsection{Environment}

%\wss{Hardware and software environment}

The game will be made to run on Windows 10 and 11 devices and be run with an internet connection to enable online multiplayer. One of the players would act as the host with the other connecting such that they can play in real-time with an online connection.

\section{Goals}

For this project to be a success both as a capstone project and as a fun game, there are some goals we must meet.

\begin{itemize}
	\item Enjoyable game. The project is more than just a capstone, and we need the game to be an enjoyable experience.\\
	\textit{Measurement:} Based on user feedback, a minimum of 75\% consider the experience as enjoyable.

	\item Online multiplayer functionality. We need to be able to connect two concurrent players to play the game together.\\
	\textit{Measurement:} Two players can connect such that both players can affect the game state and both players are notified of the updates.
	
	\item Customizable game settings. Core game elements must be modifiable to create custom Yahtzee variants. As a goal we would need these options to be implemented:
	\begin{itemize}
        \item The number of dice each player have
        \item The type of dice used such as cube, octahedron, and more
        \item Whether scoring will be done only after completing the score card, or by comparing each hand in a head-to-head fashion
        \item A round timer
    \end{itemize}
	\textit{Measurement:} There are at least 3 options for number of dice, at least 3 options for die-types, at least 2 options for scoring methods, and an option to use a turn timer. These options would be all compatible with one another.
	
	\item Preset game settings. By having some preset game configurations, it would allow players to more quickly learn the game or jump into an environment that has been tested and proven to be more fun than other configurations and can guide players into how different mechanics affect the game both directly and indirectly.\\
	\textit{Measurement:} At least 3 preset game configurations would be available for players to load up and play. Testing presets will be a process including both internal and external testers ranking and proposing improvements to designed game modes.
	
	\item 3D dice rolling. Rolling the dice will need to be or look to be three dimensional to recreate the tactile feel of the original game.\\
	\textit{Measurement:} Dice will have the appearance of the preset die shape, and of being rolled, based on a minimum of 75\% of user feedback considering it so.
	
\end{itemize}

\section{Stretch Goals}

Since we wish this game to go beyond just a capstone, and have the game be worth playing, there are stretch goals we will aim to implement as well.

\begin{itemize}
	\item Local multiplayer. This would allow for players to play together on a single computer, but would require a different user interface and allow for different user interactions.\\
	\textit{Measurement:} The ability for 2 players to play together using a single interface and game instance without an internet connection.
	
	\item Singleplayer variants. A singleplayer game could be achieved either through a computer-run opponent in a game, or through a custom designed experience that could leverage the different environment.\\
	\textit{Measurement:} A single person can play at least 1 variant made specifically to be singleplayer without requiring a second human player to update the game state.
	
	\item Online matchmaking. The game would provide users with the option to connect to another concurrent user based on a matchmaking score.\\
	\textit{Measurement:} A player can connect to another unknown concurrent player who was selected as a compatible opponent.
	
	\item Saving custom game setting. Having this ability would allow for a user who created a custom game variant to save them for the ability to replay it without the need to recreate those specific settings.\\
	\textit{Measurement:} A custom game variant, as per the "Customizable game settings" goal and "More game setting customization" stretch goals, can be saved locally and loaded up to be played.
	
	\item More game setting customization. Besides the options in the goals section, some additional game customization options would include:
	\begin{itemize}
        \item The ability to customize what hands and options are on the scorecard
        \item The points each option on the scorecard would provide
        \item Additional methods to score each round or each game
        \item A system to allow for gambling such as doubling down on a hand or folding for less of a punishment
        \item More options discovered through play-testing and theory crafting
    \end{itemize}
	\textit{Measurement:} Additional game options outside the ones listed in the "Customizable game settings" goal would be available.
	
	\item Dice customization. Dice could be made to appear differently, either as a means for personalization or for aiding with different impairments. An example could be a dice with pips versus a dice with a numbered faces.\\
	\textit{Measurement:} At least 5 different dice appearance variants players can choose from, that would appear in the game.
	
	\item Post game statistics. This could allow for players to analyze a game after completion in a more quantitative manner, aiding in better understanding statistical probabilities.\\
	\textit{Measurement:} A post-game summary showing at least 3 key game stats, available after each game.
	
	\item Multi-platform support. While most gaming experiences are for windows, this would allow for the game to be run on more than just the Windows operating system, allowing for a wider audience.\\
	\textit{Measurement:} The game can be run on operating systems other than windows.
	
	\item Dice highlighting. This would aid in determining what dice are used when scoring.\\
	\textit{Measurement:} Dice used in scoring will be highlighted when appropriate.
	
\end{itemize}

\section{Challenge Level and Extras}

\iffalse
\wss{State your expected challenge level (advanced, general or basic).  The
challenge can come through the required domain knowledge, the implementation or
something else.  Usually the greater the novelty of a project the greater its
challenge level.  You should include your rationale for the selected level.
Approval of the level will be part of the discussion with the instructor for
approving the project.  The challenge level, with the approval (or request) of
the instructor, can be modified over the course of the term.}

\wss{Teams may wish to include extras as either potential bonus grades, or to
make up for a less advanced challenge level.  Potential extras include usability
testing, code walkthroughs, user documentation, formal proof, GenderMag
personas, Design Thinking, etc.  Normally the maximum number of extras will be
two.  Approval of the extras will be part of the discussion with the instructor
for approving the project.  The extras, with the approval (or request) of the
instructor, can be modified over the course of the term.}
\fi

Through the implementation of not just a single Yahtzee game variant, but by implementing a system where the user can create a custom game variant and connect to another online player to play it, we are looking to achieve the advanced challenge level. We would also provide some pre-set game variants that have been tested and were found to be more fun than others.

We will include extra elements to aid in better developing our game and allow us to design for a fun gaming experience, as we wish this game to not be just a capstone project, but a game we can be proud of and put out into the world. Following industry practices in game development, we wish to potentially do the following extras:

\begin{itemize}
	\item Focus groups, Surveys, and Interviews
	\item Usability testing
	\item User personas
	\item User documentation
\end{itemize}

\newpage{}

\section*{Appendix --- Reflection}

%\wss{Not required for CAS 741}

%The purpose of reflection questions is to give you a chance to assess your own
learning and that of your group as a whole, and to find ways to improve in the
future. Reflection is an important part of the learning process.  Reflection is
also an essential component of a successful software development process.  

Reflections are most interesting and useful when they're honest, even if the
stories they tell are imperfect. You will be marked based on your depth of
thought and analysis, and not based on the content of the reflections
themselves. Thus, for full marks we encourage you to answer openly and honestly
and to avoid simply writing ``what you think the evaluator wants to hear.''

Please answer the following questions.  Some questions can be answered on the
team level, but where appropriate, each team member should write their own
response:


\begin{enumerate}
    \item What went well while writing this deliverable?\\
    
	\begin{itemize}
		\item The team as a whole was excited to make our capstone a fun game, and our supervisor gave us a large degree of creative freedom. This meant that we had discussed game mechanics and what we envision from this project both as a group but also with the supervisor, and through gathering input from multiple people created the idea of a game with specific goals that we believe could work together. This passion also meant we were generating a lot of ideas that we could do but were not the main focus of the game, and the best of those are highlighted in the stretch goals section.
		\item Having a well defined template where all the necessary components have headings means that we knew what the expectation was of us and could work ahead of time on the required components.
		\item We had meetings, both informal and formal, to help discuss the developmental plan, which helped inform aspects of this deliverable as well.
	\end{itemize}
    
    \item What pain points did you experience during this deliverable, and how
    did you resolve them?
    
	\begin{itemize}
		\item The challenge level and extras were the hardest section to know what to fill it in with since we have no way of knowing what extras would be the best for our project or if they would even make sense. We tried to resolve this through making educated guesses, but were worried we might need to revisit this section later once we have a better idea of what extras would have the most impact.
		\item When discussing with the professor, we wanted this project to be in the advanced category, so we added given suggestions, but on looking for official approval of the project we were told we need to write it as general but "We can revisit this, but I'll need a better feel for how you are going to handle program families before we classify the project as Advanced." Since we did discuss this before, and because of the wording and intentions of having this family of games be advanced, we looked to resolve this in section 4 by updating it to mention we are looking to achieve the advanced level rather than outright saying we are.
		\item Because this project has a lot of player interactions through means that are not yet selected, but also because we needed it to not be too specific, deciding on what the inputs and outputs are to be was difficult. While in theory we know we need a player to select options, select which dice to reroll, see the dice, etc, writing this into inputs and outputs was complex. We mostly made references to "game states" and that players would interact with it as an input (since we did not want to specify how - whether though buttons, mouse drags, typing, etc) and also did something similar mentioning a graphical user interface that displays the game state as an output, since we do not know the exact methods other than that it will be graphical.
	\end{itemize}    
    
    \item How did you and your team adjust the scope of your goals to ensure
    they are suitable for a Capstone project (not overly ambitious but also of
    appropriate complexity for a senior design project)?
    
	\begin{itemize}
		\item We wanted to ensure our project achieved four things: be achievable, be challenging enough to meet the criteria for a capstone project (ideally at the advanced level), create the game our supervisor wanted, but also be a fun game outside the scope of this course. We believe we can best achieve this through a suggestion that we allow for users to customize game settings before playing. This would increase the challenge level while still being achievable, but most importantly it would allow all types of games to be created, including the specific one our supervisor may have requested, but also test different variants and make specially tailored presets that we believe would be the most fun to play.
		\item Since we had many ideas for this game, but recognized that having them all be implemented would be unrealistic, we decided which would be most important to achieve and which could be done after achieving the crucial ones. This was perfectly reflected in the requirement that we split goals and stretch goals, and allowed us to keep our project focused with the option of later achieving more.
		\item With the game customization options and the stretch goals including singleplayer variants, we can allow for every one of us to create our own version of the game once we achieve the base project. This is important all that much more to maintain passion for the project since everyone can, in the end, create the idealized game they envision, as long as we as a team create the base capstone project first.
	\end{itemize}    
    
\end{enumerate}  


\end{document}