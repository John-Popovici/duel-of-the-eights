\documentclass{article}

\usepackage{float}
\usepackage{hyperref}  % Required for clickable links
\usepackage{tcolorbox} % Required for info boxes
\restylefloat{table}

\usepackage{booktabs}
\usepackage{pgf}
\usepackage{graphicx}
\graphicspath{ {./images/} }

\date{}


\begin{document}

\newpage
\section{Introduction}

\subsection{Purpose of the Usability Testing Plan}
The purpose of this usability testing plan is to ensure that \textbf{Dice Duels}, a spin-off variant of Yahtzee, provides an intuitive, engaging, and accessible experience for players. Usability testing is a critical part of the development process, helping identify potential issues in user interactions, game mechanics, and overall player satisfaction. 

The testing plan aims to:
\begin{itemize}
    \item Evaluate the clarity and intuitiveness of the game's interface and controls.
    \item Identify pain points and difficulties encountered by new and experienced players.
    \item Measure engagement, fairness, and strategic depth across multiple playtesting sessions.
    \item Gather structured feedback through surveys, observations, and interviews to drive iterative improvements.
\end{itemize}

\subsection{Goals of Usability Testing}
The usability testing for \textbf{Dice Duels} is designed to assess the game’s overall user experience and ensure that it aligns with the expectations of its intended audience. The specific goals of the testing are as follows:

\begin{itemize}
    \item \textbf{User Experience Evaluation:} Determine whether the game mechanics, UI, and interactions are intuitive and easy to understand.
    \item \textbf{Engagement and Enjoyment:} Assess the level of player engagement and satisfaction through structured surveys and behavioral observations.
    \item \textbf{Game Flow and Clarity:} Identify whether players can smoothly progress through the game without unnecessary confusion or frustration.
    \item \textbf{Accessibility and Inclusivity:} Ensure that the game can be played by a wide range of users, including those unfamiliar with dice-based strategy games.
    \item \textbf{Game Balance and Challenge:} Measure the fairness of different game modes and customization mechanics, ensuring no overpowered strategies dominate gameplay.
\end{itemize}

This usability testing plan provides a structured approach to systematically gather user feedback, analyze test results, and implement necessary improvements in multiple development iterations.

\newpage

\section{Testing Methodology}

\subsection{Target Audience}
The usability testing for \textbf{Dice Duels} will involve a diverse group of participants to ensure the game is accessible and engaging for a broad audience. The target audience includes:
\begin{itemize}
    \item Casual players interested in dice-based and strategy games.
    \item Experienced board game players who are familiar with games like Yahtzee and Poker.
    \item Competitive players looking for strategic depth and risk-reward mechanics.
    \item First-time players unfamiliar with dice games, to assess learnability and onboarding effectiveness.
\end{itemize}
Participants will be selected based on their gaming experience and background to capture different perspectives and potential usability challenges.

\subsection{Types of Testing}
Usability testing for \textbf{Dice Duels} will follow a mixed-methods approach, using both quantitative and qualitative data collection methods.

\subsubsection{Controlled Playtesting Sessions}
\begin{itemize}
    \item Observing participants in structured testing sessions where they play the game without prior guidance.
    \item Recording user interactions, decision-making processes, and pain points encountered during gameplay.
    \item Identifying any UI/UX issues, such as unclear buttons, hard-to-understand game mechanics, or overwhelming decision points.
\end{itemize}

\subsubsection{Surveys and Feedback Forms}
\begin{itemize}
    \item Structured surveys to evaluate user experience, ease of use, engagement, and perceived fairness of game mechanics.
    \item A mix of Likert-scale questions, multiple-choice questions, and open-ended responses to capture a range of feedback.
    \item Surveys will be conducted after playtesting sessions and after a certain number of matches to assess long-term user retention factors.
\end{itemize}

\subsubsection{Interviews}
\begin{itemize}
    \item Conducting in-depth interviews with selected testers to understand their thoughts on game mechanics, UI design, and potential improvements.
    \item Allowing players to freely discuss what they enjoyed, what frustrated them, and what changes they would suggest.
\end{itemize}

\subsection{Testing Environment}
\begin{itemize}
    \item \textbf{In-Person Testing:} Conducted in a structured setting with direct observation, allowing for real-time feedback and note-taking.
    \item \textbf{Remote Testing:} Players will test the game on their own devices with instructions provided, and their responses will be collected through online surveys.
    \item \textbf{Device Compatibility:} Due to the scope of the project, testing will only be available for PC (Windows) users.
\end{itemize}

\newpage

\section{Testing Timeline and Iterative Process}

\subsection{Phase 1: Pre-Alpha Testing (Internal Team Feedback)}
\begin{itemize}
    \item Conduct initial playtests within the development team.
    \item Identify and resolve major usability blockers before external testing.
    \item Ensure that core game mechanics function as intended.
\end{itemize}
This section (3.1) has been completed as development has progressed and will not be covered as part of this document

\subsection{Phase 2: Alpha Testing (First External Playtesting Round)}
\begin{itemize}
    \item Conduct structured testing with a small group of external testers.
    \item Focus on onboarding clarity, UI intuitiveness, and core mechanics usability.
    \item Collect structured feedback via surveys and interviews.
    \item Implement major usability fixes based on feedback.
\end{itemize}

\subsection{Phase 3: Beta Testing (Expanded Playtesting)}
\begin{itemize}
    \item Expand playtesting to a wider audience with varied experience levels.
    \item Collect observational data and surveys to measure engagement and game balance.
    \item Focus on difficulty curve, fairness, and user enjoyment.
    \item Prioritize fixes for UI enhancements, game balance, and quality-of-life changes.
\end{itemize}

\subsection{Phase 4: Final Playtesting and Validation}
\begin{itemize}
    \item Conduct final testing with a mix of first-time and experienced players.
    \item Ensure that all usability concerns raised in previous phases have been addressed.
    \item Make final refinements based on testing outcomes.
\end{itemize}

This structured usability testing methodology ensures that feedback is collected systematically and applied iteratively, leading to an engaging and accessible final version of \textbf{Dice Duels}.

\newpage

\section{Feedback Collection and Analysis}

The usability testing for \textbf{Dice Duels} will involve systematic feedback collection using multiple methods, including observations, structured surveys, and post-playtesting interviews. This section outlines the strategies for gathering, analyzing, and utilizing feedback to enhance the game’s usability and overall player experience.

\subsection{Observation Metrics}
During playtesting sessions, key gameplay interactions and player behaviors will be observed and recorded to assess usability. The following metrics will be tracked:
\begin{itemize}
    \item \textbf{Time to Learn Controls:} How long it takes players to understand the game’s mechanics and UI.
    \item \textbf{Common Areas of Confusion:} Identifying sections of the game where players consistently struggle.
    \item \textbf{Frequency of UI Misclicks:} Tracking incorrect button selections or unexpected interactions with the UI.
    \item \textbf{Player Behavior During Decision-Making:} Observing how players react to in-game choices, such as rolling dice, raising/folding, and selecting scores.
    \item \textbf{Game Completion Rate:} Identifying how many players finish a full game without quitting early.
    \item \textbf{Engagement Indicators:} Monitoring non-verbal cues (for in-person testing) or gameplay analytics (for remote testing) to measure enjoyment and frustration.
\end{itemize}

\subsection{Survey Structure}
After each playtesting session, participants will complete a structured survey designed to collect both quantitative and qualitative data on their experience. The survey will be divided into the following sections:

\subsubsection{Gameplay Clarity and Usability}
\begin{itemize}
    \item Was the tutorial or onboarding process intuitive? (Yes/No)
    \item Were the game controls easy to understand? (Likert Scale: 1-5)
    \item Did you encounter any areas of confusion while playing? (Open-ended)
\end{itemize}

\subsubsection{Engagement and Enjoyment}
\begin{itemize}
    \item How engaging did you find the game? (Likert Scale: 1-5)
    \item Would you play this game again? (Yes/No)
    \item What aspect of the game did you enjoy the most? (Open-ended)
\end{itemize}

\subsubsection{Multi-Player Interaction}
\begin{itemize}
    \item Did you play the health scored variant? (Yes/No)
    \item Did you feel like you were playing against another player? (Yes/No)
    \item Was the multi-player aspect of the game engaging? (Likert Scale: 1-5)
    \item What aspect of the multi-player game did you enjoy the most? (Open-ended)
    \item What aspect of the multi-player game could be improved? (Open-ended)
\end{itemize}

\subsubsection{Difficulty and Challenge}
\begin{itemize}
    \item Did the game feel fair and balanced? (Likert Scale: 1-5)
    \item Was there a particular strategy that felt overpowered? (Open-ended)
    \item Did you feel that your decisions significantly impacted the outcome of the game? (Yes/No)
\end{itemize}

\subsubsection{Suggestions and Improvements}
\begin{itemize}
    \item What changes would you suggest to improve the game experience? (Open-ended)
    \item Are there any additional features or modifications you would like to see? (Open-ended)
\end{itemize}

\subsection{Interview Structure}
In addition to structured surveys, a subset of players will participate in post-game interviews to provide deeper insights into their experience. The interviews will follow a semi-structured format, covering topics such as:

\begin{itemize}
    \item \textbf{First Impressions:} How did the game feel upon first playing it?
    \item \textbf{Favorite and Least Favorite Features:} What aspects of the game did you enjoy the most and least?
    \item \textbf{Game Flow and Pacing:} Did the game feel too slow, too fast, or well-paced?
    \item \textbf{Strategic Depth:} Did the game provide enough meaningful choices to feel strategic and engaging?
    \item \textbf{Multi-Player System:} Did the multi-player aspect feel engaging and fun?
    \item \textbf{Final Thoughts:} Is this a game you would recommend to others? Why or why not?
\end{itemize}

\newpage

\section{Feedback Analysis and Implementation}
Once feedback has been collected through observations, surveys, and interviews, the next step is to analyze the data and prioritize changes based on the findings. 

\subsection{Categorization of Feedback}
Feedback will be categorized into three priority levels:
\begin{itemize}
    \item \textbf{Critical Issues:} Game-breaking bugs, major usability issues, and severe balance problems that must be addressed immediately.
    \item \textbf{Moderate Issues:} Minor bugs, usability concerns, and improvements that would significantly enhance the user experience but do not prevent gameplay.
    \item \textbf{Minor Suggestions:} Quality-of-life improvements, aesthetic preferences, and feature requests that are nice-to-have but not essential.
\end{itemize}

\subsection{Iteration Plan}
The collected feedback will be used to iterate on game design in multiple development phases:
\begin{itemize}
    \item \textbf{Immediate Fixes (Next Development Cycle):} Address critical issues that impact gameplay.
    \item \textbf{Mid-Term Adjustments (Next Playtesting Phase):} Implement moderate usability improvements and balance adjustments.
    \item \textbf{Long-Term Refinements (Final Release):} Integrate minor suggestions, polish animations, and enhance overall aesthetics.
\end{itemize}

\newpage

\section{Usability Testing Report}
After each playtesting phase, a usability testing report will be generated summarizing the following and appended to this document:
\begin{itemize}
    \item Key findings from observations, surveys, and interviews.
    \item A prioritized list of usability issues and suggestions.
    \item Changes implemented in response to feedback.
    \item Recommendations for the next phase of testing.
\end{itemize}

This structured approach ensures that the usability testing process for \textbf{Dice Duels} leads to continuous improvements, resulting in a refined and polished gameplay experience.

\newpage

\section{Dice Duels - Alpha Playtest Invitation}

Hello, and thank you for helping us test \textbf{Dice Duels}!

This is an early Alpha build, and we appreciate any feedback you can provide. Please follow the steps below to play and complete the survey.

\subsection{Download \& Setup Instructions}
\begin{enumerate}
    \item Download the game here: \href{https://drive.google.com/file/d/1eDyBru4fAjn-ErjmaTQIKcBtgJJSfgcL/view?usp=sharing}{\textbf{Download Link}}.
    \item Unzip the file to a folder on your computer.
    \item Open the extracted folder and run \texttt{duel-of-the-eights.exe} to start the game.
\end{enumerate}

\subsection{How to Play (2-Player Online Game)}
\begin{enumerate}
    \item One player must \textbf{Host} the game, while the second player \textbf{Joins} using a \textbf{Connect Code}.
    \item The host should \textbf{copy the code and share it} with the second player.
    \item Once both players are connected, the host can \textbf{start the game}.
\end{enumerate}

\textbf{Note:} The Alpha Build can only be played if both players are on the same WiFi/Network.

\begin{tcolorbox}[colback=gray!10, colframe=black, title=Tip]
You can use \textbf{Discord, WhatsApp, or any messaging app} to quickly share the Connect Code with your opponent.
The host can click on the code button to automatically copy, to make copy/pasting easier
\end{tcolorbox}

\subsection{Provide Feedback (Survey Link)}

Once you’ve played a full game (or multiple rounds), please take \textbf{5 minutes} to fill out our feedback survey:

\href{https://forms.office.com/r/87fhfHWjjW}{\textbf{Microsoft Forms Link}}

Any feedback is valuable, whether it's about \textbf{UI, game mechanics, bugs, or suggestions}!

\subsection{Need Help?}
If you run into issues or have any questions, feel free to reply to this message.

\bigskip

\centering
\textbf{Thank you for your time and support!}


\end{document}
