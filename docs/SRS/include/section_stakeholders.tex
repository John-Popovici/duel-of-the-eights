\section{Stakeholders}

The stakeholders for this game project are those with vested interests in its development, release, and continued use. While there may be individuals with vested interest in \textit{Dice Duels: Duel of the Eights}, they all fall within the broader categories below and generally act as a member of the whole rather than an individual stakeholder. Stakeholders for the capstone but not for the project itself are included in some sections below, but are not considered a stakeholder of the project due to not influencing the direction and development of the game outside of requiring a certain challenge level and documentation.

\subsection{Traditional Yahtzee enthusiasts}

\begin{itemize}
	\item \textbf{Demographics:} Generally older players who have nostalgia for the classic Yahtzee game. They may have experience with the physical dice game playing with family and friends and seek a familiar experience to bring back memories and connect with people. 
	\item \textbf{Technical Comfort Level:} May be lacking in experience with technology and be more comfortable with straightforward,  user-friendly interfaces and are likely to appreciated clear instructions.
	\item \textbf{Motivations:}
	\begin{itemize}
		\item \textit{Nostalgia.} They seek to relive the classic Yahtzee experience in a more accessible format.
		\item \textit{Social connection.} They are interested in playing with friends they may not be able to meet in person.
		\item \textit{Relaxed gameplay.} They generally prefer low-stakes slower-paced gameplay that doesn't require complex strategies or fast decision making. They are more likely to appreciate a more relaxed experience.
	\end{itemize}
	\item \textbf{Preferences:}
	\begin{itemize}
		\item \textit{Classic game mode.} They'll likely be drawn to a game mode that replicates the original Yahtzee experience.
		\item \textit{Simplicity over customization.} While they may be willing to try out some customization, they may prefer options that don't stray too far from the classic.
		\item \textit{Minimalistic design.} An interface that's easy on the eyes with intuitive navigation will help make the experience enjoyable for them.
	\end{itemize}
\end{itemize}

\subsection{Video Game enthusiasts}

\begin{itemize}
	\item \textbf{Demographics:} A diverse group spanning casual players to more experienced players focused on optimization, they are familiar with online multiplayer games and are comfortable with technology and video games.
	\item \textbf{Technical Comfort Level:} Most typical gamers are comfortable with technology, online multiplayer, and faster-paced games. They enjoy exploring game mechanics and may be open to more complex overlapping game mechanics. Are comfortable downloading a new video game with straightforward procedures.
	\item \textbf{Motivation:}
	\begin{itemize}
		\item \textit{Challenge and skill expression.} They're interested in games that allow for strategy and competitive play where skill and quick decision-making have an impact on the outcome.
		\item \textit{Engagement.} Video game enthusiasts seek a fast-paced experience with more action and reward cycles that provide instant feedback and keep them engaged.
		\item \textit{Replayability.} They generally enjoy experiences with unique challenges every time they play and can develop strategies over multiple runs.
	\end{itemize}
	\item \textbf{Preferences:}
	\begin{itemize}
		\item \textit{Customization and variability.} They are likely to appreciate the ability to customize gameplay to explore possibilities and develop a game system that is challenging but enjoyable for them.
		\item \textit{Faster game modes.} They may enjoy options that speed up gameplay, such as round timers or different scoring methods that add intensity and variety.
		\item \textit{Reactive feedback.} Faster-paced action and instant feedback to keep them engaged would be preferred over slower gameplay.
	\end{itemize}
\end{itemize}

\subsection{Personas}

\begin{itemize}
	\item \textbf{Joan, 55 years old (The Nostalgic Yahtzee Player)} Joan enjoys simple board games that remind her of family gatherings. She values a straightforward interface, classic gameplay, and the option to play casually with friends over the internet when they cannot meet in person.
	\item \textbf{James, 24 years old (The Casual Gamer)} James plays games on the weekends with friends. He enjoys the flexibility of customizable game modes and fast-paced rounds. Alex values multiplayer gameplay with friends and occasional solo play.
	\item \textbf{Julian, 19 years old (The Competitive Gamer)} Julian enjoys strategic games that involve skill, competition, and learning over time. He prefers playing with friends that are similarly competitive, customization options, and post-game statistics to analyze his performance.
\end{itemize}

\subsection{Priorities Assigned to Users}

\begin{itemize}
	\item \textbf{Primary Priority:} Casual and competitive video game enthusiasts would be the primary priority as they are the most plentiful and would be most likely to come across the game to play it. There is currently no game that provides customizable Yahtzee-like mechanics, and with feedback and testing, the game can be made to tailor to their preferences and be engaging.
	\item \textbf{Secondary Priority:} Traditional Yahtzee enthusiasts are a secondary priority in that using customization mechanics their preferred classic game can be recreated, but the game system overall would be more than just that. Their needs for a simple and intuitive user interface must be taken into consideration, as must options be available for their experience to likewise be enjoyable.
	\item \textbf{Tertiary Priority:} Outside of the context of stakeholders for the program, there are stakeholders for the capstone project. Their considerations do not shape the game itself as they are unlikely to play the game, but rather require a specific challenge level and documentation to be produced. As such, they can be considered tertiary despite not being stakeholders to the game unless also included in one of the above stakeholder groups.
\end{itemize}

\subsection{User Participation}

\begin{itemize}
	\item \textbf{Development Phase:} Selected playtesters and the development team will regularly engage with the game throughout the development cycle to identify bugs and shape the user experience.
	\item \textbf{Testing Phase:} Both playtesters and target user groups will participate in beta testing to validate usability, customization options, and multiplayer functionality.
	\item \textbf{Feedback Phase:} In the feedback phase, when functionality is mostly validated, users will provide feedback through surveys and interviews to refine the game before a final release.
\end{itemize}

\subsection{Ongoing Support}

This project will be released with a final build that is complete and of quality. Since this game will also be released under a permissive license, the codebase will be publicly available on GitHub, allowing others to further develop it and add features they themselves may wish to have available, or fix any bugs that may be present. The current developers may also be available in their free time if this project or a spin-off is something that will continue past the capstone project timeframe.
