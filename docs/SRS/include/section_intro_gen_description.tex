\section{Introduction}

Do you know the game Yahtzee? Played with dice, it's like Poker but more varied, requiring different skills, without betting. It uses 5 dice, with the usual dots representing the numbers 1 to 6. In rolling the dice, players try to create specific formations, thereby scoring points. At the end of a set number of rounds of dice rolls, the player with the higher score wins. But the cube isn't the only possibility for dice as the octahedron, with 8 sides could be used. Scoring could likewise be done each round. The number of dice could be changed. These suggest an expanded version of Yahtzee.\\

This project creates an online multiplayer game platform that allows for the creation of custom Yahtzee-like games and variants. This family of games will come with some presets such as classic Yahtzee, Dr. Paul's octahedron version, and more, but will allow for the users to set their own variables such as number of dice, what kind of dice, and some elements of scoring, and then play that game. Kinds of dice would include cubes and octahedrons, among other multi-sided dice, and scoring could be calculated at the end of the game, as in classic Yahtzee, or on a per-round basis, where hands go in a head-to-head matchup.\\

This section includes a general overview of the entire SRS document providing descriptions of all sections. It also outlines of the areas of knowledge needed to grasp the documentation accurately.

\subsection{Purpose of Document}

  
The Software Requirements Specification (SRS) document aims to clearly define the functional and non-functional needs of the project. So all parties involved have a common understanding of the objectives and requirements of the project. Throughout the whole software lifecycle, the SRS will serve as the development team's fundamental guide, aiding in the phases of design, implementation, and testing. In addition, it will facilitate effective communication between stakeholders and the development team while creating a framework that ensures that the final result satisfies user requirements and is in line with the project's pre-determined goals.

\subsection{Scope of Requirements} 

The scope of the project involves developing a modified Yahtzee game that features adjustable numbers of playable dice and various game attribute variations. This game will be available for both offline and online play, offering single-player and multiplayer. The scope does not include the implementation of advanced graphics, advanced computer opponents and other board games.

\subsection{Characteristics of Intended Reader} \label{sec_IntendedReader}

The intended readers of this SRS document are mainly software developers and game designers with expertise in game development and design using game engines, in this case, Godot. They should possess a solid understanding of game mechanics, particularly those related to dice games, in this case, Yahtzee, and have experience in understanding how multiplayer games work. Additionally, experience with C\# and .NET is necessary for readers due to the development being done using C\# on a .NET framework. A high school level of probability theory understanding is also recommended for comprehending the game's underlying mechanics and probabilities due to the differing number of playable dice. It is assumed that readers are well-versed in these areas as it will allow readers to understand certain decisions made by the team.

\subsection{Organization of Document}

This SRS document is structured to provide a clear roadmap for readers. The sections are as follows:

\begin{itemize}
    \item \textbf{Introduction} 
    \begin{itemize}
        \item Outlines the purpose and scope of the SRS.
    \end{itemize}
    
    \item \textbf{General System Description} 
    \begin{itemize}
        \item Outlines an overview of the project, along with user characterstics, system constraint and the interfaces between the system and its environment.
    \end{itemize}
    
    \item \textbf{Specific System Description} 
    \begin{itemize}
        \item Indepth system description containg  high-level probelem and goal description, along with solutions characteristics, assumptions, definiations and instance models  detailed functionalities and features of the system.
    \end{itemize}
    
    \item \textbf{Requirements} 
    \begin{itemize}
        \item Outlines both functional and non-functional requirements of the system.
    \end{itemize}
    
    \item \textbf{Likely Changes} 
    \begin{itemize}
        \item Outlines anticipated modifications to the system.
    \end{itemize}
    
    \item \textbf{Unlikely Changes} 
    \begin{itemize}
        \item Outlines ascpects of the system that will remain static.
    \end{itemize}
    
    \item \textbf{Traceability Matrices and Graphs} 
    \begin{itemize}
        \item Tracking of requirements throughout the development process.
    \end{itemize}
    
    \item \textbf{Development Plan} 
    \begin{itemize}
        \item Outlines the general timeline and milestones for the project.
    \end{itemize}
    
    \item \textbf{Values of Auxiliary Constants} 
    \begin{itemize}
        \item Outlines constant parameters used in report.
    \end{itemize}
    
    \item \textbf{Commonalities} 
    \begin{itemize}
        \item Outlines commonalities between differnt aspects of the system.
    \end{itemize}
    
    \item \textbf{Variabilities} 
    \begin{itemize}
        \item Outlines variables between differnt aspects of the system.
    \end{itemize}
    
    \item \textbf{Parameters of Variations} 
    \begin{itemize}
        \item Outlines the different variations of aspects of the system.
    \end{itemize}
    
    \item \textbf{Appendix — Reflection} 
    \begin{itemize}
        \item Indepth insights into the development process and lessons learned.
    \end{itemize}
\end{itemize}


\newpage
\section{General System Description}

This section provides general information about the system.  It identifies the
interfaces between the system and its environment, describes the user
characteristics and lists the system constraints.

\subsection{System Context}

System context includes an abstract view of the software following the design pattern of Inputs → Calculations → Outputs. The user provides inputs, such as game settings and preferences, the system performs the necessary calculations (e.g., score tracking, dice roll simulation), and then outputs the results to the user. The system will also interact with external entities, such as players in a multiplayer environment.\\
\\
\noindent The following high-level requirements are relevant to this context:

\begin{itemize}
    \item The game must support online multiplayer functionality, allowing two players to play against each other.
    \item The system must calculate scores based on standard Yahtzee rules.
    \item The game will feature realistic dice rolling physics and random outcomes.
    \item The game will provide a user interface that displays essential game information (scores, player names, dice states).
    \item Players will be able to customize game settings and access different game modes.
\end{itemize}

\noindent\textbf{Inputs:}
  \begin{itemize}
        \item Game settings (number of dice, player names, game modes).
        \item Player actions (dice selection, roll decisions).
  \end{itemize}
  
  \noindent\textbf{Outputs:}
  \begin{itemize}
      \item Game results (scores, dice outcomes).
      \item Feedback on game progress (turns, current scores, remaining rolls).
  \end{itemize}
  
  \noindent\textbf{External Entities:}
  \begin{itemize}
      \item Other players (for online multiplayer functionality).
      \item Peer-to-peer networking services for connecting players.
  \end{itemize}

  \noindent\textbf{User Responsibilities:}
\begin{itemize}
    \item Provide valid game input data, such as names and settings.
    \item Make strategic decisions regarding dice rolls and game modes.
\end{itemize}

\noindent\textbf{System Responsibilities:}
\begin{itemize}
    \item Ensure that all input data is properly validated (e.g., detect type mismatches).
    \item Perform calculations for scorekeeping and game logic.
    \item Provide feedback and results to the user in an intuitive interface.
\end{itemize}

\noindent Additionally, the system is intended for casual use, primarily focused on entertainment and social interaction. It is not a mission-critical or safety-critical system, which will influence the degree of formality in the system design.

\subsection{User Characteristics} \label{SecUserCharacteristics}

This section summarizes the knowledge and skills expected of the user:
  
\begin{itemize}
    \item The end user of the Yahtzee 3D game should have an understanding of the Yahtzee game, including its rules and strategies.
    \item Users should have a rudimentary (elementary school level) understanding of probability theory, as it is important for comprehending the game's mechanics and making informed decisions during gameplay.
    \item Familiarity with casual gaming and turn-based gameplay will be beneficial for the user's overall enjoyment and understanding of the pacing of the game.
    \item Basic digital literacy is expected, particularly in understanding how to adjust game settings, access help menus, or troubleshoot simple connectivity issues.
\end{itemize}

\subsection{System Constraints}

\noindent The following constraints must be adhered to during the development of the system:
  \begin{itemize}
    \item The game must use the Godot engine exclusively for its development.
    \item The codebase must be written in C\# and  use .NET framework exclusively.
    \item The system must support peer-to-peer and multiplayer capabilities.
  \end{itemize}