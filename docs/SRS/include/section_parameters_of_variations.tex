\section{Parameters of Variations}

This section details the specific values that the variabilities between the different games in the product family can take, corresponding to the variabilities outlined in the previous section.

\subsection{Number of Dice}
The number of dice used in the game can range from 1 to a maximum based on testing results. Having a high maximum would allow for more customization options, but a reasonable limit based on what is realistically usable and fun but also based on physics simulation performace needs to be set. This allows for variations in gameplay complexity, depending on the specific variant.

\subsection{Sides on Dice}
The usable dice will be a subset of dice modeled with the following number of sides: 4, 6, 8, 10, 12. This flexibility in dice sides introduces variability in the probability and potential outcomes for each roll.

\subsection{Individual Sides}
The values displayed on the individual sides of each die can range from 1 to the respective number of sides on that die and can be displayed in different representations. This variability allows for different numerical or symbolic configurations to match specific game variants.

\subsection{Scoring Calculation}
Each hand can be assigned any integer score value or modified by factors based on the dice rolls. This provides flexibility in defining how each hand is valued in the different game variants.

\subsection{Time Per Turn}
The time allotted for each turn can range from 5 seconds to 2 minutes, or be set to unlimited. This parameter affects the pacing of the game, allowing for both fast-paced and more thoughtful gameplay.

\subsection{Hand Restrictions}
Hands can be restricted by removing up to all but one from play. Additionally, hands can be allowed to be repeated between 1 to 10 times or an unlimited number of times, giving options for different scoring strategies.

