\section{Specific System Description}

This section presents the problem description, which gives a high-level
view of the problem to be solved.  This is followed by the solution characteristics
specification, which presents the assumptions, theories, definitions and finally
the instance models.

\subsection{Problem Description} \label{Sec_pd}

Games are a staple of entertainment and are used to bring people together for both competition and fun.
There can often be a desire to share the experience of playing a game with someone but meeting can be hard or impossible in-person and so having an online version of popular games allows for such opportunities.
One such game is the game of Yahtzee, which while it does have online versions, are limited to the classic rule-set and do not allow for variants to be designed and played.
\textit{Dice Duels: Duel of the Eights} looks to solve the issue of not having online access to Yahtzee and the ability to create custom versions of the game and play them.

\subsubsection{Terminology and Definitions}

This subsection provides a list of terms that are used in the subsequent
sections and their meaning, with the purpose of reducing ambiguity and making it
easier to correctly understand the requirements:

\begin{itemize}

\item Dice rolls can be given in the form $AdX$ where $A$ and $X$ are variables representing the number of dice and number of sides, respectively. When $A$ is $1$, its inclusion is optional. For example $4d6$ represents rolling 4 6-sided dice, and $d12$ represent 1 die with 12 sides. These terms can also be used in contexts such as "we are playing with d8s".

\item Since a fundamental mechanic of \textit{Dice Duels: Duel of the Eights} is that of re-rolling dice, an additional variable will be added to the notation to denote the number of rolls, $AdXrB$. For example, typical Yahtzee with 3 rolls would be denoted as $5d6r3$.

\item Limits on dice parameters can be expressed as inequalities. For example:
\begin{itemize}
	\item $1 \geq A \geq A_{max}$ where $A$ is the number of dice and $A_{max}$ is the maximum allowable dice.
	\item $1 \geq X \geq X_{max}$ where $X$ is the number of dice sides and $X_{max}$ is the maximum allowable dice sides, in values acceptable and modeled by the system.
	\item $1 \geq B \geq B_{max}$ where $B$ is the number of dice rolls and $B_{max}$ is the maximum allowable dice rolls.
\end{itemize}
These limits can vary by game mode or player-defined settings.

\item Dice typically take the forms of platonic solids, meaning where the dice faces are congruent regular polygons. The five such polyhedra are:
\begin{itemize}
	\item A tetrahedron has four faces ($d4$)
	\item A cube has six faces ($d6$)
	\item An octahedron has eight faces ($d8$)
	\item A dodecahedron has twelve faces ($d12$)
	\item An icosahedron has twenty faces ($d20$)
\end{itemize}
Other die shapes can be used such as a pentagonal trapezohedron with ten faces ($d10$) or even dice meant to be rolled lengthwise such as a triangular prism which despite having five total faces, is used as having three ($d3$), or has caps to prevent rolling an unintended face.

\item Dice rolls indicate a value integer when rolled. These can be represented by the numeric value of the integer or by representing the integer value as dots, called pips.

\item In typical Yahtzee some patterns for scoring have terms. These will have to be added to and abstracted for a game with a different number of dice, but for a $5d6r3$ these would be the scoring opportunities:
\begin{itemize}
	\item Rolling for aces, twos, threes, fours, fives, or sixes is rolling for as many of the number.
	\item Chance is any combination of dice as a sum of all dice values.
	\item A yahtzee is rolling all five of five dice with matching faces.
	\item A three of a kind is having at least three of five dice match. For four of a kind, it is having at least four dice the same.
	\item A straight is a set of sequential dice values. This can come in the small straight variant with four sequential values, or a large straight where all five dice are part of a sequence.
	\item A full house is having two dice of a kind and three of another.
\end{itemize}

\item A gameplay mechanic where players perform their turns concurrently with outcomes revealed simultaneously is referred to as a "simultaneous turn-based mechanism" in this document.

\item This document makes reference to "game settings" and "game modes" where game settings are some specific settings that players will be able to customize, such as the number and type of dice, scoring methods, time limits, etc. and any permutation of these settings can define a "game mode".
\end{itemize}

\iffalse
\subsubsection{Physical System Description} \label{sec_phySystDescrip}

\plt{The purpose of this section is to clearly and unambiguously state the
  physical system that is to be modelled. Effective problem solving requires a
  logical and organized approach. The statements on the physical system to be
  studied should cover enough information to solve the problem. The physical
  description involves element identification, where elements are defined as
  independent and separable items of the physical system. Some example elements
  include acceleration due to gravity, the mass of an object, and the size and
  shape of an object. Each element should be identified and labelled, with their
  interesting properties specified clearly. The physical description can also
  include interactions of the elements, such as the following: i) the
  interactions between the elements and their physical environment; ii) the
  interactions between elements; and, iii) the initial or boundary conditions.}

\plt{The elements of the physical system do not have to correspond to an actual
physical entity.  They can be conceptual.  This is particularly important when
the documentation is for a numerical method. }

The physical system of \progname{}, as shown in Figure~?,
includes the following elements:

\begin{itemize}

\item[PS1:] 

\item[PS2:] ...

\end{itemize}

\plt{A figure here makes sense for most SRS documents}
\fi

\subsubsection{Goal Statements}

These primary goals should be achieved in the development of our system, providing criteria for completeness. We have additionally organized stretch goals for further development, but they are not to be a metric for system completeness.

\iffalse
\plt{The goal statements refine the ``Problem Description''
  (Section~\ref{Sec_pd}).  A goal is a functional objective the system under
  consideration should achieve. Goals provide criteria for sufficient
  completeness of a requirements specification and for requirements
  pertinence. Goals will be refined in Section “Instanced Models”
  (Section~\ref{sec_instance}). Large and complex goals should be decomposed
  into smaller sub-goals.  The goals are written abstractly, with a minimal
  amount of technical language.  They should be understandable by non-domain
  experts.}
\fi

The goal statements are:

\begin{itemize}

	\item[GS\refstepcounter{goalnum}\thegoalnum \label{G_enjoyable}:] Enjoyable game. The project is more than just a capstone, and we need the game to be an enjoyable experience.
	\begin{itemize}
        \item User feedback collection. Implement a simple feedback system such as rating or comments for use in interviews or surveys to gather player insight.
        \item Testing iteration. Conduct at least two rounds of user testing to identify and iteratively improve the system based on user experience.
        \item Quality assets. Ensure that graphics, animations, and sounds are of a quality to add positively to the overall enjoyment.
    \end{itemize}
	\textit{Measurement:} Based on user feedback, a minimum of 75\% consider the experience as enjoyable.

	\item[GS\refstepcounter{goalnum}\thegoalnum \label{G_multiplayer}:] Online multiplayer functionality. We need to be able to connect two concurrent players to play the game together.
	\begin{itemize}
        \item Connection setup. Develop a server-client connection system where two players can connect.
        \item Game state synchronization. Each player's actions are to be reflected to both user outputs using real-time synchronization.
        \item Disconnection handling. Design a method to handle disconnections.
    \end{itemize}
	\textit{Measurement:} Two players can connect such that both players can affect the game state and both players are notified of the updates.

	\item[GS\refstepcounter{goalnum}\thegoalnum \label{G_customization}:] Customizable game settings. Core game elements must be modifiable to create custom Yahtzee variants. As a goal we would need these options to be implemented:
	\begin{itemize}
        \item Dice quantity option. Create an interface element for users to select from at least three options for number of dice.
        \item Dice type option. Create an interface element for users to select from at least three options for type of dice. These different types of dice will have different number of faces.
        \item Scoring method option. Offer at least two scoring systems that can be used.
        \item Timer feature. Implement a timer that can be turned on or off, providing a countdown for turn time.
    \end{itemize}
	\textit{Measurement:} The above options are implemented and are compatible with each another.
	
	\item[GS\refstepcounter{goalnum}\thegoalnum \label{G_presets}:] Preset game settings. By having some preset game configurations, it would allow players to more quickly learn the game or jump into an environment that has been tested.
	\begin{itemize}
        \item Create presets. Develop at least three preset configurations and test them for gameplay balance.
        \item Preset selection menu. Create a simple way to select a preset.
        \item Preset names and description. Give each available preset a name and a brief description to help players understand them.
    \end{itemize}
	\textit{Measurement:} At least three preset game configurations would be available for players to load up and play.
	
	\item[GS\refstepcounter{goalnum}\thegoalnum \label{G_3D}:] 3D dice rolling. Rolling the dice will need to be or look to be three dimensional to recreate the tactile feel of the original game.
	\begin{itemize}
        \item Dice 3D models. Develop 3D dice models, or the appearance of such.
        \item Dice interaction. Allow for user interaction with rolling dice through clicks or drags.
    \end{itemize}
	\textit{Measurement:} Dice will have the appearance of the preset die shape, and of being rolled, based on a minimum of 75\% of user feedback considering it so.

\end{itemize}

\vspace{50px}
Additional stretch goal statements are the following, and can additionally be considered for when looking to add to the system's complexity and to better fulfill the intended goal of being an enjoyable game.\\
The stretch goals are:

\begin{itemize}

	\item[SG\refstepcounter{goalnum}\thegoalnum \label{G_local_multiplayer}:] Local multiplayer. This would allow for players to play together on a single computer, but would require a different user interface and allow for different user interactions.
	\begin{itemize}
        \item Player input methods. Develop a system to handle inputs from multiple players on the same device.
        \item User interface adjustments. Design a UI (user interface) layout suitable for two players sharing a single screen.
    \end{itemize}
	\textit{Measurement:} The ability for two players to play together using a single interface and game instance without an internet connection.

	\item[SG\refstepcounter{goalnum}\thegoalnum \label{G_singleplayer}:] Singleplayer variants. A singleplayer game could be achieved either through a computer-run opponent in a game, or through a custom designed experience that could leverage the different environment.
	\begin{itemize}
        \item AI opponent development. Create an AI that can play against a human player.
        \item Difficulties. Design and allow players to adjust the AI opponent difficulty.
        \item Solo game modes. Design at least one unique solo challenge or mode that provides a self-contained experience.
    \end{itemize}
	\textit{Measurement:} A single person can play at least one variant made specifically to be singleplayer without requiring a second human player to update the game state.

	\item[SG\refstepcounter{goalnum}\thegoalnum \label{G_matchmaking}:] Online matchmaking. The game would provide users with the option to connect to another concurrent user based on a matchmaking score.
	\begin{itemize}
        \item Player rating system. Develop a rating system to categorize players by experience and skill.
        \item Matchmaking algorithm. Implement a system to match players.
    \end{itemize}
	\textit{Measurement:} A player can connect to another unknown concurrent player who was selected as a compatible opponent.

	\item[SG\refstepcounter{goalnum}\thegoalnum \label{G_saving}:] Saving custom game setting. Having this ability would allow for a user who created a custom game variant to save them for the ability to replay it without the need to recreate those specific settings.
	\begin{itemize}
        \item Save and load system. Build a system for players to save custom setting to a file or local storage.
        \item Edit and delete options. Allow players to edit or delete saved custom settings for better management.
    \end{itemize}
	\textit{Measurement:} A custom game variant, as per the "Customizable game settings" goal and "More game setting customization" stretch goals, can be saved locally and loaded up to be played.
	
	\item[SG\refstepcounter{goalnum}\thegoalnum \label{G_customization2}:] More game setting customization. Besides the options in the goals section, some additional game customization options would include:
	\begin{itemize}
        \item Scorecard customization. Provide methods for players to adjust what options and hands appear on the scorecards.
        \item Scoring points options. Offer players the ability to modify the scores of scoring options.
        \item Additional scoring mechanisms. Include options for different methods of round or game scoring.
        \item Gambling mechanism. Add additional ways to act on probabilities such as wagering on specific rolls.
        \item Feedback-based features. Gather ideas through user feedback and testing to further expand available customization options.
    \end{itemize}
	\textit{Measurement:} Additional game options outside the ones listed in the "Customizable game settings" goal would be available.

	\item[SG\refstepcounter{goalnum}\thegoalnum \label{G_skins}:] Dice customization. Dice could be made to appear differently, either as a means for personalization or for aiding with different impairments. An example could be a dice with pips versus a dice with a numbered faces.
	\begin{itemize}
        \item Dice colour options. Create at least three different dice appearances.
        \item Dice number representation. Create different ways to represent dice face values such as traditional pips, numbers, and symbols.
        \item Personalization menu. Implement a menu to allow for easy selection of dice appearance.
    \end{itemize}
	\textit{Measurement:} At least five different dice appearance variants players can choose from, that would appear in the game.
	
	\item[SG\refstepcounter{goalnum}\thegoalnum \label{G_stats}:] Post game statistics. This could allow for players to analyze a game after completion in a more quantitative manner, aiding in better understanding statistical probabilities.
	\begin{itemize}
        \item Key statistic tracking. Track important game stats during play.
        \item Post-game summary screen. Present collected stats on a summary screen after a game.
    \end{itemize}
	\textit{Measurement:} A post-game summary showing at least three key game stats, available after each game.
	
	\item[SG\refstepcounter{goalnum}\thegoalnum \label{G_platforms}:] Multi-platform support. While most gaming experiences are for windows, this would allow for the game to be run on more than just the Windows operating system, allowing for a wider audience.
	\begin{itemize}
        \item Compile for systems. Compile the created system for other operating systems.
        \item Platform testing. Test the game on multiple operating systems.
        \item Cross-platform functionality. Verify the features such as online multiplayer work across different operating systems.
    \end{itemize}
	\textit{Measurement:} The game can be run on operating systems other than Windows.
	
	\item[SG\refstepcounter{goalnum}\thegoalnum \label{G_highlights}:] Dice highlighting. This would aid in determining what dice are used when scoring.
	\begin{itemize}
        \item Automatic dice highlighting. Implement a system to automatically highlight the dice that contribute to a player's score.
        \item Optional setting. Allow players to enable or disable the feature.
    \end{itemize}
	\textit{Measurement:} Dice used in scoring will be highlighted when appropriate.

\end{itemize}


\subsection{Solution Characteristics Specification}

This section, along with Physical System Description are not included within our document and have been removed from the template. The purpose of this section is to reduce the problem into one expressed in mathematical terms. Mathematical expertise is used to extract the essentials from the underlying physical description of the problem, and to collect and substantiate all physical data pertinent to the problem. Important elements that might otherwise have been in this section can be found in other sections where they may be more pertinent. Given the focus of this section on the physical description, and the fact there is no physical description as we have a software based project that is mostly hardware agnostic where the operating system would be more pertinent to the execution of the program. 

\iffalse
\plt{This section specifies the information in the solution domain of the system
  to be developed. This section is intended to express what is required in
  such a way that analysts and stakeholders get a clear picture, and the
  latter will accept it. The purpose of this section is to reduce the problem
  into one expressed in mathematical terms. Mathematical expertise is used to
  extract the essentials from the underlying physical description of the
  problem, and to collect and substantiate all physical data pertinent to the
  problem.}

\plt{This section presents the solution characteristics by successively refining
  models.  It starts with the abstract/general Theoretical Models (TMs) and
  refines them to the concrete/specific Instance Models (IMs).  If necessary
  there are intermediate refinements to General Definitions (GDs).  All of these
  refinements can potentially use Assumptions (A) and Data Definitions (DD).
  TMs are refined to create new models, that are called GMs or IMs. DDs are not
  refined; they are just used. GDs and IMs are derived, or refined, from other
  models. DDs are not derived; they are just given. TMs are also just given, but
  they are refined, not used.  If a potential DD includes a derivation, then
  that means it is refining other models, which would make it a GD or an IM.}

\plt{The above makes a distinction between ``refined'' and ``used.'' A model is
  refined to another model if it is changed by the refinement. When we change a
  general 3D equation to a 2D equation, we are making a refinement, by applying
  the assumption that the third dimension does not matter. If we use a
  definition, like the definition of density, we aren't refining, or changing
  that definition, we are just using it.}

\plt{The same information can be a TM in one problem and a DD in another.  It is
  about how the information is used.  In one problem the definition of
  acceleration can be a TM, in another it would be a DD.}

\plt{There is repetition between the information given in the different chunks
  (TM, GDs etc) with other information in the document.  For instance, the
  meaning of the symbols, the units etc are repeated.  This is so that the
  chunks can stand on their own when being read by a reviewer/user.  It also
  facilitates reuse of the models in a different context.}

\noindent \plt{The relationships between the parts of the document are show in
  the following figure.  In this diagram ``may ref'' has the same role as
  ``uses'' above.  The figure adds ``Likely Changes,'' which are able to
  reference (use) Assumptions.}

\begin{figure}[H]
  \includegraphics[scale=0.9]{figures/RelationsBetweenTM_GD_IM_DD_A.pdf}
\end{figure}

The instance models that govern \progname{} are presented in
Subsection~\ref{sec_instance}.  The information to understand the meaning of the
instance models and their derivation is also presented, so that the instance
models can be verified.

\subsubsection{Types}

\plt{This section is optional. Defining types can make the document easier to
understand.}

\subsubsection{Scope Decisions}

\plt{This section is optional.}
\subsubsection{Modelling Decisions}

\plt{This section is optional.}

\subsubsection{Assumptions} \label{sec_assumpt}

\plt{The assumptions are a refinement of the scope.  The scope is general, where
  the assumptions are specific.  All assumptions should be listed, even those
  that domain experts know so well that they are rarely (if ever) written down.}
\plt{The document should not take for granted that the reader knows which
  assumptions have been made. In the case of unusual assumptions, it is
  recommended that the documentation either include, or point to, an explanation
  and justification for the assumption.}
\plt{If it helps with the organization and understandability, the assumptions can be presented as sub sections.  The following sub-sections are options: background theory assumptions, helper theory assumptions, generic theory assumptions, problem specific assumptions, and rationale assumptions}

This section simplifies the original problem and helps in developing the
theoretical model by filling in the missing information for the physical system.
The numbers given in the square brackets refer to the theoretical model [TM],
general definition [GD], data definition [DD], instance model [IM], or likely
change [LC], in which the respective assumption is used.

\begin{itemize}

\item[A\refstepcounter{assumpnum}\theassumpnum \label{A_meaningfulLabel}:]
  \plt{Short description of each assumption.  Each assumption
    should have a meaningful label.  Use cross-references to identify the
    appropriate traceability to TM, GD, DD etc., using commands like dref, ddref
    etc.  Each assumption should be atomic - that is, there should not be an
    explicit (or implicit) ``and'' in the text of an assumption.}

\end{itemize}

\subsubsection{Theoretical Models}\label{sec_theoretical}

\plt{Theoretical models are sets of abstract mathematical equations or axioms
  for solving the problem described in Section ``Physical System Description''
  (Section~\ref{sec_phySystDescrip}). Examples of theoretical models are
  physical laws, constitutive equations, relevant conversion factors, etc.}

\plt{Optionally the theory section could be divided into subsections to provide more structure and improve understandability and reusability.  Potential subsections include the following: Context theories, background theories, helper theories, generic theories, problem specific theories, final theories and rationale theories.}

This section focuses on the general equations and laws that \progname{} is based
on.  \plt{Modify the examples below for your problem, and add additional models
  as appropriate.}

~\newline

\noindent
\deftheory
% #2 refname of theory
{TM:COE}
% #3 label
{Conservation of thermal energy}
% #4 equation
{
  $-{\bf \nabla \cdot q} + g$ = $\rho C \frac{\partial T}{\partial t}$
}
% #5 description
{
  The above equation gives the conservation of energy for transient heat transfer in a material
  of specific heat capacity $C$ (\si{\joule\per\kilogram\per\celsius}) and density $\rho$ 
  (\si{\kilogram\per\cubic\metre}), where $\bf q$ is the thermal flux vector (\si{\watt\per\square\metre}),
  $g$ is the volumetric heat generation
  (\si{\watt\per\cubic\metre}), $T$ is the temperature
  (\si{\celsius}),  $t$ is time (\si{\second}), and $\nabla$ is
  the gradient operator.  For this equation to apply, other forms
  of energy, such as mechanical energy, are assumed to be negligible in the
  system (\aref{A_OnlyThermalEnergy}).  In general, the material properties ($\rho$ and $C$) depend on temperature.
}
% #6 Notes
{
None.
}
% #7 Source
{
  \url{http://www.efunda.com/formulae/heat_transfer/conduction/overview_cond.cfm}
}
% #8 Referenced by
{
  \dref{ROCT}
}
% #9 Preconditions
{
None
}
% #1 derivation - not applicable by default
{}

\plt{``Ref.\ By'' is used repeatedly with the different types of information.
  This stands for Referenced By.  It means that the models, definitions and
  assumptions listed reference the current model, definition or assumption.
  This information is given for traceability.  Ref. By provides a pointer in the
  opposite direction to what we commonly do.  You still need to have a reference
  in the other direction pointing to the current model, definition or
  assumption.  As an example, if TM1 is referenced by GD2, that means that GD2 will
  explicitly include a reference to TM1.}

~\newline

\subsubsection{General Definitions}\label{sec_gendef}

\plt{General Definitions (GDs) are a refinement of one or more TMs, and/or of
  other GDs.  The GDs are less abstract than the TMs.  Generally the reduction
  in abstraction is possible through invoking (using/referencing) Assumptions.
  For instance, the TM could be Newton's Law of Cooling stated abstracting.  The
  GD could take the general law and apply it to get a 1D equation.}

This section collects the laws and equations that will be used in building the
instance models.

\plt{Some projects may not have any content for this section, but the section
  heading should be kept.}  \plt{Modify the examples below for your problem, and
  add additional definitions as appropriate.}

~\newline

\noindent
\begin{minipage}{\textwidth}
\renewcommand*{\arraystretch}{1.5}
\begin{tabular}{| p{\colAwidth} | p{\colBwidth}|}
\hline
\rowcolor[gray]{0.9}
Number& GD\refstepcounter{defnum}\thedefnum \label{NL}\\
\hline
Label &\bf Newton's law of cooling \\
\hline
% Units&$MLt^{-3}T^0$\\
% \hline
SI Units&\si{\watt\per\square\metre}\\
\hline
Equation&$ q(t) = h \Delta T(t)$  \\
\hline
Description &
Newton's law of cooling describes convective cooling from a surface.  The law is
stated as: the rate of heat loss from a body is proportional to the difference
in temperatures between the body and its surroundings.
\\
& $q(t)$ is the thermal flux (\si{\watt\per\square\metre}).\\
& $h$ is the heat transfer coefficient, assumed independent of $T$ (\aref{A_hcoeff})
	(\si{\watt\per\square\metre\per\celsius}).\\
&$\Delta T(t)$= $T(t) - T_{\text{env}}(t)$ is the time-dependent thermal gradient
between the environment and the object (\si{\celsius}).
\\
\hline
  Source & Citation here \\
  \hline
  Ref.\ By & \ddref{FluxCoil}, \ddref{FluxPCM}\\
  \hline
\end{tabular}
\end{minipage}\\

\subsubsection*{Detailed derivation of simplified rate of change of temperature}

\plt{This may be necessary when the necessary information does not fit in the
  description field.}
\plt{Derivations are important for justifying a given GD.  You want it to be
  clear where the equation came from.}

\subsubsection{Data Definitions}\label{sec_datadef}

\plt{The Data Definitions are definitions of symbols and equations that are
  given for the problem.  They are not derived; they are simply used by other
  models.  For instance, if a problem depends on density, there may be a data
  definition for the equation defining density.  The DDs are given information
  that you can use in your other modules.}

\plt{All Data Definitions should be used (referenced) by at least one other
  model.}

This section collects and defines all the data needed to build the instance
models. The dimension of each quantity is also given.  \plt{Modify the examples
  below for your problem, and add additional definitions as appropriate.}

~\newline

\noindent
\begin{minipage}{\textwidth}
\renewcommand*{\arraystretch}{1.5}
\begin{tabular}{| p{\colAwidth} | p{\colBwidth}|}
\hline
\rowcolor[gray]{0.9}
Number& DD\refstepcounter{datadefnum}\thedatadefnum \label{FluxCoil}\\
\hline
Label& \bf Heat flux out of coil\\
\hline
Symbol &$q_C$\\
\hline
% Units& $Mt^{-3}$\\
% \hline
  SI Units & \si{\watt\per\square\metre}\\
  \hline
  Equation&$q_C(t) = h_C (T_C - T_W(t))$, over area $A_C$\\
  \hline
  Description & 
                $T_C$ is the temperature of the coil (\si{\celsius}).  $T_W$ is the temperature of the water (\si{\celsius}).  
                The heat flux out of the coil, $q_C$ (\si{\watt\per\square\metre}), is found by
                assuming that Newton's Law 
                of Cooling applies (\aref{A_Newt_coil}).  This law (\dref{NL}) is used on the surface of
                the coil, which has area $A_C$ (\si{\square\metre}) and heat 
                transfer coefficient $h_C$
                (\si{\watt\per\square\metre\per\celsius}).  This equation
                assumes that the temperature of the coil is constant over time (\aref{A_tcoil}) and that it does not vary along the length
                of the coil (\aref{A_tlcoil}).
  \\
  \hline
  Sources& Citation here \\
  \hline
  Ref.\ By & \iref{ewat}\\
  \hline
\end{tabular}
\end{minipage}\\

\subsubsection{Data Types}\label{sec_datatypes}

\plt{This section is optional.  In many scientific computing programs it isn't
  necessary, since the inputs and outpus are straightforward types, like reals,
  integers, and sequences of reals and integers.  However, for some problems it
  is very helpful to capture the type information.}

\plt{The data types are not derived; they are simply stated and used by other
  models.}

\plt{All data types must be used by at least one of the models.}

\plt{For the mathematical notation for expressing types, the recommendation is
  to use the notation of~\citet{HoffmanAndStrooper1995}.}

This section collects and defines all the data types needed to document the
models. \plt{Modify the examples below for your problem, and add additional
  definitions as appropriate.}

~\newline

\noindent
\begin{minipage}{\textwidth}
\renewcommand*{\arraystretch}{1.5}
\begin{tabular}{| p{\colAwidth} | p{\colBwidth}|}
  \hline
  \rowcolor[gray]{0.9}
  Type Name & Name for Type\\
  \hline
  Type Def & mathematical definition of the type\\
  \hline
  Description & description here
  \\
  \hline
  Sources & Citation here, if the type is borrowed from another source\\
  \hline
\end{tabular}
\end{minipage}\\

\subsubsection{Instance Models} \label{sec_instance}    

\plt{The motivation for this section is to reduce the problem defined in
  ``Physical System Description'' (Section~\ref{sec_phySystDescrip}) to one
  expressed in mathematical terms. The IMs are built by refining the TMs and/or
  GDs.  This section should remain abstract.  The SRS should specify the
  requirements without considering the implementation.}

This section transforms the problem defined in Section~\ref{Sec_pd} into 
one which is expressed in mathematical terms. It uses concrete symbols defined 
in Section~\ref{sec_datadef} to replace the abstract symbols in the models 
identified in Sections~\ref{sec_theoretical} and~\ref{sec_gendef}.

The goals \plt{reference your goals} are solved by \plt{reference your instance
  models}.  \plt{other details, with cross-references where appropriate.}
\plt{Modify the examples below for your problem, and add additional models as
  appropriate.}

~\newline

%Instance Model 1

\noindent
\begin{minipage}{\textwidth}
\renewcommand*{\arraystretch}{1.5}
\begin{tabular}{| p{\colAwidth} | p{\colBwidth}|}
  \hline
  \rowcolor[gray]{0.9}
  Number& IM\refstepcounter{instnum}\theinstnum \label{ewat}\\
  \hline
  Label& \bf Energy balance on water to find $T_W$\\
  \hline
  Input&$m_W$, $C_W$, $h_C$, $A_C$, $h_P$, $A_P$, $t_\text{final}$, $T_C$, 
  $T_\text{init}$, $T_P(t)$ from \iref{epcm}\\
  & The input is constrained so that $T_\text{init} \leq T_C$ (\aref{A_charge})\\
  \hline
  Output&$T_W(t)$, $0\leq t \leq t_\text{final}$, such that\\
  &$\frac{dT_W}{dt} = \frac{1}{\tau_W}[(T_C - T_W(t)) + {\eta}(T_P(t) - T_W(t))]$,\\
  &$T_W(0) = T_P(0) = T_\text{init}$ (\aref{A_InitTemp}) and $T_P(t)$ from \iref{epcm} \\
  \hline
  Description&$T_W$ is the water temperature (\si{\celsius}).\\
  &$T_P$ is the PCM temperature (\si{\celsius}).\\
  &$T_C$ is the coil temperature (\si{\celsius}).\\
  &$\tau_W = \frac{m_W C_W}{h_C A_C}$ is a constant (\si{\second}).\\
  &$\eta = \frac{h_P A_P}{h_C A_C}$ is a constant (dimensionless).\\
  & The above equation applies as long as the water is in liquid form,
  $0<T_W<100^o\text{C}$, where $0^o\text{C}$ and $100^o\text{C}$ are the melting
  and boiling points of water, respectively (\aref{A_OpRange}, \aref{A_Pressure}).
  \\
  \hline
  Sources& Citation here \\
  \hline
  Ref.\ By & \iref{epcm}\\
  \hline
\end{tabular}
\end{minipage}\\

%~\newline

\subsubsection*{Derivation of ...}

\plt{The derivation shows how the IM is derived from the TMs/GDs.  In cases
  where the derivation cannot be described under the Description field, it will
  be necessary to include this subsection.}

\subsubsection{Input Data Constraints} \label{sec_DataConstraints}    

Table~\ref{TblInputVar} shows the data constraints on the input output
variables.  The column for physical constraints gives the physical limitations
on the range of values that can be taken by the variable.  The column for
software constraints restricts the range of inputs to reasonable values.  The
software constraints will be helpful in the design stage for picking suitable
algorithms.  The constraints are conservative, to give the user of the model the
flexibility to experiment with unusual situations.  The column of typical values
is intended to provide a feel for a common scenario.  The uncertainty column
provides an estimate of the confidence with which the physical quantities can be
measured.  This information would be part of the input if one were performing an
uncertainty quantification exercise.

The specification parameters in Table~\ref{TblInputVar} are listed in
Table~\ref{TblSpecParams}.

\begin{table}[!h]
  \caption{Input Variables} \label{TblInputVar}
  \renewcommand{\arraystretch}{1.2}
\noindent \begin{longtable*}{l l l l c} 
  \toprule
  \textbf{Var} & \textbf{Physical Constraints} & \textbf{Software Constraints} &
                             \textbf{Typical Value} & \textbf{Uncertainty}\\
  \midrule 
  $L$ & $L > 0$ & $L_{\text{min}} \leq L \leq L_{\text{max}}$ & 1.5 \si[per-mode=symbol] {\metre} & 10\%
  \\
  \bottomrule
\end{longtable*}
\end{table}

\noindent 
\begin{description}
\item[(*)] \plt{you might need to add some notes or clarifications}
\end{description}

\begin{table}[!h]
\caption{Specification Parameter Values} \label{TblSpecParams}
\renewcommand{\arraystretch}{1.2}
\noindent \begin{longtable*}{l l} 
  \toprule
  \textbf{Var} & \textbf{Value} \\
  \midrule 
  $L_\text{min}$ & 0.1 \si{\metre}\\
  \bottomrule
\end{longtable*}
\end{table}

\subsubsection{Properties of a Correct Solution} \label{sec_CorrectSolution}

\noindent
A correct solution must exhibit \plt{fill in the details}.  \plt{These
  properties are in addition to the stated requirements.  There is no need to
  repeat the requirements here.  These additional properties may not exist for
  every problem.  Examples include conservation laws (like conservation of
  energy or mass) and known constraints on outputs, which are usually summarized
  in tabular form.  A sample table is shown in Table~\ref{TblOutputVar}}

\begin{table}[!h]
\caption{Output Variables} \label{TblOutputVar}
\renewcommand{\arraystretch}{1.2}
\noindent \begin{longtable*}{l l} 
  \toprule
  \textbf{Var} & \textbf{Physical Constraints} \\
  \midrule 
  $T_W$ & $T_\text{init} \leq T_W \leq T_C$ (by~\aref{A_charge})
  \\
  \bottomrule
\end{longtable*}
\end{table}

\plt{This section is not for test cases or techniques for verification and
  validation.  Those topics will be addressed in the Verification and Validation
  plan.}

\fi