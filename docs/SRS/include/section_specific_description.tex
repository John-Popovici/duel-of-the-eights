\section{Specific System Description}

This section first presents the problem description, which gives a high-level
view of the problem to be solved.  This is followed by the solution characteristics
specification, which presents the assumptions, theories, definitions and finally
the instance models.  \plt{Add any project specific details that are relevant
  for the section overview.}

\subsection{Problem Description} \label{Sec_pd}

\progname{} is intended to solve ... \plt{What problem does your program solve?
The description here should be in the problem space, not the solution space.}

\subsubsection{Terminology and  Definitions}

\plt{This section is expressed in words, not with equations.  It provide the
  meaning of the different words and phrases used in the domain of the problem.
The terminology is used to introduce concepts from the world outside of the
mathematical model  The terminology provides a real world connection to give the
mathematical model meaning.}

This subsection provides a list of terms that are used in the subsequent
sections and their meaning, with the purpose of reducing ambiguity and making it
easier to correctly understand the requirements:

\begin{itemize}

\item 

\end{itemize}

\subsubsection{Physical System Description} \label{sec_phySystDescrip}

\plt{The purpose of this section is to clearly and unambiguously state the
  physical system that is to be modelled. Effective problem solving requires a
  logical and organized approach. The statements on the physical system to be
  studied should cover enough information to solve the problem. The physical
  description involves element identification, where elements are defined as
  independent and separable items of the physical system. Some example elements
  include acceleration due to gravity, the mass of an object, and the size and
  shape of an object. Each element should be identified and labelled, with their
  interesting properties specified clearly. The physical description can also
  include interactions of the elements, such as the following: i) the
  interactions between the elements and their physical environment; ii) the
  interactions between elements; and, iii) the initial or boundary conditions.}

\plt{The elements of the physical system do not have to correspond to an actual
physical entity.  They can be conceptual.  This is particularly important when
the documentation is for a numerical method. }

The physical system of \progname{}, as shown in Figure~?,
includes the following elements:

\begin{itemize}

\item[PS1:] 

\item[PS2:] ...

\end{itemize}

\plt{A figure here makes sense for most SRS documents}

% \begin{figure}[h!]
% \begin{center}
% %\rotatebox{-90}
% {
%  \includegraphics[width=0.5\textwidth]{<FigureName>}
% }
% \caption{\label{<Label>} <Caption>}
% \end{center}
% \end{figure}

\subsubsection{Goal Statements}

\plt{The goal statements refine the ``Problem Description''
  (Section~\ref{Sec_pd}).  A goal is a functional objective the system under
  consideration should achieve. Goals provide criteria for sufficient
  completeness of a requirements specification and for requirements
  pertinence. Goals will be refined in Section “Instanced Models”
  (Section~\ref{sec_instance}). Large and complex goals should be decomposed
  into smaller sub-goals.  The goals are written abstractly, with a minimal
  amount of technical language.  They should be understandable by non-domain
  experts.}

\noindent Given the \plt{inputs}, the goal statements are:

\begin{itemize}

\item[GS\refstepcounter{goalnum}\thegoalnum \label{G_meaningfulLabel}:] \plt{One
    sentence description of the goal.  There may be more than one.  Each Goal
    should have a meaningful label.}

\end{itemize}

\subsection{Solution Characteristics Specification}

\plt{This section specifies the information in the solution domain of the system
  to be developed. This section is intended to express what is required in
  such a way that analysts and stakeholders get a clear picture, and the
  latter will accept it. The purpose of this section is to reduce the problem
  into one expressed in mathematical terms. Mathematical expertise is used to
  extract the essentials from the underlying physical description of the
  problem, and to collect and substantiate all physical data pertinent to the
  problem.}

\plt{This section presents the solution characteristics by successively refining
  models.  It starts with the abstract/general Theoretical Models (TMs) and
  refines them to the concrete/specific Instance Models (IMs).  If necessary
  there are intermediate refinements to General Definitions (GDs).  All of these
  refinements can potentially use Assumptions (A) and Data Definitions (DD).
  TMs are refined to create new models, that are called GMs or IMs. DDs are not
  refined; they are just used. GDs and IMs are derived, or refined, from other
  models. DDs are not derived; they are just given. TMs are also just given, but
  they are refined, not used.  If a potential DD includes a derivation, then
  that means it is refining other models, which would make it a GD or an IM.}

\plt{The above makes a distinction between ``refined'' and ``used.'' A model is
  refined to another model if it is changed by the refinement. When we change a
  general 3D equation to a 2D equation, we are making a refinement, by applying
  the assumption that the third dimension does not matter. If we use a
  definition, like the definition of density, we aren't refining, or changing
  that definition, we are just using it.}

\plt{The same information can be a TM in one problem and a DD in another.  It is
  about how the information is used.  In one problem the definition of
  acceleration can be a TM, in another it would be a DD.}

\plt{There is repetition between the information given in the different chunks
  (TM, GDs etc) with other information in the document.  For instance, the
  meaning of the symbols, the units etc are repeated.  This is so that the
  chunks can stand on their own when being read by a reviewer/user.  It also
  facilitates reuse of the models in a different context.}

\noindent \plt{The relationships between the parts of the document are show in
  the following figure.  In this diagram ``may ref'' has the same role as
  ``uses'' above.  The figure adds ``Likely Changes,'' which are able to
  reference (use) Assumptions.}

\begin{figure}[H]
  \includegraphics[scale=0.9]{figures/RelationsBetweenTM_GD_IM_DD_A.pdf}
\end{figure}

The instance models that govern \progname{} are presented in
Subsection~\ref{sec_instance}.  The information to understand the meaning of the
instance models and their derivation is also presented, so that the instance
models can be verified.

\subsubsection{Types}

\plt{This section is optional. Defining types can make the document easier to
understand.}

\subsubsection{Scope Decisions}

\plt{This section is optional.}
\subsubsection{Modelling Decisions}

\plt{This section is optional.}

\subsubsection{Assumptions} \label{sec_assumpt}

\plt{The assumptions are a refinement of the scope.  The scope is general, where
  the assumptions are specific.  All assumptions should be listed, even those
  that domain experts know so well that they are rarely (if ever) written down.}
\plt{The document should not take for granted that the reader knows which
  assumptions have been made. In the case of unusual assumptions, it is
  recommended that the documentation either include, or point to, an explanation
  and justification for the assumption.}
\plt{If it helps with the organization and understandability, the assumptions can be presented as sub sections.  The following sub-sections are options: background theory assumptions, helper theory assumptions, generic theory assumptions, problem specific assumptions, and rationale assumptions}

This section simplifies the original problem and helps in developing the
theoretical model by filling in the missing information for the physical system.
The numbers given in the square brackets refer to the theoretical model [TM],
general definition [GD], data definition [DD], instance model [IM], or likely
change [LC], in which the respective assumption is used.

\begin{itemize}

\item[A\refstepcounter{assumpnum}\theassumpnum \label{A_meaningfulLabel}:]
  \plt{Short description of each assumption.  Each assumption
    should have a meaningful label.  Use cross-references to identify the
    appropriate traceability to TM, GD, DD etc., using commands like dref, ddref
    etc.  Each assumption should be atomic - that is, there should not be an
    explicit (or implicit) ``and'' in the text of an assumption.}

\end{itemize}

\subsubsection{Theoretical Models}\label{sec_theoretical}

\plt{Theoretical models are sets of abstract mathematical equations or axioms
  for solving the problem described in Section ``Physical System Description''
  (Section~\ref{sec_phySystDescrip}). Examples of theoretical models are
  physical laws, constitutive equations, relevant conversion factors, etc.}

\plt{Optionally the theory section could be divided into subsections to provide more structure and improve understandability and reusability.  Potential subsections include the following: Context theories, background theories, helper theories, generic theories, problem specific theories, final theories and rationale theories.}

This section focuses on the general equations and laws that \progname{} is based
on.  \plt{Modify the examples below for your problem, and add additional models
  as appropriate.}

~\newline

\noindent
\deftheory
% #2 refname of theory
{TM:COE}
% #3 label
{Conservation of thermal energy}
% #4 equation
{
  $-{\bf \nabla \cdot q} + g$ = $\rho C \frac{\partial T}{\partial t}$
}
% #5 description
{
  The above equation gives the conservation of energy for transient heat transfer in a material
  of specific heat capacity $C$ (\si{\joule\per\kilogram\per\celsius}) and density $\rho$ 
  (\si{\kilogram\per\cubic\metre}), where $\bf q$ is the thermal flux vector (\si{\watt\per\square\metre}),
  $g$ is the volumetric heat generation
  (\si{\watt\per\cubic\metre}), $T$ is the temperature
  (\si{\celsius}),  $t$ is time (\si{\second}), and $\nabla$ is
  the gradient operator.  For this equation to apply, other forms
  of energy, such as mechanical energy, are assumed to be negligible in the
  system (\aref{A_OnlyThermalEnergy}).  In general, the material properties ($\rho$ and $C$) depend on temperature.
}
% #6 Notes
{
None.
}
% #7 Source
{
  \url{http://www.efunda.com/formulae/heat_transfer/conduction/overview_cond.cfm}
}
% #8 Referenced by
{
  \dref{ROCT}
}
% #9 Preconditions
{
None
}
% #1 derivation - not applicable by default
{}

\plt{``Ref.\ By'' is used repeatedly with the different types of information.
  This stands for Referenced By.  It means that the models, definitions and
  assumptions listed reference the current model, definition or assumption.
  This information is given for traceability.  Ref. By provides a pointer in the
  opposite direction to what we commonly do.  You still need to have a reference
  in the other direction pointing to the current model, definition or
  assumption.  As an example, if TM1 is referenced by GD2, that means that GD2 will
  explicitly include a reference to TM1.}

~\newline

\subsubsection{General Definitions}\label{sec_gendef}

\plt{General Definitions (GDs) are a refinement of one or more TMs, and/or of
  other GDs.  The GDs are less abstract than the TMs.  Generally the reduction
  in abstraction is possible through invoking (using/referencing) Assumptions.
  For instance, the TM could be Newton's Law of Cooling stated abstracting.  The
  GD could take the general law and apply it to get a 1D equation.}

This section collects the laws and equations that will be used in building the
instance models.

\plt{Some projects may not have any content for this section, but the section
  heading should be kept.}  \plt{Modify the examples below for your problem, and
  add additional definitions as appropriate.}

~\newline

\noindent
\begin{minipage}{\textwidth}
\renewcommand*{\arraystretch}{1.5}
\begin{tabular}{| p{\colAwidth} | p{\colBwidth}|}
\hline
\rowcolor[gray]{0.9}
Number& GD\refstepcounter{defnum}\thedefnum \label{NL}\\
\hline
Label &\bf Newton's law of cooling \\
\hline
% Units&$MLt^{-3}T^0$\\
% \hline
SI Units&\si{\watt\per\square\metre}\\
\hline
Equation&$ q(t) = h \Delta T(t)$  \\
\hline
Description &
Newton's law of cooling describes convective cooling from a surface.  The law is
stated as: the rate of heat loss from a body is proportional to the difference
in temperatures between the body and its surroundings.
\\
& $q(t)$ is the thermal flux (\si{\watt\per\square\metre}).\\
& $h$ is the heat transfer coefficient, assumed independent of $T$ (\aref{A_hcoeff})
	(\si{\watt\per\square\metre\per\celsius}).\\
&$\Delta T(t)$= $T(t) - T_{\text{env}}(t)$ is the time-dependent thermal gradient
between the environment and the object (\si{\celsius}).
\\
\hline
  Source & Citation here \\
  \hline
  Ref.\ By & \ddref{FluxCoil}, \ddref{FluxPCM}\\
  \hline
\end{tabular}
\end{minipage}\\

\subsubsection*{Detailed derivation of simplified rate of change of temperature}

\plt{This may be necessary when the necessary information does not fit in the
  description field.}
\plt{Derivations are important for justifying a given GD.  You want it to be
  clear where the equation came from.}

\subsubsection{Data Definitions}\label{sec_datadef}

\plt{The Data Definitions are definitions of symbols and equations that are
  given for the problem.  They are not derived; they are simply used by other
  models.  For instance, if a problem depends on density, there may be a data
  definition for the equation defining density.  The DDs are given information
  that you can use in your other modules.}

\plt{All Data Definitions should be used (referenced) by at least one other
  model.}

This section collects and defines all the data needed to build the instance
models. The dimension of each quantity is also given.  \plt{Modify the examples
  below for your problem, and add additional definitions as appropriate.}

~\newline

\noindent
\begin{minipage}{\textwidth}
\renewcommand*{\arraystretch}{1.5}
\begin{tabular}{| p{\colAwidth} | p{\colBwidth}|}
\hline
\rowcolor[gray]{0.9}
Number& DD\refstepcounter{datadefnum}\thedatadefnum \label{FluxCoil}\\
\hline
Label& \bf Heat flux out of coil\\
\hline
Symbol &$q_C$\\
\hline
% Units& $Mt^{-3}$\\
% \hline
  SI Units & \si{\watt\per\square\metre}\\
  \hline
  Equation&$q_C(t) = h_C (T_C - T_W(t))$, over area $A_C$\\
  \hline
  Description & 
                $T_C$ is the temperature of the coil (\si{\celsius}).  $T_W$ is the temperature of the water (\si{\celsius}).  
                The heat flux out of the coil, $q_C$ (\si{\watt\per\square\metre}), is found by
                assuming that Newton's Law 
                of Cooling applies (\aref{A_Newt_coil}).  This law (\dref{NL}) is used on the surface of
                the coil, which has area $A_C$ (\si{\square\metre}) and heat 
                transfer coefficient $h_C$
                (\si{\watt\per\square\metre\per\celsius}).  This equation
                assumes that the temperature of the coil is constant over time (\aref{A_tcoil}) and that it does not vary along the length
                of the coil (\aref{A_tlcoil}).
  \\
  \hline
  Sources& Citation here \\
  \hline
  Ref.\ By & \iref{ewat}\\
  \hline
\end{tabular}
\end{minipage}\\

\subsubsection{Data Types}\label{sec_datatypes}

\plt{This section is optional.  In many scientific computing programs it isn't
  necessary, since the inputs and outpus are straightforward types, like reals,
  integers, and sequences of reals and integers.  However, for some problems it
  is very helpful to capture the type information.}

\plt{The data types are not derived; they are simply stated and used by other
  models.}

\plt{All data types must be used by at least one of the models.}

\plt{For the mathematical notation for expressing types, the recommendation is
  to use the notation of~\citet{HoffmanAndStrooper1995}.}

This section collects and defines all the data types needed to document the
models. \plt{Modify the examples below for your problem, and add additional
  definitions as appropriate.}

~\newline

\noindent
\begin{minipage}{\textwidth}
\renewcommand*{\arraystretch}{1.5}
\begin{tabular}{| p{\colAwidth} | p{\colBwidth}|}
  \hline
  \rowcolor[gray]{0.9}
  Type Name & Name for Type\\
  \hline
  Type Def & mathematical definition of the type\\
  \hline
  Description & description here
  \\
  \hline
  Sources & Citation here, if the type is borrowed from another source\\
  \hline
\end{tabular}
\end{minipage}\\

\subsubsection{Instance Models} \label{sec_instance}    

\plt{The motivation for this section is to reduce the problem defined in
  ``Physical System Description'' (Section~\ref{sec_phySystDescrip}) to one
  expressed in mathematical terms. The IMs are built by refining the TMs and/or
  GDs.  This section should remain abstract.  The SRS should specify the
  requirements without considering the implementation.}

This section transforms the problem defined in Section~\ref{Sec_pd} into 
one which is expressed in mathematical terms. It uses concrete symbols defined 
in Section~\ref{sec_datadef} to replace the abstract symbols in the models 
identified in Sections~\ref{sec_theoretical} and~\ref{sec_gendef}.

The goals \plt{reference your goals} are solved by \plt{reference your instance
  models}.  \plt{other details, with cross-references where appropriate.}
\plt{Modify the examples below for your problem, and add additional models as
  appropriate.}

~\newline

%Instance Model 1

\noindent
\begin{minipage}{\textwidth}
\renewcommand*{\arraystretch}{1.5}
\begin{tabular}{| p{\colAwidth} | p{\colBwidth}|}
  \hline
  \rowcolor[gray]{0.9}
  Number& IM\refstepcounter{instnum}\theinstnum \label{ewat}\\
  \hline
  Label& \bf Energy balance on water to find $T_W$\\
  \hline
  Input&$m_W$, $C_W$, $h_C$, $A_C$, $h_P$, $A_P$, $t_\text{final}$, $T_C$, 
  $T_\text{init}$, $T_P(t)$ from \iref{epcm}\\
  & The input is constrained so that $T_\text{init} \leq T_C$ (\aref{A_charge})\\
  \hline
  Output&$T_W(t)$, $0\leq t \leq t_\text{final}$, such that\\
  &$\frac{dT_W}{dt} = \frac{1}{\tau_W}[(T_C - T_W(t)) + {\eta}(T_P(t) - T_W(t))]$,\\
  &$T_W(0) = T_P(0) = T_\text{init}$ (\aref{A_InitTemp}) and $T_P(t)$ from \iref{epcm} \\
  \hline
  Description&$T_W$ is the water temperature (\si{\celsius}).\\
  &$T_P$ is the PCM temperature (\si{\celsius}).\\
  &$T_C$ is the coil temperature (\si{\celsius}).\\
  &$\tau_W = \frac{m_W C_W}{h_C A_C}$ is a constant (\si{\second}).\\
  &$\eta = \frac{h_P A_P}{h_C A_C}$ is a constant (dimensionless).\\
  & The above equation applies as long as the water is in liquid form,
  $0<T_W<100^o\text{C}$, where $0^o\text{C}$ and $100^o\text{C}$ are the melting
  and boiling points of water, respectively (\aref{A_OpRange}, \aref{A_Pressure}).
  \\
  \hline
  Sources& Citation here \\
  \hline
  Ref.\ By & \iref{epcm}\\
  \hline
\end{tabular}
\end{minipage}\\

%~\newline

\subsubsection*{Derivation of ...}

\plt{The derivation shows how the IM is derived from the TMs/GDs.  In cases
  where the derivation cannot be described under the Description field, it will
  be necessary to include this subsection.}

\subsubsection{Input Data Constraints} \label{sec_DataConstraints}    

Table~\ref{TblInputVar} shows the data constraints on the input output
variables.  The column for physical constraints gives the physical limitations
on the range of values that can be taken by the variable.  The column for
software constraints restricts the range of inputs to reasonable values.  The
software constraints will be helpful in the design stage for picking suitable
algorithms.  The constraints are conservative, to give the user of the model the
flexibility to experiment with unusual situations.  The column of typical values
is intended to provide a feel for a common scenario.  The uncertainty column
provides an estimate of the confidence with which the physical quantities can be
measured.  This information would be part of the input if one were performing an
uncertainty quantification exercise.

The specification parameters in Table~\ref{TblInputVar} are listed in
Table~\ref{TblSpecParams}.

\begin{table}[!h]
  \caption{Input Variables} \label{TblInputVar}
  \renewcommand{\arraystretch}{1.2}
\noindent \begin{longtable*}{l l l l c} 
  \toprule
  \textbf{Var} & \textbf{Physical Constraints} & \textbf{Software Constraints} &
                             \textbf{Typical Value} & \textbf{Uncertainty}\\
  \midrule 
  $L$ & $L > 0$ & $L_{\text{min}} \leq L \leq L_{\text{max}}$ & 1.5 \si[per-mode=symbol] {\metre} & 10\%
  \\
  \bottomrule
\end{longtable*}
\end{table}

\noindent 
\begin{description}
\item[(*)] \plt{you might need to add some notes or clarifications}
\end{description}

\begin{table}[!h]
\caption{Specification Parameter Values} \label{TblSpecParams}
\renewcommand{\arraystretch}{1.2}
\noindent \begin{longtable*}{l l} 
  \toprule
  \textbf{Var} & \textbf{Value} \\
  \midrule 
  $L_\text{min}$ & 0.1 \si{\metre}\\
  \bottomrule
\end{longtable*}
\end{table}

\subsubsection{Properties of a Correct Solution} \label{sec_CorrectSolution}

\noindent
A correct solution must exhibit \plt{fill in the details}.  \plt{These
  properties are in addition to the stated requirements.  There is no need to
  repeat the requirements here.  These additional properties may not exist for
  every problem.  Examples include conservation laws (like conservation of
  energy or mass) and known constraints on outputs, which are usually summarized
  in tabular form.  A sample table is shown in Table~\ref{TblOutputVar}}

\begin{table}[!h]
\caption{Output Variables} \label{TblOutputVar}
\renewcommand{\arraystretch}{1.2}
\noindent \begin{longtable*}{l l} 
  \toprule
  \textbf{Var} & \textbf{Physical Constraints} \\
  \midrule 
  $T_W$ & $T_\text{init} \leq T_W \leq T_C$ (by~\aref{A_charge})
  \\
  \bottomrule
\end{longtable*}
\end{table}

\plt{This section is not for test cases or techniques for verification and
  validation.  Those topics will be addressed in the Verification and Validation
  plan.}
