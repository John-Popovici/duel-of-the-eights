\section*{Appendix --- Reflection}

%\wss{Not required for CAS 741}

The information in this section will be used to evaluate the team members on the
graduate attribute of Lifelong Learning.  

%\input{../Reflection.tex}

\begin{enumerate}
  \item What went well while writing this deliverable?
  
	\begin{itemize}
		\item The SRS document had a lot of work to be done in a lot of sections. In an effort to mitigate merge conflicts and better divide the workload, a main SRS.tex file was created which would reference and include other .tex files which contain the information for the sections. - John P.
	\end{itemize}  
  
  \item What pain points did you experience during this deliverable, and how did
  you resolve them?
  
	\begin{itemize}
		\item Since out project is solely a software without any hardware components or dependencies, the template was a little difficult to navigate. While we used the default SRS template as guidance, we made use of inspiration from the Volere SRS template for sections that would be more relevant to our own and removed sections that were not relevant from the default SRS template. - John P.
	\end{itemize}    
  
  \item How many of your requirements were inspired by speaking to your
  client(s) or their proxies (e.g. your peers, stakeholders, potential users)?
  
	\begin{itemize}
		\item Our primary stakeholders are those who would be interested in playing the game, so indirectly we are members of the stakeholder group as well as our supervisor. As such, through consensus and group discussion, game mechanics that would be interesting and enjoyable were drawn up which determine most of our functional requirements. - John P.
	\end{itemize}    
  
  \item Which of the courses you have taken, or are currently taking, will help
  your team to be successful with your capstone project.
  
	\begin{itemize}
		\item For the calculations of probabilities, SFWRENG 4E03 proved to be helpful and discrete time markov chains were used to determine probabilities in rolls and how many of a kind. - John P.
		\item In thinking how we are to organize the program in terms of architecture styles, SFWRENG 3A04 is helpful in providing examples and information on different architectures and their applications. - John P.
		\item Since our project is a game, there will be constant player input and output and interaction through an interface. As such, SFWRENG 4HC3 provided helpful information on the design of user interfaces and principles of good interface design. - John P.
	\end{itemize}    
  
  \item What knowledge and skills will the team collectively need to acquire to
  successfully complete this capstone project?  Examples of possible knowledge
  to acquire include domain specific knowledge from the domain of your
  application, or software engineering knowledge, mechatronics knowledge or
  computer science knowledge.  Skills may be related to technology, or writing,
  or presentation, or team management, etc.  You should look to identify at
  least one item for each team member.
  
	\begin{itemize}
		\item I will need to learn to develop in Godot, as that is the game engine we will be using, as well as learning C\# as a language. I have also already learnt more GitHub that before with branched, merging, issues, etc. Finally, I hope to further explore probability throughout this project and more advanced calculations such as DTMC. - John P.
	\end{itemize}    
  
  \item For each of the knowledge areas and skills identified in the previous
  question, what are at least two approaches to acquiring the knowledge or
  mastering the skill?  Of the identified approaches, which will each team
  member pursue, and why did they make this choice?
  
	\begin{itemize}
		\item In terms of learning Godot, there are many online tutorials that can be found and followed along, and all of us should be making use of this resource before we contribute to the project to become accustomed to the process. A second way to acquire experience in Godot is through making a small project on our own once we have followed the online tutorials. This would also be the case for C\# and the scripting that is available in Godot. - John P.
	\end{itemize}    
  
\end{enumerate}