\section{Requirements}

\subsection{Functional Requirements}

\begin{enumerate}[label=R\arabic*, start=1, left=0pt]

    \item \label{R1} The game shall support an online player vs player mode, where 2 players can play against each other.
    \begin{description}
        \item[Rationale:] Player vs player support is a central functionality for this game and one that enhances the overall enjoyment of the game by fostering a more social game dynamic.
    \end{description}

    \item \label{R2} The game shall handle score calculations using similar rules to standard Yahtzee under default game settings.
    \begin{description}
        \item[Rationale:] Score calculations that follow from a well established, and well balanced game will help users jump into the game faster and foster a more fair game experience.
    \end{description}

    \item \label{R3} The game shall simulate realistic physics for 3D dice rolls with true randomness on every roll that replicates accurately to all users.
    \begin{description}
        \item[Rationale:] It is important that players get to visually see the dice roll and feel that the outcome of the roll is not clearly predetermined in order to enhance the user experience and maintain continuity between the game and real life.
    \end{description}

    \item \label{R4} The game shall use the real outcome of a roll to get the values from the dice.
    \begin{description}
        \item[Rationale:] In order to get true randomness off a simulated dice roll, it makes the most sense to actually just simulate the roll and read the result rather than precalculating the results and forcing the roll to match the output.
    \end{description}

    \item \label{R5} The game shall support both regular six-sided dice and dice with more sides, such as octahedral dice.
    \begin{description}
        \item[Rationale:] By providing support for dice with more sides, we can increase the level of complexity of the game for users that are looking for a more interesting variation of the game.
    \end{description}

    \item \label{R6} The game shall implement a simultaneous turn based mechanism such that each player (or computer) takes a turn at the same time and then results are revealed simultaneously to each other at each dice roll.
    \begin{description}
        \item[Rationale:] This allows users to play together syncronously, while preserving the secrecy of each dice roll so that no player gets an advantage by knowing whether their opponent has rolled well or not.
    \end{description}

    \item \label{R7} The game shall allow players to pick which dice they would like to use for each roll, and which dice they would like to omit.
    \begin{description}
        \item[Rationale:] This is an important part of standard game rules, and also increases the range of strategic decision making that players get to have.
    \end{description}

    \item \label{R8} The game shall display some sort of user interface to display scores, number of rolls, time limits, state of dice, and player names.
    \begin{description}
        \item[Rationale:] Some form of user interface is necessary to convey important game information to the player.
    \end{description}

    \item \label{R9} The game shall provide controls for the player to modify the game settings to access unique variants of the game.
    \begin{description}
        \item[Rationale:] A core part of our team's vision for this game is that players should be able to customize their playing experience by altering things like; number of dice, type of dice used, time settings, and scoring methods.
    \end{description}

    \item \label{R10} The game shall provide presets for different game modes.
    \begin{description}
        \item[Rationale:] This should help new players get used to the game before exploring more unique setting configurations.
    \end{description}

%   REQUIREMENTS RELATING TO STRETCH GOALS LISTED BELOW:::::::
    \textit{**** The following requirements relate to our stretch goals ****}
    \item \label{R11} The game shall support a local player vs player mode, where 2 players can play against each other on the same computer.
    \begin{description}
        \item[Rationale:] Given that this is a turn based game it is perfectly possible for this game to be setup with 2 player functionality on one device. This simply increases the number of ways in which the game can be played.
    \end{description}

    \item \label{R12} The game shall support a player vs computer mode, where 1 player can play against another entity without needing to find a human match.
    \begin{description}
        \item[Rationale:] Player vs computer support is important for instances where a user may not be able to find a human match to play against.
    \end{description}

    \item \label{R13} The game shall implement some sort of algorithmic computer opponent for  player vs computer gameplay option.
    \begin{description}
        \item[Rationale:] For player vs computer to function, some sort of algorithmic opponent will be necessary to implement the computer player or else there will be no challenge for the human player.
    \end{description}

    \item \label{R14} The game shall implement online matchmaking.
    \begin{description}
        \item[Rationale:] For players that want to play the game against a real opponent but don't know anyone who is available to play against them, this is a great way for that player to find someone to play against.
    \end{description}

    \item \label{R15} The game shall provide the option to save specific game settings to be reused in future sessions.
    \begin{description}
        \item[Rationale:] For frequent players of the game, they may have a preferred custom variation of the game settings that they may wish to save rather than having to input the game settings on every new game.
    \end{description}

    \item \label{R16} The game shall display round statistics to each player at the end of every round.
    \begin{description}
        \item[Rationale:] Players will likely want to see some statistics at the end of the round to quantify how well they played.
    \end{description}

%   REQUIREMENTS ADDED DUE TO HAZARD ANALYSIS LISTED BELOW:::::::
    \textit{**** The following requirements relate to our hazard analysis ****}
    \item \label{R17} The game shall always show the correct current state accurately.
    \begin{description}
        \item[Rationale:] The player needs to be able to understand the current state of the game to understand their current standings, and strategic the next move. Similarly, the current state indicates what is the next action to be done by the user. This is extended to states outside of an active game such as game settings selection, match-up, and system settings selection.
    \end{description}

\end{enumerate}

\subsection{Non-Functional Requirements}

\begin{enumerate}[label=NFR\arabic*, start=1, left=0pt]

    \item \label{NFR1} \textbf{Performance} The game shall maintain a frame rate of at least 30 FPS at all times.
    \begin{description}
        \item[Rationale:] Players need the frame rate of the game to be high enough at all times to see what’s going on.
    \end{description}

    \item \label{NFR2} \textbf{Usability} The game shall implement a clear and easy to use/understand user interface.
    \begin{description}
        \item[Rationale:] Users should not struggle to figure out where controls are or how certain features work. This would create a barrier to entry for new players, and it would be best if players could pick up this game and its controls with little difficulty or time required.
    \end{description}

    \item \label{NFR3} \textbf{Portability} The game shall be supported on systems running Windows 10 or later.
    \begin{description}
        \item[Rationale:] Most PC users use Windows 10 or 11, so ensuring that the game ports well to these operating systems is an important requirement if we want to reach a large demographic of users.
    \end{description}

    \item \label{NFR4} \textbf{Reliability} Multiplayer games of Yahtzee should crash less than 1\% of the time.
    \begin{description}
        \item[Rationale:] Players need to feel that the game is reliable and should not need to be concerned about the game crashing in the middle of a round.
    \end{description}

    \item \label{NFR5} \textbf{Responsiveness} The game shall respond to inputs from the user within 500 milliseconds in the worst case.
    \begin{description}
        \item[Rationale:] This figure represents a minimum acceptable response. If a user needs to wait longer than this to see the result of their input, they may become frustrated and even try spamming the control.
    \end{description}

    \item \label{NFR6} \textbf{Modularity} The game’s codebase shall be modular, such that it is easily extendable and reusable, and allows for quick fixes to bugs that may occur.
    \begin{description}
        \item[Rationale:] Our team has minimum goals in mind for what this game must be upon release, but to be able to continue updating the game to meet stretch goals, our codebase should be designed with modularity in mind to streamline the development process.
    \end{description}

    \item \label{NFR7} \textbf{Efficiency} The game shall be optimized to run on lower end systems which might have minimal CPU and GPU resources.
    \begin{description}
        \item[Rationale:] The game that our team is developing is not such a graphically or computationally intensive game that it should require high-end PCs to run. Additionally, being able to run the game on lower-end systems will allow our game to reach a broader audience.
    \end{description}

    \item \label{NFR8} \textbf{Enjoyability} The game shall be found enjoyable by at least 75\% of users.
    \begin{description}
        \item[Rationale:] The game should appeal to the majority of its users, because if it doesn't then the game serves no particular purpose.
    \end{description}

    \item \label{NFR9} \textbf{Appearance} The game shall maintain a consistent UI style and 3D visual style throughout all in game views.
    \begin{description}
        \item[Rationale:] A style that evokes some sense of continuity throughout the game is important to help users get used to the layout faster and to give the appearance of a more professionally developed gaming experience.
    \end{description}

%   REQUIREMENTS RELATING TO STRETCH GOALS LISTED BELOW:::::::
    \textit{**** The following requirements relate to our stretch goals ****}
    \item \label{NFR10} \textbf{Portability} The game shall be supported on MacOS devices.
    \begin{description}
        \item[Rationale:] There are many users who use MacOS devices and would be a good target audience for this game.
    \end{description}

\end{enumerate}