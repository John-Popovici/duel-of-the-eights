\section{Reference Material}

%This section records information for easy reference.\\

This section has been simplified due to the simplistic nature of our problem space. No unit symbols are used in our document and anywhere a unit is used, it is stated. Additionally all abbreviations and acronyms are explained where they first appear in the document and are from there on out self-explanatory and contextually understandable. There are also no mathematical formulas or mathematical notation within our software requirements specification document. This document was compiled for the SFWRENG 4G06 course, the software engineering capstone course at McMaster University.

\subsection{Symbols, Abbreviations, and Acronyms}

\begin{table}[H]
\begin{tabular}{l l} 
  \toprule		
  \textbf{symbol} & \textbf{description}\\
  \midrule
  SFWRENG 4G06 & Program Capstone Course\\
  SRS & Software Requirements Specification\\
  X in AdXrB & number of dice faces\\
  A in AdXrB & number of dice\\
  B in AdXrB & number of dice rolls\\
  GSX & Goal statement number X\\
  SGX & Stretch goal number X\\
  RX & Functional Requirement number X\\
  NFRX & Non-Functional Requirement number X\\
  LCX & Likely Change number X\\
  UCX & Unlikely Change number X\\
  UI & User Interface\\
  CPU & Central Processing Unit\\
  GPU & Graphics Processing Unit\\
  PC & Personal Computer\\
  UI & User Interface\\
  UX & User Experience\\
  3D & Three dimensional\\
  OS & Operating System\\
  ms & milliseconds\\
  FPS & Frames Per Second\\
  PvP & Player versus Player\\
  GDPR & General Data Protection Regulation\\
  CCPA & California Consumer Privacy Act\\
  COPPA & Children’s Online Privacy Protection Act\\
  IEEE & Institute of Electrical and Electronics Engineers\\
  LAN & Local Area Network\\
  RFC & Request For Comments\\
  ISO & International Organization for Standardization\\
  IEC & International Electrotechnical Commission\\
  EULA & End-User License Agreement\\
  ESRB & Entertainment Software Rating Board\\
  PEGI & Pan European Game Information\\
  WCAG & Web Content Accessibility Guidelines\\
  

  \bottomrule
  
\end{tabular}\\
\caption{Table of Symbols, Abbreviations, and Acronyms}
\end{table}
% \plt{Add any other abbreviations or acronyms that you add}









\iffalse
\subsection{Table of Units}

Throughout this document SI (Syst\`{e}me International d'Unit\'{e}s) is employed
as the unit system.  In addition to the basic units, several derived units are
used as described below.  For each unit, the symbol is given followed by a
description of the unit and the SI name.
~\newline

\renewcommand{\arraystretch}{1.2}
%\begin{table}[ht]
  \noindent \begin{tabular}{l l l} 
    \toprule		
    \textbf{symbol} & \textbf{unit} & \textbf{SI}\\
    \midrule 
    \si{\metre} & length & metre\\
    \si{\kilogram} & mass	& kilogram\\
    \si{\second} & time & second\\
    \si{\celsius} & temperature & centigrade\\
    \si{\joule} & energy & joule\\
    \si{\watt} & power & watt (W = \si{\joule\per\second})\\
    \bottomrule
  \end{tabular}
  %	\caption{Provide a caption}
%\end{table}

\plt{Only include the units that your SRS actually uses.}

\plt{Derived units, like newtons, pascal, etc, should show their derivation
    (the units they are derived from) if their constituent units are in the
    table of units (that is, if the units they are derived from are used in the
    document).  For instance, the derivation of pascals as
    $\si{\pascal}=\si{\newton\per\square\meter}$ is shown if newtons and m are
    both in the table.  The derivations of newtons would not be shown if kg and
    s are not both in the table.}

\plt{The symbol for units named after people use capital letters, but the name
  of the unit itself uses lower case.  For instance, pascals use the symbol Pa,
  watts use the symbol W, teslas use the symbol T, newtons use the symbol N,
  etc.  The one exception to this is degree Celsius.  Details on writing metric
  units can be found on the 
  \href{https://www.nist.gov/pml/weights-and-measures/writing-metric-units}
  {NIST} web-page.}

\subsection{Table of Symbols}

The table that follows summarizes the symbols used in this document along with
their units.  The choice of symbols was made to be consistent with the heat
transfer literature and with existing documentation for solar water heating
systems.  The symbols are listed in alphabetical order.

\renewcommand{\arraystretch}{1.2}
%\noindent \begin{tabularx}{1.0\textwidth}{l l X}
\noindent \begin{longtable*}{l l p{12cm}} \toprule
\textbf{symbol} & \textbf{unit} & \textbf{description}\\
\midrule 
$A_C$ & \si[per-mode=symbol] {\square\metre} & coil surface area
\\
$A_\text{in}$ & \si[per-mode=symbol] {\square\metre} & surface area over 
which heat is transferred in
\\ 
\bottomrule
\end{longtable*}
\plt{Use your problems actual symbols.  The si package is a good idea to use for
  units.}

\subsection{Abbreviations and Acronyms}

\renewcommand{\arraystretch}{1.2}
\begin{tabular}{l l} 
  \toprule		
  \textbf{symbol} & \textbf{description}\\
  \midrule 
  A & Assumption\\
  DD & Data Definition\\
  GD & General Definition\\
  GS & Goal Statement\\
  IM & Instance Model\\
  LC & Likely Change\\
  PS & Physical System Description\\
  R & Requirement\\
  SRS & Software Requirements Specification\\
  \progname{} & \plt{put an expanded version of your program name here (as appropriate)}\\
  TM & Theoretical Model\\
  \bottomrule
\end{tabular}\\

\plt{Add any other abbreviations or acronyms that you add}

\subsection{Mathematical Notation}

\plt{This section is optional, but should be included for projects that make use
  of notation to convey mathematical information.  For instance, if typographic
  conventions (like bold face font) are used to distinguish matrices, this
  should be stated here.  If symbols are used to show mathematical operations,
  these should be summarized here.  In some cases the easiest way to summarize
  the notation is to point to a text or other source that explains the
  notation.}

\plt{This section was added to the template because some students use very
  domain specific notation.  This notation will not be readily understandable to
  people outside of your domain.  It should be explained.}
\fi