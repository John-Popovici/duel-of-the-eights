\section{Standards, Codes, Legal, and Regulatory Factors}    

Developing an online multiplayer game requires adherence to various standards, codes, legal, and regulatory factors to ensure a safe, fair, and compliant gaming experience. The following standards are identified for the multiplayer aspect of the game, especially in the context of player-versus-player (PvP) matches:

\subsection*{1. Data Privacy and Protection}
\begin{itemize}
    \item \textbf{GDPR (General Data Protection Regulation)}: Given that the game may involve players from the European Union, compliance with GDPR is essential. This includes ensuring user consent for data collection, providing clear privacy policies, and safeguarding player data.
    \item \textbf{CCPA (California Consumer Privacy Act)}: If players are based in California, the game must comply with CCPA to give users control over their personal information.
    \item \textbf{COPPA (Children's Online Privacy Protection Act)}: If the game targets children under 13, compliance with COPPA is mandatory to protect children’s privacy.
\end{itemize}

\subsection*{2. Network and Online Standards}
\begin{itemize}
    \item \textbf{IEEE 802.11 (Wi-Fi Standards)}: Ensuring reliable local area network (LAN) connectivity follows IEEE 802.11 standards to provide consistent communication quality.
    \item \textbf{WebSockets (RFC 6455)}: The use of WebSockets for real-time communication between players must comply with RFC 6455 to ensure reliable, secure data exchange.
\end{itemize}

\subsection*{3. Game Fairness and Anti-Cheating Measures}
\begin{itemize}
    \item \textbf{ISO/IEC 27001 (Information Security Management)}: Implementing practices from ISO/IEC 27001 helps in managing game security, reducing the risk of cheating, and protecting the game from malicious activities.
    \item \textbf{Fair Play Guidelines}: To ensure fair play in PvP matches, the game should adopt fair matchmaking and anti-cheating measures, such as monitoring unusual behavior and implementing server-side verification of actions.
\end{itemize}

\subsection*{4. Legal Considerations for Online Play}
\begin{itemize}
    \item \textbf{Terms of Service and End-User License Agreement (EULA)}: A clear EULA must be provided, detailing acceptable behavior, limitations of liability, and consequences for misuse or cheating.
    \item \textbf{Content Rating Standards}: The game must be rated appropriately, such as following the \textbf{ESRB (Entertainment Software Rating Board)} or \textbf{PEGI (Pan European Game Information)} standards, to ensure suitability for the intended audience.
\end{itemize}

\subsection*{5. Accessibility Standards}
\begin{itemize}
    \item \textbf{WCAG (Web Content Accessibility Guidelines) 2.1}: For an inclusive experience, the game’s online features and user interface should follow WCAG 2.1 guidelines, ensuring accessibility for players with disabilities.
\end{itemize}

\subsection*{6. Intellectual Property and Copyright}
\begin{itemize}
    \item \textbf{Copyright Compliance}: The game must avoid unauthorized use of copyrighted material, including sound effects, graphics, and other assets. Original content must be used, or appropriate licenses must be obtained.
    \item \textbf{Trademark Considerations}: Any use of trademarks must be properly authorized to avoid infringement issues.
\end{itemize}

\subsection*{7. Regulatory Requirements for Online Play}
\begin{itemize}
    \item \textbf{Online Gambling Regulations}: If any aspect of the game involves virtual currency or random rewards, it may fall under online gambling regulations in certain jurisdictions. The game must be reviewed for compliance with applicable laws to avoid any unintended legal issues.
    \item \textbf{Consumer Protection Laws}: Compliance with consumer protection laws is crucial, ensuring transparency in in-game purchases and providing users with refund policies.
\end{itemize}

By adhering to these standards, codes, and legal factors, the development of the online multiplayer game will ensure a fair, compliant, and enjoyable experience for all players, while minimizing legal and regulatory risks.

\section{Standards Compliance Roadmap}

The following compliance roadmap outlines when each of the identified standards will be adhered to during the development of the online multiplayer game:

\begin{itemize}
\item \textbf{Data Privacy and Protection (GDPR, CCPA, COPPA)}: Compliance will be met by the Final Demo.
\item \textbf{Network and Online Standards (IEEE 802.11)}: Compliance will be met by the Final Demo.
\item \textbf{Network and Online Standards (WebSockets)}: Compliance will be addressed post-capstone based on the game's scale.
\item \textbf{Game Fairness and Anti-Cheating Measures (ISO/IEC 27001)}: Compliance will be addressed post-capstone based on the game's scale.
\item \textbf{Game Fairness and Anti-Cheating Measures (Fair Play Guidelines)}: Compliance will be met by the Final Demo.
\item \textbf{Legal Considerations for Online Play (Terms of Service, EULA, Content Rating Standards)}: Compliance will be met by the Final Demo.
\item \textbf{Accessibility Standards (WCAG 2.1)}: Compliance will be addressed post-capstone based on the game's scale.
\item \textbf{Intellectual Property and Copyright (Copyright Compliance, Trademark Considerations)}: Compliance will be continuously verified, with final verification by the Final Demo.
\item \textbf{Regulatory Requirements for Online Play (Online Gambling Regulations)}: Compliance may be required post-capstone based on the direction of the game and will be evaluated accordingly.
\item \textbf{Regulatory Requirements for Online Play (Consumer Protection Laws)}: Compliance will be met by the Final Demo if the game direction leads to the introduction of in-game purchases.
\end{itemize}